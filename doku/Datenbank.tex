\section{Datenbank}

\Diskussionspunkt{Beginn Einlesen in DB}
Verschiedene Arten von Datenbanken:
\begin{itemize}
\item relationale DB
\item hierarchisches Datenmodell
\item Netzwerkdatenmodell
\item Objekt relationale Datenbank
\end{itemize}

Das relationale Datenmodell ist das weit verbreitetste Modell, dass hierarchische wegen der beschränkten Anwendbarkeit kaum noch vorhanden.

Vorgehen Datenbankentwicklung:

\begin{itemize}
\item Externe Phase (Ermittlung der Informationsstruktur)
\item Konzeptionelle Phase (ER-Modell)
\item Logische Phase (relationales Datenmodell)
\item Physische Phase (Erstellung des Datenmodell)
\end{itemize}

Punkte zur Überlegung neuer Datenbankstruktur:
\begin{itemize}
\item Welche Daten?
\item In welchem Intervall?
\item Welche Tabellen? (Welche Daten zusammen?, eine grosse Tabelle?)
\item Tabellennamen?
\item 
\end{itemize}

Datentypen angeschaut auf w3schoolss:
\begin{itemize}
\item Welcher Datumstyp?

\end{itemize}

Einlesen DB


-Unterschied MariaDB und MySQL?
-Wieviele Daten werden gespeichert?
Vor der neukonzipierung werden täglich 1440 Datensätze gespeichert. Das bedeutet jede Minute einen Datensatz. Ein Datensatz beinhaltet 65 einträge. Die gesamte relevante Datenbank igwetter_wettertest benötigt 323.17  Mb. Die Tabelle wx_data benötigt stand 1.3.18 311.94Mb, daraus erfolgt das ein Datensatz ca 0.025 Mb benötigt 
-Wieviele Daten können gespeichert werden?
Die igWetter hat bei Hostpoint einen Server mit 50 GB speicherplatz, rechnet man die vorherdehenden Zahlen hoch mit dem zur Verfügung stehenden Speicherplatz, hat es genügend Platz für die kommenden 45 Jahren. 

https://entwickler.de/online/datenbanken/datenbanken-grundlagen-und-entwurf-115676.html
\Diskussionspunkt{Ende Einlesen in DB}
