\section{Datenbank}

\Diskussionspunkt{Beginn Einlesen in DB}
Verschiedene Arten von Datenbanken:
\begin{itemize}
\item relationale DB
\item hierarchisches Datenmodell
\item Netzwerkdatenmodell
\item Objekt relationale Datenbank
\end{itemize}

Das relationale Datenmodell ist das weit verbreitetste Modell, dass hierarchische wegen der beschränkten Anwendbarkeit kaum noch vorhanden.

Vorgehen Datenbankentwicklung:

\begin{itemize}
\item Externe Phase (Ermittlung der Informationsstruktur)
\item Konzeptionelle Phase (ER-Modell)
\item Logische Phase (relationales Datenmodell)
\item Physische Phase (Erstellung des Datenmodell)
\end{itemize}

Punkte zur Überlegung neuer Datenbankstruktur:
\begin{itemize}
\item Welche Daten?
\item In welchem Intervall?
\item Welche Tabellen? (Welche Daten zusammen?, eine grosse Tabelle?)
\item Tabellennamen?
\item 
\end{itemize}

Datentypen angeschaut auf w3schoolss:
\begin{itemize}
\item Welcher Datumstyp?

\end{itemize}


-Unterschied MariaDB und MySQL?
-Wieviele Daten werden gespeichert?
Vor der neukonzipierung werden täglich 1440 Datensätze gespeichert. Das bedeutet jede Minute einen Datensatz. Ein Datensatz beinhaltet 65 einträge. Die gesamte relevante Datenbank igwetter_wettertest benötigt 323.17  Mb. Die Tabelle wx_data benötigt stand 1.3.18 311.94Mb, daraus erfolgt das ein Datensatz ca 0.025 Mb benötigt 
-Wieviele Daten können gespeichert werden?
Die igWetter hat bei Hostpoint einen Server mit 50 GB speicherplatz, rechnet man die vorherdehenden Zahlen hoch mit dem zur Verfügung stehenden Speicherplatz, hat es genügend Platz für die kommenden 45 Jahren. 

https://entwickler.de/online/datenbanken/datenbanken-grundlagen-und-entwurf-115676.html
\Diskussionspunkt{Ende Einlesen in DB}
\Diskussionspunkt {Beginn Einlesen DB Sicherheit}

Bei der Recherche nach Datenbanksicherheit taucht immer wieder das Wort Injection auf. Laut den OWASP top 10, eine Liste welche die wichtigsten Schwachstellen aufzeigt, ist die SQL injection in 2017 auf dem Platz 1. Was ist den eigentlich SQL Injection? SQL injection ist eine Methode eine Datenbankabfrage so zu manipulieren, dass der Angreifer im schlimmsten Fall auf die gespeicherten Daten des Administators kommt. Ein anderes beispiel wäre, dass der Angreifer an die Daten der Benutzer eines Online-Shops mit Kreditkartendaten oder ähnlichen sensitiven Daten kommt.
Weitere Fragen die auftauchen bei der Suche nach Datenbanksicherheit sind:
\begin{itemize}
\item Was für Arten von Daten beherbergt die Datenbank?
\item Hat es sensitive Daten?
\item Ist die Datenbank überhaupt ein potentielles Angriffsziel?
\item Wer sind die Benutzer der Datenbank?
\end{itemize}
Neben dem Schutz vor potentiellen Angriffen ist auch der Schutz vor Datenverlust wichtig. Dieser Schutz kann sehr einfach durch ein Backup der Datenbank umgesetzt werden. Jedoch stellen sich auch hier folgende Fragen:
\begin{itemize}
\item Welche Daten sind wichtig?
\item Wie wird das Backup umgesetzt?
\item Wie oft wird ein Backup gemacht?
\end{itemize}

\Diskussionspunkt{Ende Einlesen DB Sicherheit}
\Diskussionspunkt{Beginn Konzept DB Sicherheit}
Um ein Konzept zu erstellen, werden die bei der Recherche aufgekommenen Fragen beantwortet. Die Datenbank der Wetterstation beinhaltet die Wetterdaten jeder Minute des Tages, des Weiteren erhält sich auch die historischen Daten der Wetterstation. Sensitive Daten sind in der Datenbank, welche bearbeitet wird nicht vorhanden, trotzdem sollte eine gewisse Grundsicherheit der Datenbank gewährleistet werden, damit die Daten nicht manipuliert bzw. die Datenbank nicht anderweitig benutzt werden kann. Die Datenbank stellt nur in dieser hinsicht ein potentielles Ziel dar. Benutzer der Datenbank hat es im eigentlichen Sinne nur zwei. Das ist der Administrator der Seite, sowie die Webseite. Jedoch sind die Benutzer der Seite auch Benutzer der Datenbank. Doch was haben diese direkt mit der Datenbank zu tun? Auf der zukünftigen Seite der historischen Daten, sollen die Benutzer entscheiden können von wann bis wann die Daten angezeigt werden sollen. Somit müssen diese eine Abfrage tätigen. Diese gilt es gegen SQL-injection zu sichern.

Um die Datenbank auch gegen einen allfälligen Datenverlust zu sichern ist ein Backup von wichtiger Bedeutung. Im Grunde sind alle Daten welche die Wetterstation ablegt von Wichtigkeit. Dabei muss aber entschieden werden, ob ein tägliches Backup Sinn machen würde. Bezüglich Speicherplatz und Aufwand, da die Wetterstation von einem Verein betrieben wird, ist es wichtig den Aufwand mit dem Ertrag zu vergleichen. Bis zur  Umsetzung der neuen Struktur und Backup Möglichkeiten wurden \Diskussionspunkt{x. Anzahl Updates pro Jahr} erstellt. Diese wurden jedoch auf Hostpoint selber gespeichert. Was bei einem Ausfall des Providers keinen Mehrwert hat, jedoch einen Mehrwert wenn nur die Datenbank selber versagt. Deswegen ist es wichtig auch ein "externes" Backup zu erstellen. Es wird empfohlen ein wöchentliches Backup zu erstellen, dies aufgrund der Datenmengen. Fehlt eine Woche aufgrund eines Ausfalls ist der schaden geringer als ein monatliches oder jährliches Update. Das update wird mittels eines cronejobs, welches ein Backup-Script ausführt, auf dem Server durchgeführt. Die Daten werden anschliessend in einer Cloud nach Wahl hochgeladen. Zusätzlich wird das Backup auch auf dem gemieteten Server von Hostpoint gespeichert um einen schnelleren Zugriff zu gewährleisten. Um nicht zu viele Daten "unnötig" zu speichern, wird empfohlen die Daten der vorhergehenden Woche zu überspielen.    


