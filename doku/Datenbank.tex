%%%%%%%%%%%%%%%%%%%%%%%%%%%%%%%%%%%
%%  Datenbank
%%%%%%%%%%%%%%%%%%%%%%%%%%%%%%%%%%%
\section{Analyse back end}
Der Kombi-Wettert-Tansmitter sendet seine Daten über eine RS-484 Schnittstelle. Die Software \textit{WeatherDisplay} verarbeitet diese Daten und erstellt unterschiedliche Text-Files, die anschliessend von andere Anwendungen, wie zum Beispiel \textit{WeatherDisplay Live}, verwendet werden. Zusätzlich sendet \textit{WeatherDisplay} einmal pro Minute sämtliche Werte an die MySQL-Datenbank.

\begin{figure}[h!]
	\centering
	\includegraphics[width=1\linewidth]{img/datenfluss}
	\caption{Datenfluss vom Sensor bis zu Webseite}
	\label{img:datenfluss}
\end{figure}



% ################################
% Text-Files
% ################################
\subsection{Schnittstelle zum Kombi-Wetter-Transmitter}
Die Software \textit{WeatherDisplay} \footnote{ \url{http://www.weather-display.com}} dient als Schnittstelle zum Kombi-Wetter-Transmitter. Der Auftraggeber möchte diese Software beibehalten. Interessant sind deshalb die Daten, die  \textit{WeatherDisplay} liefert. Dies sind vier verschiedene Textfiles, deren Inhalt in Tabelle~\ref{table:text-files}) genauer erläutet sind sowie ein minütlicher Eintrag sämtlicher Messwerte in die Datenbank.
\newline

\begin{table}[h!]
\centering
\begin{tabular}{|l|l|l|l|l|}
\hline
 Name			&  Anzahl Daten	& 	Zeitspanne  		& 	Intervall			\\ \hline
 Clientraw.txt 		&  174			&  	10min/20h zurück 	& 	5 Sekunden 		\\ \hline
 Clientrawextra.txt	&  767 			&  	24h zurück 		& 	1 Stunde 			\\ \hline
 Clientrawdaily.txt 	&  442 			&  	30 Tage zurück 	&  	1 Tag 			\\ \hline
 Clientrawhour.txt	&  673			&  	1h zurück 			& 	1 Minute 			\\ \hline
\end{tabular}
\caption{Von WeatherDisplay erstellte Text-Files}
\label{table:text-files}
\end{table}

\noindent
%\subsection*{Problem}
Der Nachteil ist, dass der Inhalt der Text-Dateien nicht angepasst werden kann. Das bedeutet insbesondere, dass der Intervall der Messresultate nicht verändert werden kann. Zudem sind in diesen Text-Files nur die Daten des Kombi-Wetter-Transmitters aufgeführt. Zusatzsensoren wie Pegel und Wassertemperatur sind nicht enthalten.
\newline

\noindent
%\subsection*{Lösungsansatz}
Während der Bachelor-Arbeit soll geprüft werden in wie weit die Daten aus den Text-Files für die Erstellung der neuen Anzeige-Elemente verwendet werden kann. 


% Abbildung (A3)
%\afterpage{ 
%\clearpage
%\KOMAoptions{paper=a3, paper=landscape} 
%\recalctypearea
%\newgeometry{left=30mm,right=30mm,top=30mm,bottom=50mm}
%
%\begin{figure}[h!p]
%	\centering
%	\includegraphics[width=2.5\linewidth]{img/Sequenzdiagramm_Wetter}
%	\caption{Ablauf von der Datenerfassung bis zur Anzeige}
%	\label{img:Sequenzdiagramm}
%\end{figure}
%
%
%\restoregeometry
%\KOMAoptions{paper=A4,pagesize}
%\recalctypearea
%}

% ################################
%Historische Daten (Datenbank)
% ################################
\subsection{Speicherung historischer Daten}
Um die Daten der Wetterstation zu speichern wird eine MySQL-Datenbank verwendet. Diese besteht aus mehreren Tabellen, die sich in Inhalt, Häufigkeit und Zeitraum unterscheiden, wie in Tabelle~\ref{table:db-tables} dargestellt. Auf der Webseite ist bereits eine Unterseite reserviert, um die historische Daten abfragen zu können. Bisher ist diese Funktion nicht implementiert.

\begin{table}[h!]
\resizebox{\textwidth}{!}{%
\begin{tabular}{|l|l|l|l|l|}
\hline
 Tabelle		&  Inhalt								& 	von  			& 	bis 			& 	Intervall\\ \hline
 tblgestern 	&  min und max Werte					&  	25.02.2005 	& 	14.07.2012 	& 	24h \\ \hline
 tblwellen		&  Pegel, Wellenhöhe, und Wassertemperatur 	&  	29.10.2013 	& 	28.01.2014  	& 	10min \\ \hline
 tblwind 		&  Windgeschwindigkeit- und Windrichtung 	&  	29.10.2013 	& 	28.01.2014 	&  	1min \\ \hline
 wx-data 		&  all von WXT gemessenen Werte			&  	25.02.2015 	&	heute  		& 	1min \\ \hline
 wx-pegel  	&  \multicolumn{4}{l|}{enthält keine Daten, da der Pegelsensor defekt ist} \\ \hline
\end{tabular}%
}
\caption{Vorhandene Daten in der Datenbank}
\label{table:db-tables}
\end{table}

\subsection*{Problem}
Um auf die Daten zugreifen zu können, müssen mehrere Tabellen abgefragt werden, was die Query und die Anzeige der Daten erschwert. Zudem fehlt ein Konzept welche Daten wo, wie und wie häufig gespeichert werden sollen.

\subsection*{Lösungsansatz}
Während der Bachelor-Arbeit soll ein Konzept erarbeitet werden, wie die Daten möglichst einfach in der Datenbank gespeichert und abgerufen werden können.


% ################################
% Datenmanagment und Datensicherung
% ################################
\subsection{Datenmanagement und Datensicherung}
Täglich fallen 93600 Datenpunkte an, diese Daten werden alle seit 2015 gespeichert und nicht ausgedünnt und für die Datenbank gibt es kein Backup beispielsweise auf eine externe Harddisk. Zusätzlich zur Datenbank werden auch die erwähnten clientraw Dateien benutzt um Anzeigen zu erstellen. Von diesen wird jährlich ein Backup erstellt und beim Hosting-service selber gespeichert.

\subsection*{Problem}
Ein weiterer Punkt sind die riesigen Mengen an Daten, diese verzögert eine Abfrage in der Datenbank enorm. zum anderen werden die seit dem erstellten Daten nicht auch in die "historische" Tabelle tblgestern abgelegt.


\subsection*{Lösungsansatz}
Hierfür wird vorgeschlagen die Datenbank igwetter openfile64Light so zu belassen, da diese für das CMS zuständig ist. Es wird eine neue Datenbank erstellt welche klarer strukturiert wird, hierbei sollten neue Tabellen entstehenen, wobei Daten ausgedünnt werden können um keinen unnötigen Speicherplatz zu belasten. Die Haupttabelle soll alle aktuelle Daten enthalten. Zusätzlich soll Tabelle erstellt werden mit zukünftigen und vorhandenen historischen Daten. Um die Übersichtlichkeit zu gewähren wird eine Tabelle mit Maximal sowie Minimal Daten erstellt. Die Zeitabstände, in der die Daten gelöscht bzw. zu historischen Daten werden, müssen noch mit den Mitgliedern der IG-Wetter abgesprochen werden. Ein weiterer Punkt auf der Liste sollten die zukünftigen Backups sein, d.h. diese sollten nicht als .txt sonder auch als .sql File gespeichert sein damit im Falle eines Datenverlustes die Datenbank einfach wiederherzustellen ist.



