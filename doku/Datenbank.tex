\section{Datenmanagment mit Eigenheiten}

% ################################
% Datenfluss (Datenbanken und Textfiles)
% ################################
\subsection{Datenfluss: Datenbanken und Text-Files}
\Diskussionspunkt{Übersicht: in welcher DB ist welche Tabelle}
\Diskussionspunkt{Bild des Datenflusses inkl. DB und Text-Files}



% Datenbanken
Für die Webseite und die Wetterstation, hat es vier verschiedene Datenbanken. Diese werden in diesem Kapitel einzeln behandelt und erklärt wie Sie zusammenhängen bzw. welche Rolle sie für die Webseite spielen. Die vier Datenbanken heissen:
\begin{itemize}  
\item igwetter meteotmpl
\item igwetter wettertest
\item gwetter wp0
\item gwetter openfile64Light
\end{itemize}

Die Datenbank igwetter meteotmpl beinhaltet alle relevanten Datenpunkte, d.h. von der Temperatur bis zur Windrichtung. In dieser Datenbank sind jedoch keine Daten vorhanden und dient momentan nur als Template.\\
Die zweite Datenbank welche nicht aktiv ist, heisst igwetter wp0, diese wurde für eine kurze Zeit für eine Wordpress-Seite benutzt.\\


Die dritte Datenbank ist die igwetter wettertest. Diese ist im Gegensatz zu den vorherigen beiden Datenbanken im Gebrauch. In der Tabelle wx data sind die Daten ab dem 25.02.2015 bis zum jetzigen Zeitpunkt gespeichert. Daten zwischen dem 14.07.2012 und 25.02.2015 sind nicht in der Datenbank hinterlegt. Vor diesem Zeitpunkt bis zum 25.02.2005 sind die täglichen Minimum, sowie Maximum Daten in der Tabelle tblgestern gespeichert.\\
Anders als bei den vorherigen Datenbanken hat die igwetter opfile64Light Datenbank eine Funktion für die Webseite. In diesem Fall ist die ganze Webseite abhängig von dieser, denn das CMS basiert auf einer Datenbank. Dies wurde im Kapitel Wie und wo werden Applikationen erstellt bereits erläutert. Für die BA, sowie das Fachmodul ist nur die Tabelle applications interessant, denn dort werden die Applikationen unter einem bestimmten Namen abgespeichert und aufgerufen. Die Webseite weiss dann welche Datei sie öffnen muss, damit die Applikation läuft.\\


\Diskussionspunkt{Datenbankabbildung, besprechen}\\

Die Daten, welche von der Wetterstation an einen Server in der Uni Liechstein gesendet werden, werden über die Software Weather-Display direkt in die Datenbank ig wettertest Tabelle wxdata geschrieben. Diese Daten werden hier im Minutentakt eingelesen und in der tabelle wx data gespeichert. 


% Text Files
Die Weahter-Display Software speichert zusätzlich alle Daten, welche von der Wetterstation sind, in ein .txt File. Das Weather-Display live nimmt die Daten anschliessend direkt aus diesem File um die Anzeigen zu erstellen. Die Daten werden in 4 verschiedenen Textfiles mit unterschiedlichen Funktionen gespeichert.
\begin{itemize}  
\item clientraw.txt
\item clientrawextra.txt
\item clientrawhour.txt
\item clientrawdaily.txt
\end{itemize}
\Diskussionspunkt{lieber ein Bild aller Files als eine Aufzählung}
\Diskussionspunkt{Warum kann nicht auf die txt-Files verzichtet werden}
\Diskussionspunkt{API für Badi (Luft- und Wassertemperatur}


Die clientraw.txt Datei enthält die aktuellen Wetterdaten der Station. Der Intervall der Aktualisierung dieser Datei wird in der Datei wdlconfig.xml eingestellt. Im Fall von Arbon wird die Datei alle 5 Sekunden aktualisiert.  Hier werden auch die restlichen Parameter bzw. Einheiten in der die Daten gespeichert werden sollen eingestellt. Die clientrawextra.txt enthält die historischen Extremwerte. Die Datei clientrawhour.txt enthält die aufgezeichneten Daten der letzten Stunde im Minutentakt.\cite{WeatherDisplay} \\


% ################################
% Datenmanagment
% ################################
\subsection{Datenmanagement}
\Diskussionspunkt{Ausdünnung der Daten; wieviel Daten fallen pro Tag an?}

% ################################
% Datensicherung
% ################################
\subsection{Datensicherung}


% ################################
% API
% ################################
\subsection{API}
\Diskussionspunkt{REST-Servie?}

Das Problem hierbei ist folgendes schaut man die Datenbanken zum jetzigen Zeitpunkt an, scheint es chaotisch zu sein. Im Grunde werden nur die igwetter openfile64Light und die igwetter wettertest Datenbank benutzt. Da es mehrere Datenbanken gibt,  ist es auf dem ersten Blick nicht sichtbar, was wo gemacht wird und welche Datenbank für wofür zuständig ist. Des weiteren wird nicht nur eine Datenbank sondern auch .txt Files benutzt um ein Backup zu erstellen, sowie die aktuellen Daten zu speichern. Die historischen Daten in der Datenbank sind nicht vollständig, zum einen gibt es für den gesagten Zeitraum zwischen 2012 und 2015 keine Daten und zum anderen werden die seit dem erstellten Daten nicht auch in die "historische", tblgestern, Tabelle abgelegt. \\


In der Bachelorarbeit soll dieses Problem folgendermassen gelöst werden. Bei der Datenbank sollte den Mitglieder der IG-Wetter Arbon auf dem ersten Blick klar sein, was wofür benutzt wird. Hierfür wird vorgeschlagen die Datenbank igwetter openfile64Light so zu belassen, da diese für das CMS zuständig ist. Es wird eine neue Datenbank erstellt welche klarer strukturiert wird, hierbei sollten neue Tabellen entstehenen, wobei Daten gelöscht werden können um unnötigen Speicherplatz nicht zu belasten. Die "Haupttabelle" soll alle aktuelle Daten enthalten. Zusätzlich soll Tabelle erstellt werden mit zukünftigen und vorhandenen historischen Daten. Um die Übersichtlichkeit zu gewähren wird eine Tabelle mit Maximal sowie Minimal Daten erstellt. Die Zeitabstände, in der die Daten gelöscht bzw. zu historischen Daten werden, müssen noch mit den Mitgliedern der IG-Wetter abgesprochen werden. Ein weiterer Punkt auf der Liste sollten die zukünftigen Backups sein, d.h. diese sollten nicht als .txt sonder auch als .sql File gespeichert sein damit im Falle eines Datenverlustes die Datenbank einfach wiederherzustellen ist.
