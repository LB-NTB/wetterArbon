
\documentclass[10pt]{article}
\usepackage[utf8]{inputenc} % this is needed for umlauts
\usepackage[ngerman]{babel} % this is needed for umlauts
\usepackage[T1]{fontenc}    % this is needed for correct output of umlauts in pdf
\usepackage{geometry}
\geometry{a4paper, top=20mm, left=20mm, right=20mm, bottom=20mm,
headsep=10mm, footskip=12mm}
\usepackage{multicol}
\setlength{\columnsep}{0.5cm}
 
\begin{document}
\begin{multicols}{2}
[
%  Title and authors
    \begin{center}
      {\huge\sffamily \Large Erweiterung und Modernisierung der Webapplikation der Wetterstation Arbon}\\
       \vspace{2ex}
       \textsc{erstellt von: Ladina Bilgery und Thomas Wieling}\\
       \textsc{Referent: Prof. Dr. Ulrich Hauser, Co-Referent: Lukas Toggenburger}\\
       \textsc{Industriepartner: Interessengemeinschaft Wetterstation Arbon}      
    \end{center}
]

%Einleitung
Die Wetterstation Arbon wurde 2005 als Lehrlingsarbeit des Berufsbildungszentrums Arbon auf Initiative der Technischen Gesellschaft Arbon aufgebaut und in Betrieb genommen. Diese bestand aus mehreren Wettersensoren und einer Webcam, die auf einer Plattform auf dem See draussen montiert waren. Aus technischen Gründen wurde die Wetterstation Ende 2013 ausser Betrieb genommen und im Jahre 2015 von einigen Interessenten wieder in Betrieb genommen, hierbei wurde ein Teil durch Standard Hard- und Software ersetzt. Während der Bachelor-Arbeit wurden diverse Funktionserweiterungen und Modernisierungen vorgenommen bzw. hinzugefügt. Die Arbeiten wurden in sechs Arbeitspakete unterteilt: Hardware, Webseite, Server, Programmierschnittstelle (API), Alarm-Meldungen (Benachrichtigungsservice) und Webcam.
%Aufbau und deren Sensoren
\paragraph{Hardware}
Die Wetterstation Arbon verfügt über fünf Sensor-Einheiten: Webcam, Kombi-Wetter-Transmitter, Wassertemperatur-Sensoren, Pegelsensor und Sonnenstrahlungssensor. Während der Arbeit wurde der Pegel- und Strahlungssensor ersetzt beziehungsweise hinzugefügt.\\ 
Der Strahlungssensor basiert auf dem Messprinzip eines Thermoelements. Die eintreffende Strahlung welche auf einen Absorber trifft, misst die Strahlung, hiermit kann die Sonnenscheindauer bestimmt werden. Der Zähler für die Sonnenstunden wird um eine Minute erhöht, wenn der Messwert grösser ist als ein Schwellwert.\\
Für die Messung des Bodensee-Pegels wurden verschiedene Messprinzipien verglichen, mit dem Hintergedanken, die Pegelmesswerte ebenfalls zur Messung der Wellenhöhe verwenden zu können. Jedoch überwog hier die Einfachheit und Robustheit des bisher verwendeten hydrostatischen Sensors, sodass der alte Drucksensor durch ein baugleiches Produkt ersetzt wurde. 
%Der Pegel- und Strahlung-Sensor ist an einem Web-Interface angeschlossen, welches den Strom-Messwert über eine Web-Schnittstelle zur Verfügung stellt und mittels einer Formel in die zu bestimmende Einheit umgerechnet wird. Ursprünglich war geplant mit dem Pegelsensor ebenfalls die Wellenhöhe zu messen. Dies ist aber aus technischen Gründen nicht möglich.\\ 

%Weiter beherbergt die Wetterstation acht Wasser-Temperatursensoren (PT100-Elemente) über welche die offizielle Wassertemperatur einen Meter sowie für die Schwimmer einen halben Meter unter der Wasseroberfläche, gemessen wird. Die Web-Schnittstelle basiert auf dem Prinzip wie beim Pegelsensor. Mit dem Unterschied, dass hier direkt die Temperatur berechnet wird.
%---NEU---

Die Sensoren bis auf den Wettertransmitter sind an einem WEB-Interface angeschlossen, welches den Strom-Messwert bzw. die Wassertemperatur über eine Webschnittstelle zur Verfügung stellt.
%Frontend
\paragraph{Webseite}
Die bisherige Anzeige der Wetterdaten basiert auf Adobe Flash, welches nicht von allen Browsern unterstützt wird. Mit der neuen HTML5-Spezifikation können dynamische Grafiken erzeugt werden, die von allen Web-Browsern dargestellt werden können. Die bestehende Webseite wurde mittels Google Analytics analysiert. Darin lässt sich erkennen, dass der Grossteil der Nutzer mobile Geräte verwenden. Aus diesem Grund wurde in diesem Projekt nach dem Designkonzept \textit{Mobile First} entwickelt. Zudem zeigte sich beim Erstellen der Designentwürfe, dass die einfachste Möglichkeit, die Daten auf allen Bildschirmgrössen darzustellen, die Informationen in kleine logische Blöcke zu unterteilen. Mittels dem \textit{W3schools} CSS, der \textit{Weather Icons} Icon-Bibliothek, der \textit{JustGage/Raphaël} und \textit{chartist} JS-Bibliothek konnte das Design für die Anzeige der aktuellen Werte wunschgemäss umgesetzt werden.
%Die Wetterverlaufsdarstellung soll einen Überblick über die Wettertendenz der letzten beiden Tage liefern. Die Samplerate beträgt 1h, d.h. die Daten werden aus der Tabelle der historischen Werte abgerufen. Für die Darstellung der Messwertverläufe soll ebenfalls auf eine Bibliothek zurückgegriffen werden. Zur Verlaufsdarstellung wird primär das Linien- und Balkendiagramm verwendet, wie in Abbildung \ref{img:charts} aufgezeigt. Für jede Grafik wurde entschieden, ob eine automatische Y-Achs-Skalierung sinnvoll ist oder nicht. Bei der Windgeschwindigkeit und beim Pegel wird bewusst eine fixe Skalierung verwendet, damit auf den ersten Blick klar ist, ob der Wert eher hoch oder tief ist. Beim Luftdruck hingegen ist die Tendenz wichtig, weshalb möglichst die gesamte Höhe des Diagramms genutzt werden soll. Es wird daher eine automatische Y-Achs-Skalierung verwendet. Die Anzeige des Windrichtungsverlaufs ist nicht ganz trivial, da sie von 0 bis 360 Grad geht und ohne Unterbruch wieder zu 0 Grad. In der Praxis wird dazu häufig die Darstellung von Pfeilen verwendet, wie in Abbildung \ref{img:windrichtung}, links dargestellt. Es konnte jedoch keine Bibliothek gefunden werden, das diese Darstellungsart als Template zur Verfügung stellt. Die Anzeige der Windrichtung wird deshalb über ein Punktdiagramm, wie in Abbildung \ref{img:windrichtung}, rechts dargestellt, verwendet. 
Zusätzlich wird neu der Wind mit einem kostenlosen Vorhersagediensten verglichen und die Sturmwarnung auf dem Bodensee integriert, wobei die Daten von der Seite der Kantonspolizei stammen. Damit eine geöffnete Webseite immer auf dem aktuellen Stand ist, wird eine poll-Funktion verwendet, welche die Aktualisierung asynchron mittels AJAX durchführt. Dabei wird nicht die ganze Seite neu geladen, sondern nur die Anzeigewerte. Nebst den aktuellen Werten können neu auch folgende historische Daten eingesehen werden: Daten der neuen Wetterstation (2015 bis heute), Daten der alten Wetterstation (2005 bis 2012) und die historischen Pegeldaten (1953 bis 2005). Dazu wurden mit Tableau Public, einem Daten-Visualisierungsprogramm, Diagramme erstellt, welche interaktiv vom Benutzer filterbar sind.\\ 
Zusätzlich zu den neuen Anzeigen sollte die Webseite gestaltet werden, dass sie möglichst für alle Benutzergruppen zugänglich ist. Die aktuellen \emph{Web Content Accessibility Guidelines} fordern die Einhaltung von vier Designprinzipien. Aus den Richtlinien wurden diejenigen Anforderungen ausgewählt, die auf die Webseite der Wetterstation anwendbar und die im Rahmen des \textit{screenbox}-CMS umsetzbar sind.
%Backend
\paragraph{Server}
Auch bei der  Serverseite wurden diverse Neuerungen vorgenommen, da diese die Schnittstelle zwischen dem Client und den Messdaten bildet. Die Sensoren werden in regelmässigen Abständen abgerufen und in der Datenbank gespeichert. Hierbei ist bei der Modernisierung die Datenbank zum Herzstück der Wetterstation geworden. Diese muss somit entsprechend vor Datenverlust und oder -manipulation geschützt werden. Hierfür wurden die Abfrage der einzelnen Daten, mit Ausnahme der Messwerten des Kombi-Wettertransmitters, mittels Python-Skripte durchgeführt. Die Daten vom Wettertransmitter werden weiterhin von WeatherDisplay, die Wetterdatenaufbereitungssoftware, über eine virtuelle serielle Schnittstelle abgerufen, aufbereitet und im Minutentakt in die Datenbank gespeichert. Bei den externen Sensordaten wird im Skript schon kontrolliert ob ein Datenbankeintrag erstellt werden konnte. Dies geschieht, indem sämtliche Aufgaben in den Python-Skripten in einem try...catch-Verfahren ausgeführt werden. Da das Auslesen des Kombi-Wettertransmitters nicht mittels Cronjob durchgeführt wird gibt es die Möglichkeit des Exeption-Handlings nicht. Als Alternative wird aber beim Erstellen der historischen Daten kontrolliert, ob alle 60 Einträge der letzten Stunde vorhanden sind. Weiter werden SQL-Abfragen mittels \emph{Prepared Statements} durchgeführt und die Benutzereingaben \emph{escaped}. Die Sturmwarndaten, sowie die Windvorhersagen werden periodisch mittels einem Python-Skript ausgelesen und in die Datenbank geschrieben.\\
Die Datenbank besteht aus mehreren Tabellen mit unterschiedlichem Inhalt, für die Datenspeicherung stellt der Webhosting-Provider der Wetterstation, eine Datenbank des Typs MariaDB Version 10.1 mit dem Administrationstool phpMyAdmin zur Verfügung. Die Tabellen haben untereinander keine Relation, sondern sind alle eigenständig. Das einzige, was sie verbindet ist der Zeitstempel des Messzeitpunkts. Für die Anzeige der historischen Daten, werden die 21 Millionen anfallenden Datensätze periodisch zusammengefasst und in die  historische Tabelle\emph{tblhistorical} geschrieben. Dies wird nicht mit einem direkten Tabellen zugriff umgesetzt sondern mittels einer VIEW. Die beiden VIEWs, welche für die Aggregation benötigt werden, beinhalten jeweils genau 60 Einträge.\\
Um die Datenbank gegen einen allfälligen Datenverlust zu sichern und den Aufwand klein zu halten, wird empfohlen ein monatliches Backup mit der manuellen Backup Funktion zu erstellen und dieses separat zu speichern.
%API
\paragraph{Programmierschnittstelle (API)}
Um die Kommunikation von Server zu Server zu vereinfachen werden maschinenlesbare Schnittstellen eingesetzt, sogenannte application programming interfaces (API). Die API wurde in die drei Kategorien, Messdaten (data), Zusatzinformationen (misc), Webcam-Links (webcam), aufgeteilt. \emph{Data} ist für die direkten bzw. indirekten Messdaten der Wetterstation vorgesehen. Unter \emph{misc} werden Daten von Dritten beherbergt. Unter \emph{webcam} werden Links für die Webcam zur Verfügung gestellt. Die Struktur wurde so gewählt, dass sie logisch ist, sich beliebig erweitern lässt und die URL trotzdem möglichst kurz ist. Die URL entspricht zudem genau dieser hierarchischen Datenstruktur. Die API wurde wie heutzutage üblich nach dem RESTful Prinzip entwickelt. Zudem wird bei der Wetterstation nur die GET-Methode zugelassen, da die API ein read-only-Dienst darstellt. Die Versionierung URL wurde nach der semantischen Versionierung erstellt. Um die Kompatibilität zu gewährleisten wurde die MAJOR-Nummer in die URL der API aufgenommen. Damit sich die Computer gegenseitig verstehen, ist es wichtig, dass die Kommunikation in einem standardisierten Datenformat erfolgt. Als Datenformat der API wurde JSON gewählt, hierbei handelt es sich um simples und im Webbereich häufig eingesetztes Datenformat. Die API ist komplett in php geschrieben, da \textit{Hostpoint} kein serverseitiges Javascript erlaubt und ist ohne Framework aufgebaut. Serverseitig wurde die API aus hierarchischer Sicht von unten nach oben entwickelt, die Abfrage auf die Informationseinheit bildet somit den Kern der API, zudem wurde die Dateistruktur nach dem MVC-Prinzip aufgebaut. Die ganze API ist so aufgebaut, so dass sie für weitere Anwendungen oder Messwerte ausbauen lässt. Die Dokumentation wurde in Postman dokumentiert und ist öffentlich zugänglich. Dabei werden sämtliche Endpunkte, die von der API zur Verfügung gestellt werden aufgelistet und erklärt. Zudem wird die Dokumentation automatisch aufgerufen wenn sich in der URL-Anfrage auf \texttt{api.wetter-arbon.ch} ein Fehler befindet.
%Notifications
\paragraph{Alarm-Meldungen (Benachrichtigungsservice)}
Die Idee hinter dem Benachrichtigungsservice ist, dass mit dem  Formular des Benachrichtigungsservices, der Benutzer per Mail informiert wird, sobald der von ihnen definierte Grenzwert über- bzw. unterschritten wird.  Dieser ist so aufgebaut, dass auf der Webseite das Formular mit den gewünschten Daten ausgefüllt werden kann und die Benachrichtigung ohne Benutzeraccount funktioniert. Deswegen wird in jedem Mail, welches versendet wird, ein Unsubscribe-Link mitgeschickt. So kann der Benutzer den Service selbständig wieder abmelden. Der Benachrichtigungsservice selbst basiert auf einem Cronjob, der einmal pro Stunde läuft. Unter HTML5 ist die Formularüberprüfung direkt in die Formularfelder integriert, welches die Überprüfung in PHP überflüssig macht. Die angegebenen Daten werden, zusammen mit dem Status aktiv oder inaktiv sowie den Zeitstepmel des letzen Mails in einer Tabelle für das Abo gespeichert. Das Benachrichtigungs-E-Mail wird als sogenanntes \emph{MIMEMultipart}-Mail versendet.

%Webcam
\paragraph{Webcam}
Bei der Webcam der Wetterstation handelt es sich um eine dreh-, schwenk- und zoombare Netzwerk-Kamera. Die Webseite bietet die Möglichkeit, die Webcam über den Browser zu steuern. Sind mehrere Nutzer gleichzeitig auf der Seite können sie sich gegenseitig beim Navigieren stören, deswegen wurde eine Warteschlange implementiert. Der Benutzer registriert sich per Klick auf den Registrierungsbutton, so wird eine Session-ID gelöst und diese mittels POST-Methode an den Server gesendet. Anschliessend wird in einem PHP File seine Wartezeit berechnet und zurück gegeben. Sind bereits andere Benutzer in der Warteschlange, so wird unterhalb des Registrierungsbuttons ein roter Balken mit der Wartezeit angezeigt. Ist der Benutzer an der Reihe wird die verbleibende Zeit mit einem grünen Balken, angezeigt. Das Löschen des Datenbankeintrags erfolgt automatisch nach 10 Minuten.
\paragraph{Fazit}
Es entstand eine übersichtliche Webseite welche die Daten über die API bezieht und einer Datenbank als Herzstück. Fast alle Anforderungen konnten erfüllt werden. Wobei nur drei SOLL-Anforderungen nicht  oder nur teilweise umgesetzt werden konnten. Insgesamt konnte mit der Arbeit ein merklicher Mehrwert im Dienstleistungsumfang der Wetterstation erzielt werden. 
\end{multicols}
 
\end{document}

