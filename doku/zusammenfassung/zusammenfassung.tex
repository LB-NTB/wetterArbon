
\documentclass[11pt]{article}
\usepackage[utf8]{inputenc}
\usepackage[english]{babel}
\usepackage{geometry}
\geometry{a4paper, top=25mm, left=25mm, right=25mm, bottom=30mm,
headsep=10mm, footskip=12mm}
\usepackage{multicol}
\setlength{\columnsep}{1cm}
 
\begin{document}
\begin{multicols}{2}
[
%  Title and authors
    \begin{center}
      {\huge\sffamily Multiplattform-fähiges und barrierefreies User Interface und Datenmanagement für die Wetterstation Arbon}\\
       \vspace{2ex}
       \textsc{Ladina Bilgery und Thomas Wieling}
    \end{center}
]

%Einleitung

Die Wetterstation Arbon wurde 2005 als Lehrlingsarbeit des Berufsbildungszentrums Arbon auf Initiative der Technischen Gesellschaft Arbon (TGA) aufgebaut und in Betrieb genommen. Sie bestand aus mehreren Wettersensoren und einer Webcam, die auf einer Plattform auf dem See draussen montiert waren.Die Wetterstation wurde deshalb Ende 2013 ausser Betrieb genommen.

Ein Teil der Hardware und die komplette Software wurde 2015 von einer Gruppe Freiwilliger durch Standardhard- und Software ersetzt.
Während der Bachelor-Arbeit wurde nicht nur Adobe Flash abgelöst, sondern auch diverse Modernisierungen und Funktionserweiterungen hinzugefügt.
Die verschiedenen Arbeiten wurden in sechs Blöcke unterteilt: Hardware, Webseite, Server, Programmierschnittstelle (API), Alarm-Meldungen (Benachrichtigungsservice) und Webcam.

%Aufbau und deren Sensoren
Die Wetterstation Arbon verfügt über fünf Sensoren bzw. Sensor-Einheiten: Webcam, Kombi-Wetter-Transmitter, Wassertemperatur-Sensoren, Pegelsensor und Sonnenstrahlungssensor. Sämtliche Daten werden per TCP/IP über eine Glasfaserleitung an den landseitigen Server geschickt. Der Pegel- und Strahlungssensor wurden während der Bachelor-Arbeit ersetzt beziehungsweise hinzugefügt.
Die Sonnenscheindauer dient der näherungsweisen Bestimmung der Einstrahlung an einem bestimmten Ort und gibt gleichzeitig Hinweise auf Zeit und Stärke der Bewölkung.
Das Pyranometer basiert auf dem Messprinzip eines Thermoelements, wie in Abbildung dargestellt. Die eintreffende Strahlung trifft auf einen Absorber, welcher erwärmt wird. Die (relative) Sonnenscheindauer beschreibt den Anteil der tatsächlichen an der effektiv möglichen Sonnenscheindauer in Prozent. Durch sie kann man Sonnenscheinverhältnisse verschiedener Gebiete vergleichen. Der Zähler für die Sonnenstunden wird um eine Minute erhöht, wenn der Messwert grösser ist als der Schwellwert und der Sonnenwinkel mindestens 3 Grad beträgt. Zur Bestimmung der beiden Koeffizienten konnte auf die Messwerte des NTB in Buchs zurückgegriffen werden. Da die Strahlungsintensität von Buchs und Arbon vergleichbar sind (gleiche geografische Breite und Höhe), wurden die Koeffizienten direkt übernommen. Die ursprüngliche Wetterstation verwendete einen hydrostatischen Drucksensor für die Messung des Wasserstands (Pegel). Der Drucksensor lieferte aber nach zehn Jahren Betrieb keine plausiblen Werte mehr, weshalb er ersetzt werden musste. Für die Messung des Bodensee-Pegels wurden verschiedene Messprinzipien verglichen, mit dem Hintergedanken die Pegelmesswerte ebenfalls zur Messung der Wellenhöhe verwenden zu können. Die Einfachheit und Robustheit des hydrostatischen Sensors überwog, sodass der alte Drucksensor durch das gleiche Produkt ersetzt wurde. Der Pegelsensor liefert einen Strom von 4...20mA bei einer Messhöhe von 4 Meter. Der Pegelsensor ist 1.75 Meter über dem Pegelnullstand angebracht.
Die Berechnung des Pegel Konstanz aus dem Messwert erfolgt demnach gemäss der Formel. Aus technischen Gründen ist 1.75m bzw. 5.75m der tiefste bzw. höchste von der Wetterstation messbare Pegel. Der Pegelsensor ist an einem Web-Interface angeschlossen, welches den Strom-Messwert über eine Web-Schnittstelle zur Verfügung stellt. Ursprünglich war geplant mit dem Pegelsensor ebenfalls die Wellenhöhe zu messen. Dies ist aber mit dem gewählten Pegelsensor bzw. dessen Montage nicht möglich. Das Einbaurohr wirkt als Tiefpass-Filter und verunmöglicht die Messung von schnellen Wassersäuleänderungen. Die Wetterstation verfügt über acht Wasser-Temperatursensoren (PT100-Elemente) offizielle Wassertemperatur wird einen Meter unter der Wasseroberfläche gemessen. Für Schwimmer ist aber die Temperatur auf einem halben Meter unter der Wasseroberfläche interessanter.  Mittels Cronjob werden sämtliche Temperatursensoren ausgelesen und in einem Array gespeichert. Anschliessend wird mit Hilfe des Pegelwerts der korrekte Wert entnommen. Beim Sensor 5 ist nicht klar, ob er im oder über dem Wasser ist. Auf Grund der Messdaten lässt sich auch kein konstanter Offset gegenüber den benachbarten Sensoren ermitteln. Der fehlende bzw. falsche Wert wurde trotzdem durch einen Offset angenähert.

%Frontend
Die bisherige Anzeige der Wetterdaten basiert auf Adobe Flash, was nicht von allen Browsern unterstützt wird (siehe Fachmodul-Bericht). Die neue (2014) HTML5-Spezifikation ermöglicht, dynamische Grafiken zu erzeugen, die nativ von allen Web-Browsern dargestellt werden können. bestehende Webseite mittels Google Analytics analysiert. Darin lässt sich erkennen, dass der Anteil an mobilen Geräten zwischen 50\% und 80\% beträgt. Aus diesem Grund wurde in diesem Projekt die mobile Seite als Ausgangspunkt für die Entwicklung der Homepage nach dem Designkonzept \textit{Mobile First} gesetzt. Bei Mobile First beginnt der Designer mit dem Mobile-Design und arbeitet sich dann schrittweise zur grösseren Desktop-Version vor. Beim Erstellen der ersten Designentwürfe  zeigte sich, dass die einfachste Möglichkeit, die Daten auf allen Bildschirmgrössen darzustellen, darin besteht, die Informationen in kleine logische Blöcke zu unterteilen. Die Benutzeroberfläche muss Cross-Plattform-fähig sein, d.h. die Darstellung muss sich je nach Bildschirmgrösse automatisch anpassen. Man nennt dies \textit{Responsive Design}. Das Konzept des Grid-View teilt den Bildschirm bzw. den zur Verfügung stehnden Platz unabhängig von dessen Grösse in zwölf Spalten ein. Die gleiche Kachel ist also auf einem grossen Bildschirm  \nicefrac{2}{12} breit und auf einem kleinen Bildschirm \nicefrac{6}{12}. Unter \emph{w3.css} sind drei Bildschirmgrössen vorgesehen (small, medium und large). Die Grenzen sind folgendermassen festgelegt. Die Anzeigeelemente für die aktuellen Messwerte sollen ebenfalls mit Hilfe einer Bibliothek erstellt werden. Am Besten nicht nur beim Aufrufen der Seite, sondern auch beim Ändern der Fenstergrösse. Zudem muss es einfach möglich sein, die Werte zu aktualisieren. Die Messwerte sind einerseits über den Titel beschrieben, sie enthalten aber zusätzlich ein passendes Icons. Es wurde eine Icon-Bibliothek gewählt, die möglichst alle benötigen Icons enthält, die kostenlos ist, und deren Grafiken im svg-Format vorliegen. Die Wetterverlaufsdarstellung soll einen Überblick über die Wettertendenz der letzten beiden Tage liefern. Die Samplerate beträgt 1h, d.h. die Daten werden aus der Tabelle der historischen Werte abgerufen. Für die Darstellung der Messwertverläufe soll ebenfalls auf eine Bibliothek zurückgegriffen werden. Zur Verlaufsdarstellung wird primär das Linien- und Balkendiagramm verwendet, wie in Abbildung \ref{img:charts} aufgezeigt. Für jede Grafik wurde entschieden, ob eine automatische Y-Achs-Skalierung sinnvoll ist oder nicht. Bei der Windgeschwindigkeit und beim Pegel wird bewusst eine fixe Skalierung verwendet, damit auf den ersten Blick klar ist, ob der Wert eher hoch oder tief ist. Beim Luftdruck hingegen ist die Tendenz wichtig, weshalb möglichst die gesamte Höhe des Diagramms genutzt werden soll. Es wird daher eine automatische Y-Achs-Skalierung verwendet. Die Anzeige des Windrichtungsverlaufs ist nicht ganz trivial, da sie von 0 bis 360 Grad geht und ohne Unterbruch wieder zu 0 Grad. In der Praxis wird dazu häufig die Darstellung von Pfeilen verwendet, wie in Abbildung \ref{img:windrichtung}, links dargestellt. Es konnte jedoch keine Bibliothek gefunden werden, das diese Darstellungsart als Template zur Verfügung stellt. Die Anzeige der Windrichtung wird deshalb über ein Punktdiagramm, wie in Abbildung \ref{img:windrichtung}, rechts dargestellt, verwendet. Bei allen Vorhersagediensten ist jedoch nicht erkennbar, wie gut die Prognose war. Die Messdaten der Wetterstation werden daher mit zwei kostenlosen Vorhersagediensten verglichen. Auf dem Bodensee gibt es einen Sturmwarndienst, der die Schiffsführer vor aufkommendem Sturm warnen soll. Den aktuellen Status der Sturmwarnung kann sowohl beim Deutschen Wetterdienst, als auch bei Meteoschweiz kostenpflichtig bezogen werden (ca. 1300CHF/Jahr). nur zwei browserkomatible Quellen Wir haben deshalb beim Amt für Informatik des Kantons Thurgau, welches die Seite der Kantonspolizei betreut, die Erlaubnis eingeholt. Die Sturmwarnung auf dem Bodensee gesetzliche Pflichten mit sich bringt, wurde entschieden nur die offiziellen Daten anzuzeigen und keine anderweitigen Sturmwarnungen. Die Wetterstation misst sämtliche Messgrössen einmal pro Minute. Diese Minutenwerte werden einmal pro Stunde zusammengefasst und als historische Daten abgespeichert. Es gibt somit drei Grundlagen für die historischen Daten:

\begin{itemize*}
\item Historische Daten der neuen Wetterstation (2015 bis heute)
\item Historische Daten der alten Wetterstation (2005 bis 2012)
\item Exceltabelle mit historischen Pegeldaten (1953 bis 2005 )
\end{itemize*}

Tableau Public\footnote{ \url{https://public.tableau.com/de-de/s/}} ist ein Daten-Visualisierungsprogramm, welches es ermöglicht auf einfache Weise Diagramme zu erstellen, welche interaktiv vom Benutzer filterbar sind.  Tableau Public ist kostenlos mit der Einschränkung, dass sämtliche Visualisierungen öffentlich sind. In unserem Fall ist dies kein Problem, da die Visualisierungen sowieso veröffentlicht werden. Es wurde deshalb beschlossen auf der Webseite nur die historischen Daten der neuen Wetterstation d.h. ab 2015 anzuzeigen. Alleine die Pegeldaten werden verwendet und die historische Pegeldatenbank integriert. Als Grundlage diente unter anderem ein Excel-File mit den Pegelständen von 1953 bis 2005. Von der alten Wetterstation liegen die Pegeldaten von 2005 bis 2009 vor und ab 2018 kommen die Messwerte der neuen Wetterstation dazu. Damit eine geöffnete Webseite immer auf dem aktuellen Stand ist, wird eine poll-Funktion verwendet. Die gesamte Aktualisierung wird asynchron mittels AJAX durchgeführt. Die Seite wird dabei nicht neu geladen, nur die Anzeigewerte. Die Wetterstation und ihre Webseite ist eine Dienstleistung der Stadt Arbon. Es geht darum, die Webseite so zu gestalten, dass sie möglichst für alle Benutzergruppen zugänglich ist. Die aktuellen \emph{Web Content Accessibility Guidelines}\footnote{\url{https://www.w3.org/TR/WCAG20/}} fordern die Einhaltung von vier Designprinzipien. Aus den Richtlinien wurden diejenigen Anforderungen ausgewählt, die auf die Webseite der Wetterstation anwendbar und die im Rahmen des CMS\footnote{CMS: Content Management System} umsetzbar sind. 

%Backend

Die serverseitige Implementierung bildet die Schnittstelle zwischen dem Client das heisst dem Internet Browser und den Messdaten. Diese müssen von den Sensoren in regelmässigen Abständen abgerufen und in der Datenbank gespeichert werden. Da die Datenbank das Herzstück der Wetterstation darstellt muss sie auch entsprechend geschützt werden vor Datenverlust und oder -manipulation. Die einzelnen Daten werden unterschiedlich erfasst, wobei mit Ausnahme der Messwerten des Kombi-Wettertransmitters sämtliche Abfragen mittels Skripten durchgeführt werden. Die Daten des Kombi-Wettertransmitters werden weiterhin von WeatherDisplay über eine virtuelle serielle Schnittstelle abgerufen, aufbereitet und im Minutentakt in die Datenbank gespeichert. Kann ein Datenbankeintrag aus irgendwelchen Gründen nicht erstellt werden, wird dies gemeldet.  Strahlungs- und Pegelsensor sind an einem Web-IO angeschlossen (siehe Datenblatt im Anhang\,\ref{Spec_webio}), welches die analogen Werte (in Milliampere) per Web-Schnittstelle zur Verfügung stellt. Die Abfrage erfolgt einmal pro Minute durch ein Python-Skript, wie in Listing \ref{lst:webIo} dargestellt. Die Temperatursensoren (PT100-Elemente) werden ähnlich erfasst mit dem Unterschied, dass nicht der Widerstandswert sondern direkt Grad Celsius geliefert. Die Sturmwarndaten werden periodisch mittels Python-Skript ausgelesen. Da aber das Web-Scrapping die einzige kostenlose Möglichkeit darstellt um die Daten zu erhalten. Die Windvorhersagedaten von Windfinder werden ebenfalls mittels BeautifulSoup von der Webseite abgegriffen. Die Datenbank besteht aus mehreren Tabellen mit unterschiedlichem Inhalt. Für die Datenspeicherung stellt Hostpoint, der Webhosting-Provider der Wetterstation, eine Datenbank des Typs MariaDB Version 10.1 mit dem Administrationstool phpMyAdmin zur Verfügung. Die Tabellen haben untereinander keine Verknüpfung (Relation), sondern sind alle eigenständig. Das einzige, was sie verbindet ist der Zeitstempel des Messzeitpunkts. Die Datenbank der Wetterstation wurde komplett neu aufgebaut. Bei der Wetterstation fallen pro Minute rund 40 Datenpunkte an, die gespeichert werden. Pro Jahr sind dies über 21 Millionen Datenpunkte. Damit die Anzeige der historischen Messwerte nicht so viele Daten laden muss, werden die Messdaten periodisch zusammengefasst. Die minütlich gespeicherten Messdaten werden einmal pro Stunde zusammengefasst und in die Tabelle mit den historischen Werten \emph{tblhistorical} geschrieben.Das Script, welches die minütlichen Daten zu Stundendaten aggregiert greift nicht direkt auf die Messwerttabellen zu, sondern auf sogenannte VIEWs. Die beiden VIEWs, die für die Aggregation benötigt werden beinhalten jeweils genau 60 Einträge. Einen für jede Minute der vergangenen Stunde. Die Norm DIN ISO 8601\footnote{DIN ISO 8601: Informationsaustausch - Darstellung von Datum und Uhrzeit} standardisiert die Darstellung von Datum und Zeit. In der Datenbank und in der API der Wetterstation wird jedoch das nichtstandartisierte Format \texttt{2018-07-29 15:34:30} mit Leerzeichen zwischen Datum und Zeit verwendet. Dies führt dazu, dass bei der Zeitumstellung im Frühling und im Herbst jeweils eine Stunde übersprungen beziehungsweise doppelt vorhanden ist. Damit das Vorgehen bei allen Datenbankeinträgen einheitlich ist, wurde das System von Weather Display übernommen. Die Wetterstation basiert auf vielen repetitiven Aufgaben wie zum Beispiel das minütliche Einlesen der Messdaten. Linux, welches auf dem Hostpoint-Server verwendet wird, bietet mit dem Cron-Daemon ein Werkzeug um zeitbasiert Befehle beziehungsweise Skripte (Cronjobs) auszuführen. Die erwähnten Cronjob sind die wichtigsten, werden diese nicht durchgeführt, werden auch keinen Daten ausgelesen bzw. erstellt. Bei der Wetterstation Arbon sind keine persönlichen oder sicherheitsrelevanten Daten in der Datenbank gespeichert. In diesem Fall werden die
die SQL-Abfragen mittels \emph{Prepared Statements} durchgeführt und die Benutzereingaben \emph{escaped} wie in Listing\,\ref{lst:maskierung} dargestellt. Um die Datenbank gegen einen allfälligen Datenverlust zu sichern, ist ein regelmässiges Backup sinnvoll. Um den Aufwand klein zu halten, wird empfohlen ein monatliches Backup mit der manuellen Backup Funktion zu erstellen und dieses auf einer Festplatte zu speichern.  Es stellte sich anfangs die Frage, ob die historischen Daten ausgedünnt werden sollen um Speicherplatz zu sparen. Der Speicherplatz reicht somit, bei gleichbleibender Benützung, noch für die nächsten 50 Jahre.  Dies geschieht, indem sämtliche Aufgaben in den Python-Skripten in einem try...catch-Verfahren ausgeführt werden.  Hostpoint bietet die Möglichkeit, dass sämtliche Textausgaben (print-Funktion) eines Cronjobs an eine bestimmte Mailadresse gesendet werden. Da das Auslesen des Kombi-Wettertransmitters nicht mittels Cronjob durchgeführt wird gibt es die Möglichkeit des Exeption-Handlings nicht. Als Alternative wird aber beim Erstellen der historischen Daten kontrolliert, ob alle 60 Einträge der letzten Stunde vorhanden sind.


%API

Um die Kommunikation von Server zu Server zu vereinfachen werden maschinenlesbare Schnittstellen eingesetzt, sogenannte application programming interfaces (API). Die API wurde deshalb in drei Kategorien aufgeteilt:

\begin{itemize*}
\item Messdaten (data)
\item Zusatzinformationen (misc)
\item Webcam-Links (webcam)
\end{itemize*}

\emph{Data} ist für die direkten bzw. indirekten Messdaten der Wetterstation vorgesehen. Unter \emph{misc} werden verschiedene Arten von Daten. Unter \emph{webcam} werden Links für die Webcam zur Verfügung gestellt. Die Struktur wurde so gewählt, dass sie logisch ist, sich beliebig erweitern lässt und die URL trotzdem möglichst kurz ist. die URL entspricht genau dieser hierarchischen Datenstruktur. Heutzutage werden viele API nach dem REST-Prinzip\,\footnote{REST: Representational State Transfer} von Roy Fielding entwickelt. Eines der wichtigsten Prinzipen daraus ist die Zustandslosigkeit einer Anfrage. Bei der Wetterstation wird nur die GET-Methode verwendet, da die API der Wetterstation sozusagen ein read-only-Dienst darstellt. Die Versionsnummer der API ist beim Aufrufen der oberste Hierarchiestufe sichtbar, Angewendet wird dabei die semantische Versionierung\footnote{\url{https://semver.org}}.  Eine API wird häufig von einem anderen Computer das heisst von einem Programm aufgerufen. Eine inkompatible Änderung würde dazu führen, dass dieses Programm unter Umständen nicht mehr korrekt funktioniert. Aus diesem Grund wurde die MAJOR-Nummer in die URL der API aufgenommen. Damit sich die Computer gegenseitig verstehen, ist es wichtig, dass die Kommunikation in einem standartisierten Datenformat erfolgt. Als Datenformat der API wurde JSON\footnote{JSON: Javascript Object Notation} gewählt, da es sich um ein simples und im Webbereich häufig eingesetztes Datenformat handelt. Für die API gibt es jedoch eine wichtige Bedingung. Sie muss in php geschrieben werden. Wie das Resultat aus der Nutzwertanalyse (Tabelle \ref{table:php-framework}) aufzeigt, wird kein Framework benutzt. Hier kann ohne Framework auf die standard Funktionen zurückgegriffen werden. Der Ausschlaggebende Punkt in dieser Arbeit ohne Framework zu arbeiten ist die Grösse der API, sowie den Wunsch der IG Arbon, die ganze Arbeit so zu gestalten, dass praktisch kein Unterhalt notwendig ist. Serverseitig wurde die API aus hierarchischer Sicht von unten nach oben entwickelt. Das heisst die Abfrage auf die Informationseinheit bildet den Kern der API. Die Dateistruktur wurde nach dem MVC-Prinzip nachempfunden. Die ganze API ist Modular aufgebaut, so dass sie für weitere Anwendungen oder Messwerte ausgebaut werden kann. Die API wurde in Postman\footnote{\url{https://www.getpostman.com}} dokumentiert und ist über die Webseite\footnote{\url{https://documenter.getpostman.com/view/4035921/api-wetter-arbon/RW1XMhBK}} zugänglich.  Dabei werden sämtliche Endpunkte, die von der API zur Verfügung gestellt werden aufgelistet und erklärt. Die Doku-Seite wird automatisch aufgerufen wenn sich in der URL-Anfrage auf \texttt{api.wetter-arbon.ch} ein Fehler befindet.

%Notifications
Die Idee hinter dem Benachrichtigungsservice ist, dass die Benutzer per Mail informiert werden, sobald der von ihnen definierte Grenzwert über- bzw. unterschritten wird. Der Benachrichtigungsservice ist so aufgebaut, dass auf der Webseite ein Formular mit den gewünschten Daten ausgefüllt werden kann. Da es keinen Benutzeraccount gibt, wird in jedem Mail, das versendet wird, ein Unsubscribe-Link mitgeschickt. So kann der Benutzer den Service selbständig wieder löschen. Der Benachrichtigungsservice selbst basiert auf einem Cronjob, der einmal pro Stunde läuft. Das Formular des Benachrichtigungsservices besteht aus vier Eingabe- bzw. Auswahlfeldern. Unter HTML5 ist die Formularüberprüfung direkt in die Formularfelder integriert. Die Tabelle für die Speicherung der Abos in der Datenbank enthält neben den vom Benutzer eingegebenen Daten noch weitere Einträge. aktiv oder inaktiv und der Zeitstempel des letzen Mails. Das erste ist die Bestellung des Abos (\textit{subscribe}). Das zweite ist die Verifizierung der E-Mail-Adresse (\textit{verify}) und als letztes die Löschung des Abos (\textit{unsubscribe}). Das Benachrichtigungs-E-Mail wird als sogenanntes \emph{MIMEMultipart}-Mail versendet.

%Webcam
Bei der Webcam\footnote{\url{https://dev.wetter-arbon.ch/webcam}} der Wetterstation handelt es sich um eine dreh-, schwenk- und zoombare Netzwerk-Kamera der Firma AXIS. Die Webseite bietet die Möglichkeit die Webcam über den Browser zu steuern, sind mehrere Nutzer gleichzeitig auf der Seite können sie sich gegenseitig beim Navigieren stören. Sind bereits andere Benutzer in der Warteschlange, so wird unterhalb des Buttons ein roter Balken mit der Wartezeit angezeigt. Die verbleibende Zeit wird wiederum im grünen Bereich unterhalb des Buttons angezeigt. Benutzer per Klick auf den Registrierungsbutton wird eine Session-ID gelöst und diese per POST an den Server gesendet. wann sich der Benutzer registriert hat, wie lange seine Wartezeit ist und seine Session-ID. Das Löschen des Datenbankeintrags erfolgt automatisch nach 10 Minuten.




\end{multicols}
 
\end{document}

