%%%%%%%%%%%%%%%%%%%%%%%%%%%%%%%%%%%
%%  Schluss
%%%%%%%%%%%%%%%%%%%%%%%%%%%%%%%%%%%
\section{Schluss}
% viele Konzepte
Während der Analyse der Wetterstation konnten wir diverse Schwachstellen ausfindig machen, die mehr oder weniger dringend beseitigt werden müssen.
Bei vielen Problemen muss zuerst ein Lösungskonzept erstellt werden, bevor mit der Behebung begonnen werden kann. Diese Arbeit bedarf einiges an Recherchearbeit und darf nicht unterschätzt werden.
\newline

\noindent
% unbekannte Thematik
Das gesamte Themengebiet von Barrierefreiheit und User Interface Design ist für uns neu und wir müssen das gesamt Know-how von Grund auf aufbauen. Ebenso ist die Arbeit mit Skripten und die Verdünnung der Daten auf einer Datenbank neu für uns. Auch hier ist ein grosser Teil für Einlesearbeiten zu erwarten.
\newline

\noindent
% schwierige Rahmenbedingungen
Als kritisch beziehungsweise schwer abschätzbar sehen wir die gegebenen Rahmenbedingung, die uns allenfalls in der Lösungsfindung stark einschränken.
Zum Beispiel ist dies das vorgegebene CMS oder die Schnittstelle zum Kombi-Wetter-Transmitter über \textit{WeatherDisplay}.
\newline

\noindent
% Rückblick auf Stundenschätzung
Anfangs Fachmodul schätzen wir die Stundenaufwände für die während des Fachmoduls anstehenden Arbeiten ab. Die effektiven Aufwände haben wir mittels Toggl dokumentiert, sodass wir nun am Schluss des Fachmoduls die geleisteten Stunden den geplanten gegenüberstellen können. Die Auflistung befindet sich in Tabelle~\ref{table:plan-ist}. 
Sowohl die produktiven Arbeiten, d.h. die Arbeiten, die gemäss Fachmodul-Auftrag zu erledigen waren, als auch die administrativen Aufwände für Meetings und Wochenreports stimmen recht gut. Den Aufwand für die Dokumentation hingegen haben wir unterschätzt. Wenn wir anfangs Bachelor-Arbeit den definitiven Terminplan erstellen, müssen wir dies berücksichtigen.
\newline

\begin{table}[h]
\centering
\begin{tabular}{|l|c|c|r|}
\hline
 Tätigkeit			&  Plan	& Ist  	& Delta  		\\ \hline
 Produktive Arbeit	&  79		&  73		&  -8~\%		\\ \hline
 Dokumentation		&  35		&  63		&  +80~\%		\\ \hline
 Administration		&  30		&  27		&  -10~\%		\\ \hline
\end{tabular}
\caption{Vergleich der Planstunden zu den Ist-Stunden}
\label{table:plan-ist}
\end{table}

\noindent
Zusammenfassend sehen wir die Bachelor-Arbeit als machbar an. Zeitlich bleibt jedoch nicht viel Spielraum. Die grossen Unbekannten wie  einschränkende Rahmenbedingungen und schwer abzuschätzenden Aufwand für Einarbeitung und Konzepterstellung bedingen jedoch, dass der Fortschritt kontinuierlich und kritisch geprüft wird.