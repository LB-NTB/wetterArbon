\section{Erweiterungen}

\subsection{Benachrichtigungs-Service}
Des Weiteren wurde öfters der Wunsch nach einer SMS-Benachrichtigung geäussert, dies ist momentan nicht realisiert und soll auch teil unserer BA sein. 

Die Sturmwarnung soll wieder auf der Webseite sichtbar sein und der Besucher der Webseite soll die Möglichkeit haben sich für eine Notification zu registrieren. Dafür wurden 3 verschiedene Möglichkeiten ausgewählt und mit der Nutzwertanalyse ausgewertet. Ziel bei allen Möglichkeiten ist es, dass der Benutzer die Möglichkeit hat sich zu registrieren und Alarmkriterien zu bestimmen. Werden die gewählten Alarmkriterien erreicht bzw. wird eine Sturmwarnung herausgegeben, wird der Benutzer benachrichtigt. Hiermit soll die Möglichkeit gegeben werden, dass der Benutzer in Echtzeit informiert wird und somit keine Warnung oder sein "perfektes" Segelwetter verpasst. Für die evaluierung der Notifications wurde eine Nutzwertanalyse erstellt. Dies ist eine gute Möglichkeit, um verschiedene Lösungsansätze zu bewerten. Der Nachteil hierbei ist jedoch, dass die Bewertung sehr subjektiv ist. 

\begin{center}
\begin{tabular}{ |p{3.5cm}||p{1cm}|p{2cm}|p{3.5cm}|p{2.5cm}|p{1.5cm}|}
 \hline
 \multicolumn{6}{|c|}{Nutzwertanalyse} \\
 \hline
	Möglichkeiten & Kosten & Einfachheit & Programmieraufwand & Anpassbarkeit & Support\\
 \hline
	SMS & 1 & 4 & 3 & 3 & 5\\
	E-Mail & 5 & 4 & 5 & 5 & 1\\
	FacebookMessenger & 5 & 4 & 3 & 4 & 1\\
 
\hline
\end{tabular}
\end{center}

Aus der Nutzwertanalyse ist Sichtbar, dass die Benachrichtigung per E-Mail und Facebook Messenger die Lösungsansätze mit der höchsten Punktzahl sind . Der grösste Vorteil der beiden möglichen Lösungen sind, dass sie Kostenlos sind. Der Nachteil an Facebook Messenger ist, das nicht davon ausgegangen werden kann, dass jeder Benutzer ein Facebookprofil hat. 


\subsection{Vergleich Windmessung mit Prognose}
