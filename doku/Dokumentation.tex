\section{Dokumentation}
\Diskussionspunkt{Versionsverwaltung, Dokumentationskonzept, Tools?, Printscreens}

%Allgemein
Für die Bachelor-Arbeit verwenden wir unterschiedliche Dokumentationswerkzeuge. Bei der Auswahl haben wir darauf geachtet, das die Tools kostenlos nutzbar und für sämtliche Plattformen verfügbar sind (Windows, Mac, iPad, usw.) Weiter war uns wichtig, dass die Tools untereinander kommunizieren können. 

% github / Trello / Toggl
Sämtliche Artefakte speichern wir auf \textit{github}. Wir haben somit eine automatische Versionierung der Dokumente und können unabhängig voneinander an den Dokumenten arbeiten. Die Planung bzw. Darstellung des Entwicklungsprozesses erledigen wir mir \textit{Trello}. Es ist ein intuitives Tool, welche diverse Integrationsmöglichkeiten mit den anderen Tools bietet. Für die Zeiterfassung verwenden wir \textit{Toggl}, welches mittels Plugin direkt in Trello integriert werden kann.

% Kommunikation nach aussen
Damit wir keine Besprechungsprotokolle verschicken müssen und dass alle Informationen für alle immer zugänglich sind, haben wir entschieden das Reporting in Form einer öffentlichen Webseite zu erstellen. github bietet mit \textit{GitPages} einen Hosting-Service an, der genau dies ermöglicht. Der Vorteil von GitPages ist, dass wir sämtliche Daten in einem einzigen Ort bzw. Repository vereint haben. Damit wir uns nicht mit Formatierung u.a. herumschlagen müssen und uns auf den Inhalt konzentrieren können, verwenden wir \textit{mkdocs} als Template Engine. Die Webseiten-Einträge können wir dadurch auf simple Art in Form von Markdown-Files erstellen.

% Kommunikation nach innen
Innerhalb des Teams nutzen wir das Kommunikationstool \textit{Slack}. Dieses ermöglich uns Konversationen als Chat aufzuzeichnen und nach Themen zu gruppieren. Weiter lassen sich Dokumente austauschen. Sämtliche git-Posts werden von Slack automatisch geloggt und können, falls gewünscht, als push-Notification angezeigt werden.
Das wöchentliche Team-Meeting findet über \textit{Skype} statt, da wir den regelmässigen mündlichen Austausch aus zentralen Punkt erachten.

% Bericht = LaTeX
Den Bericht werden wir in LaTeX verfassen. Wir haben uns für LaTeX entschieden, da wir uns auf den Inhalt konzentrieren können und das Layout automatisiert ist. Weiter ist LaTeX in der Wissenschaft weit verbreitet. Die Bachelor-Arbeit ist deshalb eine gute Gelegenheit uns in dieses Thema einzuarbeiten.


\section{Rechtliche Ansprüche}
siehe separates Dokument

