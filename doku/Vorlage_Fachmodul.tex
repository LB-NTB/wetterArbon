\documentclass[a4paper,ngerman, 12pt]{report}

%% Päambel
\usepackage[T1]{fontenc}
\usepackage[utf8]{inputenc}
\usepackage{babel}

%%  Variablen
\newcommand{\authorName}{Ladina Bilgery \and Thomas Wieling}
\newcommand{\auftraggeber}{Interessengemeinschaft Wetterstation Arbon}
\newcommand{\auftragnehmer}{Interstaatliche Hochschule für Technik NTB}
\newcommand{\projektName}{User Interface und Datenmanagement für die Wetterstation Arbon}
\title{\projektName~(Fachmodul)}
\author{\authorName}
\date{\today}

%%  Create a shorter version for tables. DO NOT CHANGE 
\newcommand\addrow[2]{#1 &#2\\ }
\newcommand\addheading[2]{#1 &#2\\ \hline}
\newcommand\tabularhead{\begin{tabular}{lp{13cm}}
\hline
}
\newcommand\addmulrow[2]{ \begin{minipage}[t][][t]{2.5cm}#1\end{minipage}
   &\begin{minipage}[t][][t]{8cm}
    \begin{enumerate} #2   \end{enumerate}
    \end{minipage}\\ }
\newenvironment{usecase}{\tabularhead}
{\hline\end{tabular}}



%%  Beginn Dokument
\begin{document}
\pagenumbering{roman}
\input{Deckblatt}
\setcounter{page}{2}
\tableofcontents          
\clearpage
\pagenumbering{arabic}


%%%%%%%%%%%%%%%%%%%%%%%%%%%%%%%%%%%
%%  Zusammenfassung
%%%%%%%%%%%%%%%%%%%%%%%%%%%%%%%%%%%
\begin{abstract}
Braucht es eine Zusammenfassung?
\end{abstract}


%%%%%%%%%%%%%%%%%%%%%%%%%%%%%%%%%%%
%%  Einführung
%%%%%%%%%%%%%%%%%%%%%%%%%%%%%%%%%%%
\chapter{Einführung}
Ziel des Fachmoduls, Aufträge, Vorgehensweise, Vorstellung Wetterstation Arbon


%%%%%%%%%%%%%%%%%%%%%%%%%%%%%%%%%%%
%%  Hauptteil
%%%%%%%%%%%%%%%%%%%%%%%%%%%%%%%%%%%
\chapter{Hauptteil}

\section{IST-Zustand}
Hardware und Software, 
Skizze, 
Übersicht, 
Verbindungen/Verknüpfungen untereinander

\section{Vorhandene Probleme}
Beschreibung, 
Begründung, 
Erkenntnisse aus IST-Analyse

\section{SOLL-Zustand}
Lösungsansätze = zu entwickelnde Artefakte, 
Resultat aus Literaturrecherche, 
Konzepte


%%%%%%%%%%%%%%%%%%%%%%%%%%%%%%%%%%%
%%  Spezifikation
%%%%%%%%%%%%%%%%%%%%%%%%%%%%%%%%%%%
\chapter{Spezifikation / Pflichtenheft}
Anforderungen nach dem SMART-Prinzip formulieren
\section{Nutzeranalyse}
Welches sind die Nutzer und was sind deren Bedürfnisse

\section{Funktionale Anforderungen}
\begin{usecase}
  \addheading{Nummer}{Beschreibung} 
  \addrow{/FA10/}{Temperaturanzeige in Grad und Fahrenheit}
  \addrow{/FA20/}{Windgeschwindigkeitsanzeige in Knoten, Km/h, m/s, mph, Bft  }
  \addrow{/FA30/}{Luftdruckanzeige in hPa, mmHg, kPa, inHg, mb, }
  \addrow{/FA40/}{Windrichtung }
  \addrow{/FA50/}{Niederschlagsmenge in mm } 
\end{usecase}


\section{Nicht-Funktionale Anforderungen}
\begin{usecase}
  \addheading{Nummer}{Beschreibung} 
  \addrow{/FA10/}{Webseite soll im responsive Design erstellt sein}
  \addrow{/FA20/}{Webseite soll auch für Menschen mit beeinträchtigungen zur Verfügung stehen}
  \addrow{/FA30/}{Webseite soll mit HTML5 erstellt sein}
  \addrow{/FA30/}{Webseite soll mit JavaScript erstellt sein}
\end{usecase}


   


%%%%%%%%%%%%%%%%%%%%%%%%%%%%%%%%%%%
%%  Projektmanagement
%%%%%%%%%%%%%%%%%%%%%%%%%%%%%%%%%%%
\chapter{Projektmanagement}
\section{Entwicklungsprozess}
V-Modell XT oder KANBAN oder eine Mischung davon, Begründung warum kein Scrum, graphische Darstellung, müssen sämtliche KANBAN-Karten schon bekannt sein?


\section{Projektplan für die Bachelor-Arbeit}
Hier wäre ein A3-Blatt quer noch cool. Darauf sollten alle Wochen von Start BA bis zur Abgabe sein.
Alle offiziellen Termine, alle Meilensteine usw.
Eine Zeile für Meetings (Führungsrythmus)
Abhängigkeit der einzelnen Artefakte voneinander (evtl. mit MS-Project arbeiten)
Was gehört alles in eine Projektplan. Was ist der Unterschied zwischen Projekt- und Terminplan?

\section{Risikoanalyse}
Risikotabelle mit Wahrscheinlichkeit und Auswirkung gewichtet, Risikomatrix als Übersicht, Themen sind technische Umsetzung, Zusammenarbeit mit Dritten, Termine, Ressourcen, 

\section{Dokumentation}
Versionsverwaltung, Dokumentationskonzept, Tools?

\section{Rechtliche Ansprüche}
siehe separates Dokument


%%%%%%%%%%%%%%%%%%%%%%%%%%%%%%%%%%%
%%  Schluss
%%%%%%%%%%%%%%%%%%%%%%%%%%%%%%%%%%%
\chapter{Schluss}
Erkenntnisse, Einschätzungen


%%%%%%%%%%%%%%%%%%%%%%%%%%%%%%%%%%%
%  Verzeichnisse
%%%%%%%%%%%%%%%%%%%%%%%%%%%%%%%%%%%
\bibliography{Literaturverzeichnis}

\begin{thebibliography}{9}
\bibitem{Was kommt hier?} 
Autor
\textit{Verlag}. 
Welche Seite?, Jahr?
\end{thebibliography}



\end{document}
