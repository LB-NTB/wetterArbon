\documentclass[a4paper,ngerman, 11pt, DIV11]{scrartcl}
%\documentclass[a4paper,ngerman, 11pt, pagesize]{report}
% \documentclass{article}

%% Päambel
\usepackage[T1]{fontenc}
\usepackage[utf8]{inputenc}
\usepackage[ngerman]{babel}

% Grad Celsius Zeichen
\usepackage{gensymb}

\usepackage{cite}
\usepackage{xcolor}
\newcommand\Diskussionspunkt[1]{\textcolor{red}{#1}}
\usepackage{ulem}

\usepackage{url}
% keine rote Rahmen um die Links anzeigen
\usepackage[pdfborderstyle={/S/U/W 0}]{hyperref}

% Grafikpaket laden
\usepackage{graphicx}

% Tabellen
\usepackage{booktabs}
\usepackage{longtable}

% Tabellen
\usepackage{tabularx,enumitem,ragged2e}

% Formeln
\usepackage{array}
\newenvironment{conditions}
  {\par\vspace{\abovedisplayskip}\noindent\begin{tabular}{>{$}l<{$} @{${}={}$} l}}
  {\end{tabular}\par\vspace{\belowdisplayskip}}

% pdf einbinden (A3)
\usepackage{nextpage}
\usepackage{afterpage}
\usepackage{pdfpages}
\usepackage{typearea}
\usepackage{pdfpages}

\usepackage{lscape}




% Quelltext
\usepackage{listings}
\usepackage{color}

 % \definecolor{middlegray}{rgb}{0.5,0.5,0.5}
 % \definecolor{lightgray}{rgb}{0.8,0.8,0.8}
 % \definecolor{orange}{rgb}{0.8,0.3,0.3}
 % \definecolor{yac}{rgb}{0.6,0.6,0.1}
 %
 %  \lstset{
 %   basicstyle=\scriptsize\ttfamily,
 %   keywordstyle=\bfseries\ttfamily\color{orange},
 %   stringstyle=\color{green}\ttfamily,
 %   commentstyle=\color{middlegray}\ttfamily,
 %   emph={square},
 %   emphstyle=\color{blue}\texttt,
 %   emph={[2]root,base},
 %   emphstyle={[2]\color{yac}\texttt},
 %   showstringspaces=false,
 %   flexiblecolumns=false,
 %   tabsize=2,
 %   numbers=left,
 %   numberstyle=\tiny,
 %   numberblanklines=false,
 %   stepnumber=1,
 %   numbersep=10pt,
 %   xleftmargin=15pt
 % }




\definecolor{mygreen}{rgb}{0,0.6,0}
\definecolor{mygray}{rgb}{0.5,0.5,0.5}
\definecolor{mymauve}{rgb}{0.58,0,0.82}

\lstset{
  backgroundcolor=\color{white},   % choose the background color; you must add \usepackage{color} or \usepackage{xcolor}; should come as last argument
  basicstyle=\footnotesize,        % the size of the fonts that are used for the code
  breakatwhitespace=false,         % sets if automatic breaks should only happen at whitespace
  breaklines=true,                 % sets automatic line breaking
  captionpos=b,                    % sets the caption-position to bottom
  commentstyle=\color{mygreen},    % comment style
  deletekeywords={...},            % if you want to delete keywords from the given language
  escapeinside={\%*}{*)},          % if you want to add LaTeX within your code
  extendedchars=true,              % lets you use non-ASCII characters; for 8-bits encodings only, does not work with UTF-8
  frame=single,	                   % adds a frame around the code
  keepspaces=true,                 % keeps spaces in text, useful for keeping indentation of code (possibly needs columns=flexible)
  keywordstyle=\color{blue},       % keyword style
  %language=Octave,                 % the language of the code
  morekeywords={*,...},            % if you want to add more keywords to the set
  numbers=none,                    % where to put the line-numbers; possible values are (none, left, right)
  numbersep=5pt,                   % how far the line-numbers are from the code
  numberstyle=\tiny\color{mygray}, % the style that is used for the line-numbers
  rulecolor=\color{black},         % if not set, the frame-color may be changed on line-breaks within not-black text (e.g. comments (green here))
  showspaces=false,                % show spaces everywhere adding particular underscores; it overrides 'showstringspaces'
  showstringspaces=false,          % underline spaces within strings only
  showtabs=false,                  % show tabs within strings adding particular underscores
  stepnumber=1,                    % the step between two line-numbers. If it's 1, each line will be numbered
  stringstyle=\color{mymauve},     % string literal style
  tabsize=2,	                   % sets default tabsize to 2 spaces
  title=\lstname                   % show the filename of files included with \lstinputlisting; also try caption instead of title
}


%%  Variablen
\newcommand{\authorName}{Ladina Bilgery \and Thomas Wieling}
\newcommand{\auftraggeber}{Interessengemeinschaft Wetterstation Arbon}
\newcommand{\auftragnehmer}{Interstaatliche Hochschule für Technik NTB}
\newcommand{\projektName}{Multiplattform-fähiges und barrierefreies User Interface und Datenmanagement für die Wetterstation Arbon}
\title{\projektName~(Bachelorarbeit)}
\author{\authorName}
\date{\today}

%%  Create a shorter version for tables. DO NOT CHANGE
\newcommand\addrow[2]{#1 &#2\\ }
\newcommand\addheading[2]{#1 &#2\\ \hline}
\newcommand\tabularhead{\begin{tabular}{lp{13cm}}
\hline
}
\newcommand\addmulrow[2]{ \begin{minipage}[t][][t]{2.5cm}#1\end{minipage}
   &\begin{minipage}[t][][t]{8cm}
    \begin{enumerate} #2   \end{enumerate}
    \end{minipage}\\ }
\newenvironment{usecase}{\tabularhead}
{\hline\end{tabular}}


% damit Bilder im aktuellen Kapitel dargestellt werden
\usepackage{placeins}
\usepackage{geometry}



%%  Beginn Dokument
\begin{document}
\pagenumbering{roman}
\input{Deckblatt}
\setcounter{page}{2}
\tableofcontents
\clearpage
\pagenumbering{arabic}



\documentclass{article}
\usepackage[utf8]{inputenc}
\usepackage[english]{babel}
 
\usepackage{multicol}
\setlength{\columnsep}{1cm}
 
\begin{document}
\begin{multicols}{2}
[
%  Title and authors
    \begin{center}
      {\huge\sffamily Multiplattform-fähiges und barrierefreies User Interface und Datenmanagement für die Wetterstation Arbon}\\
       \vspace{2ex}
       \textsc{Ladina Bilgery und Thomas Wieling}
    \end{center}
]
Hello, here is some text without a meaning. This text should show what a printed text will look like at this place. If you read this text, you will get no information. Really? Is there no information? sIs there...
\end{multicols}
 
\end{document}


\newpage
\section{Einleitung}

Einführung, Problem, Aufgabenstellung, Struktur des Berichts, Ziel des Berichts

Kurze Zusammenfassung der Anforderungen (gruppiert nach Thema)

\section{Grafische Benutzeroberfläche (Website)}
Die bisherige Anzeige der Wetterdaten basiert auf Adobe Flash, was nicht von allen Browsern unterstützt wird (siehe Fachmodul-Bericht~\cite{BilWie2018MUIu}). Die neue (2014) HTML5-Spezifikation ermöglicht, dynamische Grafiken zu erzeugen, die nativ, das heisst ohne Plugins und von allen Web-Browsern dargestellt werden können. In den folgenden Kapiteln wird aufgezeigt, wie die neue Webseite konzipiert wurde und welche Software-Bibliotheken eingesetzt wurden.


%% ############################################################################
%% Unterkapitel
%% ############################################################################
\subsection{Nutzungsanalyse der Website mittels Google Analytics}
\label{subsec:googleAnalytics}
Zuerst wurden die Zugriffsdaten auf die bestehende Website mittels \emph{Google Analytics} ausgewertet. Als Zeitraum wurde das gesamte Jahr 2017 gewählt. Auf der bisherigen Webseite gab es zwei Unterseiten \emph{Wetter Tourismus} und \emph{Wetter Wassersport}, die sich aber nur gering voneinander unterscheiden, z.B. in der Wahl der Einheit der Windgeschwindigkeit (km/h vs. kn). Die Zugriffsdaten sind in Abbildung\,\ref{img:google_mobile} dargestellt. Darin lässt sich erkennen, dass der Anteil an mobilen Geräten zwischen 50\,\% und 80\,\% beträgt. Dies widerspiegelt den Trend zum mobilen Internet und zeigt, wie wichtig die mobile Version einer Webseite ist.

\begin{figure}[htb!]
  \fbox{\includegraphics[width=\textwidth-2\fboxsep-2\fboxrule]{img/google_mobile}}
	\centering
	\caption{Zugriffstatistik auf die Wetterwebseite, Anteil mobil in orange}
	\label{img:google_mobile}
\end{figure}

\noindent
Die Auswertung von Google Analytics, wie in Abbildung\,\ref{img:google_browser} dargestellt, zeigt zudem, dass mehr als zwei Drittel der Zugriffe über den Safari- beziehungsweise Chrome-Browser erfolgen und dass die Webcam die mit Abstand beliebteste Funktion der Wetterstation ist.

\begin{figure}[htb!]
	\centering
	\fbox{\includegraphics[width=\textwidth-2\fboxsep-2\fboxrule]{img/google_browser}}
	\caption{Zugriffstatistik auf die Webseiten, aufgeteilt nach Browser}
	\label{img:google_browser}
\end{figure}



%% ############################################################################
%% Unterkapitel
%% ############################################################################
\subsection{Konzeption der multidevicefähigen Benutzeroberfläche}
Wie im Abschnitt\,\ref{subsec:googleAnalytics} aufgezeigt, steigt der Anteil an Zugriffen von mobilen Geräten. Es ist deshalb naheliegend, das mobile Design als Ausgangspunkt der Entwicklung zu verwenden. Diese Vorgehensweise nennt sich \emph{Mobile First}. Es ist ein Designkonzept, das von Luke Wroblewski 2009 das erste Mal vorgeschlagen wurde~\cite{LuWrmf}. Die \href{https://www.w3.org/TR/mobile-bp}{\emph{ Mobile Web Best Practices}} des \emph{W3C}\footnote{W3C: World Wide Web Consortium} empfiehlt zudem, dass sämtliche Informationen, die in der Desktop-Version zur Verfügung stehen, auch von der mobilen Seite aufgerufen werden können. Dieser Grundsatz nennt sich \emph{One Web Design}.

\begin{figure}[htb!]
  \fbox{\includegraphics[width=\textwidth-2\fboxsep-2\fboxrule]{img/scribbles}}
	\centering
	\caption{Erste Designentwürfe nach dem Mobile First Prinzip}
	\label{img:scribbles}
\end{figure}


\noindent
Beim Erstellen der ersten Designentwürfe (vgl. Abbildung \ref{img:scribbles}) zeigte sich, dass die einfachste Möglichkeit, die Daten auf allen Bildschirmgrössen darzustellen, darin besteht, die Informationen in kleine logische Blöcke zu unterteilen. Diese Blöcke können dann je nach Bildschirmgrösse verschieden angeordnet werden. Dieses Kachel-Design wird häufig verwendet, um das Gestaltungsgesetz der Zusammengehörigkeit umzusetzen und eignet sich hervorragend für das responsive Grid-Design, welches im Abschnitt \ref{subsec:responsiveFactors} erklärt wird.


\subsubsection{Auswahl der CSS-Bibliothek}
Sämtliche Artefakte sollen möglichst einfach gehalten werden. Für die Gestaltung der Benutzeroberfläche wurde beschlossen, eine CSS-Bibliothek zu verwenden. Es wurden drei verschiedene CSS-Bibliotheken evaluiert und mittels Nutzwertanalyse bewertet. Das Resultat ist in Tabelle\,\ref{table:css-bibliothek} ersichtlich. Die Erklärungen dazu sind im Anschluss daran aufgelistet.

% Nutzwertanalyse
\begin{table}[htbp!]
  \setlength\extrarowheight{3pt} % for a more "open" look
  \begin{tabularx}{\textwidth}{|>{\RaggedRight\hspace{0pt}}p{3.5cm}|p{2.5cm}||X|X|X|X|}

  \hline
  & \bfseries Gewichtung
  & \bfseries \href{https://www.w3schools.com/w3css/default.asp}{W3.css}
  & \bfseries \href{https://purecss.io/start/}{Pure.css}
  & \bfseries \href{http://getbootstrap.com/docs/4.1/getting-started/introduction/}{Bootstrap} \\

  \hline
  \textbf{Grid Design}
  & 4
  & 2
  & 3
  & 3 \\

  \hline
  \textbf{Card-Design}
  & 3
  & 3
  & 1
  & 3 \\

  \hline
  \textbf{Dokumentation}
  & 3
  & 3
  & 2
  & 2 \\

  \hline
  \textbf{Footprint}
  & 2
  & 3
  & 3
  & 1 \\

  \hline
  \textbf{Funktionsumfang}
  & 1
  & 3
  & 2
  & 2 \\

  \hline
  \hline
  \textbf{Total}
  & -
  & 14
  & 11
  & 11 \\

  \hline
  \textbf{gewichtet}
  & -
  & 35
  & 29
  & 31 \\

  \hline
  \end{tabularx}
  \caption{Nutzwertanalyse der evaluierten CSS-Bibliotheken}
  \label{table:css-bibliothek} % label muss NACH caption stehen!!!!
\end{table}


\paragraph*{Kriterien für die Auswahl}
Folgende Kriterien für die Auswahl der CSS-Bibliothek wurden berücksichtigt:
\begin{itemize*}
\item Grid Design: Unterstützt die Bibliothek das Grid Design und mehrere Displaygrössen, um responsive Webseiten zu gestalten?
\item Card Design: Unterstützt die Bibliothek das Card Design?
\item Dokumentation und Usabilty: Sind die Funktionen zentral und verständlich dokumentiert und ist eine Suchfunktion vorhanden?
\item Footprint: Die CSS-Bibliothek wird auf Mobilgeräte heruntergeladen und sollte deshalb nicht allzu gross sein
\item Funktionsumfang: Sind sämtliche benötigten Funktionen vorhanden oder müssen Zusatzbibliotheken eingebunden werden?
\end{itemize*}


\paragraph*{Gewichtung der Kriterien}
Die Gewichtungsskala für die Kriterien reicht von 1 (weniger wichtig) bis 4 (sehr wichtig). Für die Auswahl der CSS-Bibliothek wurde folgende Gewichtung vorgenommen:
\begin{itemize*}
\item 4 Grid Design: nötig zum Erstellen von modernen responsiven Webseiten
\item 3 Card Design: erhöht die Wahrnehmung der Zusammengehörigkeit
\item 3 Dokumentation: vereinfacht Erstellung und Anpassung
\item 2 Footprint: Bandbreite ist heutzutage nicht mehr so kritisch
\item 1 Funktionsumfang: es werden keine Spezialfunktionen benötigt
\end{itemize*}


\paragraph*{Bewertungsskala für die Kriterien}
Als mögliche Werte für die genannten Kriterien werden die Zahlen 1 (schlecht) bis 3 (gut) verwendet. Im Folgenden wird festgelegt, wann welcher Wert bei den einzelnen Kriterien vergeben wird.
\begin{itemize*}
\item Grid Design
  \begin{enumerate*}
  \item keine Unterstützung bzw. weniger als 3 Displaygrössen
  \item volle Unterstützung, min. 3 Displaygrössen
  \item volle Unterstützung, min. 4 Displaygrössen
  \end{enumerate*}
\item Card Design
  \begin{enumerate*}
  \item keine Unterstützung
  \item kein Card Design, aber ähnliches Prinzip
  \item volle Unterstützung
  \end{enumerate*}
\item Dokumentation und Usabilty
  \begin{enumerate*}
  \item Keine zentrale Dokumentation vorhanden
  \item Zentrale Dokumentation vorhanden, unübersichtlich, keine Suchfunktion
  \item Zentrale Dokumentation vorhanden, ausführlich und verständlich
  \end{enumerate*}
\item Footprint
  \begin{enumerate*}
  \item Grösser als 100kB
  \item Zwischen 50kB und 100kB
  \item Kleiner als 50kB
  \end{enumerate*}
\item Funktionsumfang
  \begin{enumerate*}
  \item nicht alle nötigen Funktionen vorhanden
  \item Sämtliche nötigen Funktionen vorhanden, mehrere Dateien
  \item Sämtliche nötigen Funktionen vorhanden, eine Datei
  \end{enumerate*}
\end{itemize*}


\paragraph*{Entscheidung}
Als Sieger stellt sich die W3.css-Bibliothek heraus, da sie mit Ausnahme der Beschränkung auf 3\,Displaygrössen alle Kriterien vollständig erfüllt.


\subsubsection{Responsive Webdesign}
\label{subsec:responsiveFactors}
Beim \emph{Responsive Webdesign} passt sich die Benutzeroberfläche automatisch der Bildschirmgrösse des Endgeräts an. Die Idee des Responsive Webdesigns wurde 2010 von Ethan Marcotte in einem Artikel des Magazins \emph{A List Apart}~\cite{EthMarRWD} veröffentlicht. Hintergrund war, dass man nicht für jedes neue Gerät eine eigene Webseite erstellen wollte. Marcotte schreibt von drei Faktoren, die ein responsive Webdesign voraussetzt: \emph{fluid grids}, \emph{flexible images} und \emph{media queries}. Fluid grids bedeutet, dass sich die Anordnung der Elemente dynamisch der Bildschirmgrösse anpasst. Flexible images haben keine feste Grösse, sondern nutzen den ihnen zu Verfügung stehenden Platz optimal aus und die media queries sind die technische Basis. Im Folgenden wird das Konzept der fluid grids genauer erklärt.

\begin{figure}[htb!]
  \fbox{\includegraphics[width=\textwidth-2\fboxsep-2\fboxrule]{img/kacheln2}}
	\centering
	\caption{Anordnung der Kacheln auf grossen und kleinen Bildschirmen}
	\label{img:kacheln2}
\end{figure}

\noindent
Das Grid-System vereinfacht das Erstellen von responsiven Webseiten, indem es den Bildschirm beziehungsweise den zur Verfügung stehenden Platz unabhängig von dessen Grösse in mehrere gleich breite Spalten einteilt (siehe orange Markierung in Abbildung\,\ref{img:kacheln2}). Die Anzahl Spalten ist frei wählbar bzw. abhängig von der gewählten CSS-Bibliothek (bei w3.css sind es 12 Spalten). Über eine media query fragt die Webseite beziehungsweise der Browser die Grösse des Bildschirms in Pixel ab und bestimmt dann, wie viele Spalten breit ein Element sein soll (grüne Markierung in Abbildung\,\ref{img:kacheln2}). Die gleiche Kachel ist also auf einem grossen Bildschirm  \nicefrac{2}{12} breit und auf einem kleinen Bildschirm \nicefrac{6}{12}. Unter \emph{w3.css} sind drei Bildschirmgrössen vorgesehen (small, medium und large). Die Grenzen sind folgendermassen festgelegt:

\begin{description*}
  \item [Small:] s $\leq$ 600px (z.B. iPhone Hochformat)
  \item [Medium:] 600px < m $\leq$ 992px (z.B. iPhone Querformat, iPad Hochformat)
  \item [Large]: l > 992px (z.B. Desktop)
\end{description*}

\noindent
Den Kacheln kann nun über die Klasse (z.B. l2 m3 s6) jeweils ein bestimmtes Verhalten bei den drei Bildschirmgrössen vorgeschrieben werden. Die Kacheln mit den aktuellen Wetterdaten sind zum Beispiel auf grossen (large) Bildschirmen \nicefrac{2}{12}, auf mittleren Bildschirmen \nicefrac{3}{12} und auf kleinen Bildschirmen \nicefrac{6}{12} breit. Bei der Darstellung der Wetterdatenverläufe werden bei mittleren und grossen Bildschirmen zwei nebeneinander dargestellt (\nicefrac{6}{12}) und bei kleinen Bildschirmen alle untereinander (\nicefrac{12}{12} ), wie in Listing\,\ref{lst:kacheln} ersichtlich.

\vspace{3mm}
\begin{lstlisting}[label=lst:kacheln,caption=Konfiguration der Anzahl Kacheln abhängig von der Bildschirmgrösse, language=HTML5, style=htmlcssjs]
<!-- Aktuelle Daten -->
<div class="l2 m3 s6">...</div>
<!-- Datenverlauf -->
<div class="l6 m6 s12">...</div>
\end{lstlisting}
\vspace{3mm}

%% ############################################################################
%% Unterkapitel
%% ############################################################################
\subsection{Grafische Darstellung der aktuellen Wetterdaten}
Die Anzeigeelemente für die aktuellen Messwerte sollen ebenfalls mit Hilfe einer Bibliothek erstellt werden. Damit die Grafiken optimal mit dem responsiven Webdesign zusammenspielen, müssen sie ein flexibles Layout aufweisen, das heisst ihre Grösse dem vorhandenen Platz anpassen. Dies nicht nur beim Aufrufen der Seite, sondern auch beim Ändern der Fenstergrösse. Zudem muss es einfach möglich sein, die Werte zu aktualisieren.

\begin{figure}[h!]
  \fbox{\includegraphics[width=\textwidth-2\fboxsep-2\fboxrule]{img/gauges}}
	\centering
	\caption{Anzeigelemente (Gauges) mit JustGage/Raphaël}
	\label{img:gauges}
\end{figure}

\noindent
Die Anzeige sollte möglichst ein klares Design aufweisen, aber trotzdem individuell anpassbar sein. Eine sehr einfache Javascript-Bibliothek, die diese Anforderungen erfüllt, ist \emph{JustGage}\footnote{ \url{http://justgage.com}}. Die Grafiken lassen sich über eine Javascript-Datei konfigurieren, sind im vektorbasierten svg-Format und dadurch von praktisch allen Browsern darstellbar. JustGage basiert auf der \textit{Raphaël}\footnote{ \url{http://dmitrybaranovskiy.github.io/raphael/}} Javascript-Bibliothek.

\noindent
Die Konfiguration eines Anzeigelements (Gauge) ist in Listing\,\ref{lst:gaugeJS} beispielhaft dargestellt. Jedes Element besitzt eine \emph{id} und einen Wert (\emph{value}) sowie optionale Angaben wie Skalaanfangs- und Endwerte und Beschriftungszusätze.

\vspace{3mm}
\begin{lstlisting}[label=lst:gaugeJS,caption=Konfiguration einer Gauge in Javascript, language=JavaScript, mathescape, style=htmlcssjs]
var temperature_gauge = new JustGage({
  id:     'temperature_gauge',
  value:  initialValues.v1.data.temperature.value,
  min:    -10,
  max:     40,
  symbol: ${^\circ}$C,
  title:  'Luft'
});
\end{lstlisting}
\vspace{3mm}

\noindent
Um die Grafik anzuzeigen, muss nur ein \emph{div} mit derselben \emph{id} auf der Webseite erstellt werden, wie in Listing \ref{lst:gaugeHTML} dargestellt.

\vspace{3mm}
\begin{lstlisting}[label=lst:gaugeHTML,caption=Container für die Gauge auf der Webseite, language=HTML5, style=htmlcssjs]
<!-- Lufttemperatur -->
<div class="l2 m3 s6">
    <div><p>Temperatur</p></div>
    <div id="temperature_gauge" class="gauge"></div>
</div>
\end{lstlisting}
\vspace{3mm}


\noindent
Messwerte ohne Relation zu einem Maximalwert sind nicht als Gauge, sondern als einfacher Text dargestellt (siehe Abbildung\,\ref{img:icons}). Die Messgrösse wird dabei einerseits im Titel der Kachel angegeben, und andererseits grafisch als Symbol neben dem Messwert. Dazu wurde eine Icon-Bibliothek gewählt, die möglichst alle benötigen Icons enthält, kostenlos ist, und deren Grafiken im svg-Format vorliegen. Als geeignet schienen die \emph{Weather Icons} von Erik Flowers\footnote{ \url{http://erikflowers.github.io/weather-icons/}} .

% Die Messwerte sind einerseits über den Titel beschrieben, sie enthalten aber zusätzlich ein passendes Icons.

\vspace{3mm}
\begin{figure}[htb!]
  \fbox{\includegraphics[width=\textwidth-2\fboxsep-2\fboxrule]{img/icons}}
	\centering
	\caption{Ausgewählte Icons aus der Weather Icons Bibliothek}
	\label{img:icons}
\end{figure}



%% ############################################################################
%% Unterkapitel
%% ############################################################################
\subsection{Grafische Darstellung der Wetterdatenverläufe}
Die Wetterverlaufsdarstellung liefert einen Überblick über das Wettergeschehen der letzten beiden Tage, wie beispielhaft in Abbildung\,\ref{img:charts} dargestellt ist. Im Folgenden wird erläutert, welche Javascript-Bibliothek dafür ausgewählt wurde und wie die unterschiedlichen Messverläufe dargestellt werden.

% Die Samplerate beträgt 1h, d.h. die Daten werden aus der Tabelle der historischen Werte abgerufen.
% in diesem Kapitel: Auswahl JS-Bibliothek, y-Achse, Windrichtung, Windvergleich

\begin{figure}[h!]
  \fbox{\includegraphics[width=\textwidth-2\fboxsep-2\fboxrule]{img/charts}}
	\centering
	\caption{Darstellung der Messverläufe mittels Linien- und Balkendiagrammen}
	\label{img:charts}
\end{figure}


\subsubsection{Auswahl der Javascript-Bibliothek}
Für die Darstellung der Messwertverläufe wurde ebenfalls auf eine Bibliothek zurückgegriffen. Als Erstes wurden Javascript-Bibliotheken gesucht, die ein ansprechendes Design aufweisen. Übrig geblieben sind die drei Javascript-Bibliotheken: \emph{chartist}, \emph{Fusion Charts} und \emph{Google Charts}. \emph{D3} ist zwar eine beliebte Javascript-Bibliothek, wenn es um Visualisierungen im Web geht, hat aber keine vorgefertigten Diagramme und benötigt entsprechend viel Aufwand und Einarbeitungszeit, weshalb sie nicht berücksichtigt wurde. Die drei potentiellen Javascript-Bibliotheken wurden mittels Nutzwertanalyse einander gegenübergestellt. Das Resultat ist in Tabelle\,\ref{table:js-framework} aufgeführt. Die Auswahl und Gewichtung der Kriterien ist im Anschluss daran erklärt.

% Nutzwertanalyse
\vspace{3mm}
\begin{table}[htbp!]
  \setlength\extrarowheight{3pt} % for a more "open" look
  \begin{tabularx}{\textwidth}{|>{\RaggedRight\hspace{0pt}}p{3.5cm}|p{2.5cm}||X|X|X|X|}

  \hline
  & \bfseries Gewichtung
  & \bfseries \href{https://gionkunz.github.io/chartist-js/index.html}{chartist.js}
  & \bfseries \href{https://www.fusioncharts.com}{Fusion Charts}
  & \bfseries \href{https://developers.google.com/chart/}{Google Charts}\\

  \hline
  \textbf{Footprint}
  & 1
  & 3
  & 1
  & 2 \\

  \hline
  \textbf{Barrierefrei}
  & 2
  & 2
  & 2
  & 1 \\
  \hline
  \textbf{Anpassbarkeit}
  & 4
  & 2
  & 3
  & 2 \\

  \hline
  \textbf{Dokumentation}
  & 3
  & 3
  & 3
  & 3 \\

  \hline
  \textbf{Funktionsumfang}
  & 4
  & 2
  & 2
  & 2 \\

  \hline
  \hline
  \textbf{Total}
  & -
  & 15
  & 12
  & 13 \\

  \hline
  \textbf{gewichtet}
  & -
  & 44
  & 38
  & 41 \\

  \hline
  \end{tabularx}
  \caption{Nutzwertanalyse der evaluierten Javascript-Bibliotheken}
  \label{table:js-framework} % label muss NACH caption stehen!!!!
\end{table}


\paragraph*{Kriterien für die Auswahl}
Folgende Kriterien für die Auswahl der Javascript-Bibliothek wurden berücksichtigt:
\begin{itemize*}
\item Footprint: Dateigrösse der Bibliothek
\item Barrierefrei: Unterstützung für barrierefreien Zugang vorhanden?
\item Anpassbarkeit: Können die Diagramm auf einfache Weise parametrisiert werden?
\item Dokumentation: Sind die Funktionen zentral und verständlich dokumentiert und ist eine Suchfunktion vorhanden?
\item Kosten: Ist die Bibliothek kostenlos bzw. in einem vertretbaren Rahmen?
\item Funktionsumfang: Werden sämtliche benötigten Diagrammtypen zur Verfügung gestellt?
\end{itemize*}


\paragraph*{Gewichtung der Kriterien}
Die Gewichtungsskala für die Kriterien reicht von 1 (weniger wichtig) bis 4 (sehr wichtig). Für die Auswahl der Javascript-Bibliothek wurde folgende Gewichtung vorgenommen:
\begin{itemize*}
\item 1 Footprint: Heutzutage ist die Dateigrösse nicht mehr so entscheidend
\item 2 Barrierefrei: Ist bei Grafiken nur bedingt möglich
\item 4 Anpassbarkeit: Gewährleistet, dass die Daten wie gewünscht angezeigt werden können
\item 3 Dokumentation: Vereinfacht Erstellung und Anpassung
\item 4 Kosten: Muss erschwinglich sein
\item 4 Funktionsumfang: Möglichst alle Diagrammtypen sollen verfügbar sein
\end{itemize*}


\paragraph*{Bewertungsskala für die Kriterien}
Als mögliche Werte für die genannten Kriterien werden die Zahlen 1 (schlecht) bis 3 (gut) verwendet. Im Folgenden wird festgelegt, wann welcher Wert bei den einzelnen Kriterien vergeben wird.
\begin{itemize*}
\item Footprint
  \begin{enumerate*}
  \item Grösser als 500kB
  \item Zwischen 100kB...500kB
  \item Kleiner als 100kB
  \end{enumerate*}
\item Barrierefrei
  \begin{enumerate*}
  \item keine Unterstützung
  \item teilweise Unterstützung
  \item volle Unterstützung
  \end{enumerate*}
\item Anpassbarkeit
  \begin{enumerate*}
  \item Diagramme sind nur beschränkt anpassbar
  \item Die meisten Parameter sind anpassbar
  \item Alle Parameter sind anpassbar
  \end{enumerate*}
\item Dokumentation
  \begin{enumerate*}
  \item Keine zentrale Dokumentation vorhanden
  \item Zentrale Dokumentation vorhanden, unübersichtlich, keine Suchfunktion
  \item Zentrale Dokumentation vorhanden, ausführlich und verständlich
  \end{enumerate*}
\item Kosten
  \begin{enumerate*}
  \item Mehr als 10CHF/Monat
  \item Maximal 10CHF/Monat
  \item Kostenlos
  \end{enumerate*}
\item Funktionsumfang
  \begin{enumerate*}
  \item Säulen- und Balkendiagramme vorhanden
  \item Interpolation von fehlenden Werten möglich
  \item Pfeildiagramm für Windrichtungsanzeige vorhanden
  \end{enumerate*}
\end{itemize*}


\paragraph*{Entscheidung}
\emph{Chartist.js} stellt den besten Kompromiss zwischen Funktionsumfang, Einfachheit und Dokumentation dar. Insbesondere die Dateigrösse war ausschlaggebend, da die Webseite, wie in Abschnitt \ref{subsec:googleAnalytics} erklärt, häufig auf Mobilgeräten verwendet wird.


\subsubsection{Skalierung der y-Achse}
Zur Verlaufsdarstellung wird primär das Linien- und Balkendiagramm verwendet, wie in Abbildung\,\ref{img:charts} aufgezeigt. Für jede Grafik wurde entschieden, ob eine automatische Y-Achs-Skalierung sinnvoll ist oder nicht. Bei der Windgeschwindigkeit und den Sonnenstrahlungs- und Regenmesswerten wird bewusst eine fixe Skalierung verwendet, damit auf den ersten Blick erkennbar ist, ob der Wert eher hoch oder eher tief ist. Beim Luftdruck und den Temperaturen hingegen ist die Tendenz wichtig, weshalb möglichst die gesamte Höhe des Diagramms genutzt werden soll. Es wird daher eine automatische Y-Achs-Skalierung verwendet. Die Konfiguration erfolgt in einem einfachen Javascript-File, wie in Listing\,\ref{lst:charts}  dargestellt. Unter \emph{options} kann die y-Achse auf die gewünschten Werte fixiert werden.

% Code-Beispiel
\begin{lstlisting}[label=lst:charts,caption=Konfiguration eines Verlaufsdiagramms, language=HTML5, style=htmlcssjs]
//Regen
var Rain = {
	series: [dataRainHistoric]
};
var options = {
low: 0,
high: 30,
};
new Chartist.Bar('#rain-chart', Rain, options);
\end{lstlisting}

\subsubsection{Anzeige des Windrichtungsverlaufs}
Die Anzeige des Windrichtungsverlaufs ist nicht ganz trivial, da sie von 0 bis 360\,Grad geht und ohne Unterbruch wieder zu 0\,Grad. In der Praxis wird dazu häufig die Darstellung von Pfeilen verwendet, wie in Abbildung\,\ref{img:windrichtung} links dargestellt. Es konnte jedoch keine Bibliothek gefunden werden, die diese Darstellungsart als Template zur Verfügung stellt. Für die Anzeige der Windrichtung wird deshalb ein Punktdiagramm, wie in Abbildung\,\ref{img:windrichtung} rechts dargestellt, verwendet. Auf eine Eigenentwicklung wurde bewusst verzichtet, um die Webseite möglichst benutzerfreundlich und wartungsarm zu belassen.

\begin{figure}[htbp!]
	\centering
  \fbox{\includegraphics[width=\textwidth-2\fboxsep-2\fboxrule]{img/windrichtung}}
	\caption{Windrichtungsanzeige mittels Pfeilen (links) und Punkten (rechts)}
	\label{img:windrichtung}
\end{figure}

\subsubsection{Vergleich Prognose/Ist-Windgeschwindigkeit}
Insbesondere bei Seglern sind Windprognosedienste sehr beliebt. Sie liefern Windprognosen bis zu fünf Tage voraus. Bei allen Vorhersagediensten ist jedoch rückblickend nicht erkennbar, wie gut die Prognose war. Genau dies ist aber eine spannende Frage, weshalb dieser Vergleich auf der Webseite angezeigt werden soll. Die Messdaten der Wetterstation werden mit zwei kostenlosen Vorhersagediensten verglichen. Dabei wird jeweils die Dreitagesvorschau betrachtet. Für die Windvorhersage wurden folgende zwei Anbieter ausgewählt:
\begin{itemize*}
\item \href{https://openweathermap.org/city/2661731}{Openweathermap} (\url{https://openweathermap.org/city/2661731})
\item \href{https://www.windfinder.com/forecast/arbon}{Windfinder} (\url{https://www.windfinder.com/forecast/arbon})
\end{itemize*}

\noindent
Openweathermap ist einer der wenigen Anbieter, welcher eine kostenlose Web-API zur Verfügung stellt und Windfinder ist erfahrungsgemäss unter den Seglern eines der beliebtesten Tools, bietet aber keine API, sondern lediglich ein Widget an (siehe Abbildung\,\ref{img:windfinder} links). Die Vorhersagedaten müssen aus dem Windfinder-Widget mittels Web-Scrapper extrahiert werden, analog Listing\,\ref{lst:kttgCrawler} auf Seite\,\pageref{lst:kttgCrawler}.

\begin{figure}[h!]
	\centering
  \fbox{\includegraphics[width=\textwidth-2\fboxsep-2\fboxrule]{img/windfinder}}
	\caption{Widget von Windfinder und Windvergleichsdarstellung}
	\label{img:windfinder}
\end{figure}


Da die Vorhersagedaten nur in die Zukunft angezeigt werden, d.h. wie die Vorhersage von gestern war ist nicht mehr ersichtlich, müssen die Vorhersagedaten abgegriffen und gespeichert werden. Nach den ersten Auswertungen wurde festgestellt, dass die Wettervorhersagedaten von Openweathermap unbrauchbar waren. Sie lagen permanent zwischen 0 und 1 Knoten, auch wenn die Wetterstation 15 Knoten mass. Die Vorhersage von Openweathermap wurde deshalb nicht mehr weiterverfolgt. Das Diagramm des Windvergleichs ist in Abbildung\,\ref{img:windfinder} dargestellt (Vorhersage: blau; Messwert: rot).



%\Diskussionspunkt{- Eintagesvorschau}\newline
%\Diskussionspunkt{- Mittels Tableau? Eigen Webseite?}\newline
%\Diskussionspunkt{- Bild mit Vergleich von Messwerten, Vorhersage Windfinder und Vorhersage Openweathermap}\newline

% Graphen, der den heutigen Tag, links davon die letzten 14 Tage und rechts davon die nächsten drei Tage anzeigt.
% Pro Anbieter und Prognoseart (1Tag bzw. 3 Tage) wird ein eigener Graph erstellt.
% Aus diesen Gründen wird die Webseite zwar veröffentlicht, aber kein Link dazu auf der Webseite zur Verfügung gestellt.
% Werte aus der Stundenwert-Datenbank?

%% ############################################################################
%% Unterkapitel
%% ############################################################################
\subsection{Anzeige der aktuellen Sturmwarn-Situation}
\label{subsec:sturmwarnung}
% was ist der Sturmwarndienst und wie werden die verschiedenen Stufen dargestellt
Auf dem Bodensee gibt es einen Sturmwarndienst, der die Schiffsführer vor aufkommendem Sturm warnen soll. Der Sturmwarndienst wird vom Deutschen Wetterdienst in Zusammenarbeit mit MeteoSchweiz betrieben. Rund um den Bodensee sind dafür über 60 Sturmwarnleuchten installiert (Abbildung \ref{img:sturm2}). Es wird unterschieden zwischen \emph{Starkwindwarnung} und \emph{Sturmwarnung}. Die Starkwindwarnung weist auf starke Windböen zwischen 25\,Knoten und 33\,Knoten hin und wird durch Aufleuchten von orangefarbigen Blinklichtern mit ca. 40 orangefarbigen Blitzen pro Minute an den Sturmwarnleuchten signalisiert. Die Sturmwindwarnung kündigt das Auftreten von Windböen von 34 Knoten und mehr an und wird durch Aufleuchten von orangefarbigen Blinklichtern mit ca. 90 orangefarbigen Blitzen pro Minute an den Sturmwarnleuchten signalisiert~\cite{BSO1976}.

\begin{figure}[htbp!]
	\centering
  \fbox{\includegraphics[width=\textwidth-2\fboxsep-2\fboxrule]{img/sturm2}}
	\caption{Warnleuchten des Sturmwarndiensts auf dem Bodensee}
	\label{img:sturm2}
\end{figure}



\subsubsection{Eingeschränkte Bezugsquellen}
% Problem: Kostenpflichtige Informaiton
Der aktuellen Status der Sturmwarnung kann sowohl beim Deutschen Wetterdienst, als auch bei Meteoschweiz kostenpflichtig bezogen werden (ca. 1300CHF/Jahr). Als kostenlose Alternative gibt es nur zwei browserkompatible Quellen: Die eine ist die Warnseite\footnote{ \url{https://www.meteoschweiz.admin.ch/home.html?tab=alarm}}  von Meteoschweiz und die andere die Anzeige auf der Webseite\footnote{ \url{http://www.kttg.ch/kapo/htm/stwarn.shtml}} der Kantonspolizei Thurgau, siehe Abbildung \ref{img:sturmZeit} links. Eine kostenlose API steht nicht zur Verfügung. Meteoschweiz verbietet es zudem, Informationen von ihrer Website abzugreifen. Es wurde deshalb beim Amt für Informatik des Kantons Thurgau, welches die Seite der Kantonspolizei betreut, die Erlaubnis eingeholt, die Sturmwarndaten ihrer Webseite abgreifen zu dürfen (siehe Mail im Anhang\,\ref{anhang:email}) . Beim Vergleich der verschiedenen Quellen wurde festgestellt, dass die Warnleuchten jeweils mit einigen Minuten Verspätung gegenüber der Meldung von Meteoschweiz eingeschaltet werden, siehe Abbildung \ref{img:sturmZeit}. Da der Verzug nicht konstant ist, muss davon ausgegangen werden, dass die Warnleuchten manuell eingeschaltet werden.\\

\begin{figure}[h!]
	\centering
  \fbox{\includegraphics[width=\textwidth-2\fboxsep-2\fboxrule]{img/sturmZeit}}
	\caption[Zeitverzug zwischen Sturmwarnmeldung und Anzeige]{Zeitverzug zwischen der Warnmeldung von Meteoschweiz und dem Einschalten der Warnleuchten durch die Kantonale Notrufzentrale (KNZ) in Frauenfeld}
	\label{img:sturmZeit}
\end{figure}


\subsubsection{Eigener Sturmwarndienst mit Hilfe der Messwerte}
% Problem 2: Öffnungszeiten -> keine Warnung in der Nacht
Der Sturmwarndienst, wie in Abschnitt \ref{subsec:sturmwarnung} beschrieben, ist kein 24h-Service. Der Dienst ist nur tagsüber aktiv zu den aufgelisteten Warnzeiten~\cite{kapoStwrn}:

\begin{description*}
  \item[Sommer:] 1. April - 31. Oktober: 06:00 - 22:00 Uhr
  \item[Winter: ] 1. November - 31. März: 07:00 - 20:00 Uhr
\end{description*}

\noindent
% Problem 3: rechtliche Verbindlichkeit der Sturmwarnung -> Zuverlässigkeit der anzeigen
Es liegt nahe, auf Grund der eigenen Windmessdaten eine Sturmwarnung anzuzeigen bzw. einen Benachrichtungsservice, wie in Kapitel \ref{notifications} beschrieben, anzubieten. In der Bodensee-Schifffahrts-Ordnung (BSO) Art. 6.13, steht jedoch:

\begin{quote}
\flqq Bereits bei Starkwind- und Sturmwarnung muss der Schiffsführer die durch die Umstände gebotenen Massnahmen treffen.\frqq
\end{quote}

\noindent
 Da die Sturmwarnung auf dem Bodensee gesetzliche Pflichten mit sich bringt, wurde entschieden, nur die offiziellen Daten anzuzeigen und keine anderweitigen Sturmwarnungen.

% Meteoschweiz bietet zudem auf ihrer  \href{https://www.meteoschweiz.admin.ch/home/service-und-publikationen/beratung-und-service/meteoschweiz-app.html}{MeteoSchweiz-App} die Möglichkeit von Push-Nachrichten bei Windwarnungen.



%-> Meteoschweiz schickt in der Nacht keine SMS????? bzw. erstellt keine JSON????

% ins Kapitel API bzw. Cronjobs übernehmen
%Der Status der drei Teilabschnitte wird dann mintütlich in die Datenbank geschrieben. So kann der aktuelle Stand normal über die API abgefragt werden.

% nur im Vortrag erwähnen
% DAs JSON von Meteoschweiz kann leider nicht verwendet werden, da dessen URL bei jedem Update wechselt. Der genaue Pfad lässt sich nicht vorhersagen. Auf Grund des Zeitstempels lässt sich erahnen, dass es keine automatische Verbindung zwischen Meteoschweiz und den Sturmwarnleuchten am Bodensee gibt. Auf der Meteoschweiz-Seite ist z.B. der Zeitstempel 12:00 und auf der kttg.ch Seite 12:07. Wahrscheinlich erfolgt die Aktivierung manuell. Dies würde die Zeitdifferenz erklären.
% -> Foto Zeitstempel Meteoschweiz und kttg.ch
% Meteoschweiz leitet die Sturmwarnung an die Kantonspolizei Thurgau weiter. Diese schaltet die Sturmwarnung über eine eigne Software ein.


%% ############################################################################
%% Unterkapitel
%% ############################################################################
\subsection{Darstellung der historischen Daten}
Die historischen Daten der Wetterstation sollen öffentlich über die Website zugänglich sein. Die Darstellung soll möglichst interaktiv gestaltet sein und ein aussagekräftiges Bild des Wetterverhaltens aufzeigen. Bereits bei der alten Wetterstation wurden Daten gespeichert. Zudem liegt ein Excel-File vor mit den Pegeldaten der letzten 50 Jahre, welche ebenfalls auf der Webseite verfügbar sein sollen. Es gibt somit drei Grundlagen für die historischen Daten:

\begin{itemize*}
\item Historische Daten der neuen Wetterstation (2015 bis heute)
\item Historische Daten der alten Wetterstation (2005 bis 2012)
\item Exceltabelle mit historischen Pegeldaten (1953 bis 2005 )
\end{itemize*}

\noindent
Diese Daten und deren Anzeige auf der Webseite werden im Folgenden erläutert.

\subsubsection{Historische Daten der neuen Wetterstation}
\label{kap:Tableau}
 \href{https://public.tableau.com/de-de/s/}{\emph{Tableau Public}} ist ein Daten-Visualisierungsprogramm, welches es ermöglicht, auf einfache Weise Diagramme zu erstellen, welche interaktiv vom Benutzer filterbar sind. Der Benutzer kann zum Beispiel den Betrachtungszeitraum selbst wählen, worauf sich die Grafiken automatisch der Auswahl anpassen.

\begin{figure}[h!]
	\centering
  \fbox{\includegraphics[width=\textwidth-2\fboxsep-2\fboxrule]{img/tableau}}
	\caption{Darstellung der historischen Daten (Desktop und Mobile)}
	\label{img:tableau}
\end{figure}


\noindent
\emph{Tableau Public} ist kostenlos mit der Einschränkung, dass sämtliche Visualisierungen öffentlich sind. In unserem Fall ist dies kein Problem, da die Visualisierungen sowieso veröffentlicht werden. Der Vorteil von Tableau ist, dass die Visualisierungen sehr einfach angepasst werden können, wenn z.B. neue Messgrössen hinzukommen. Tableau bietet zudem die Möglichkeit, Darstellungen für verschiedene Bildschirmgrössen anzulegen. Die Umsetzung ist in Abbildung\,\ref{img:tableau} dargestellt und unter dem Link \url{https://dev.wetter-arbon.ch/wetterhistorisch} abrufbar.

%Nachteil 2: Nicht responsive -> Workaround nötig
Neben den vielen Vorteilen hat Tableau Public auch ein paar Nachteile. Das Dashboard kann zwar frei gestaltet werden, ist aber nicht responsive d.h. es passt sich nicht der Bildschirmgrösse an. Das Problem wurde so gelöst, dass zwei verschiedene Dashboards erstellt wurden, eines für die Anzeige auf einem grossen Bildschirm und eines für die Anzeige auf Mobiltelefonen (siehe Abbildung \ref{img:tableau}). Beim Aufruf der Webseite wird die Bildschirmgrösse abgefragt und das entsprechende Dashboard geladen wie in Listing \ref{lst:histViewport} dargestellt.

\vspace{3mm}
\begin{lstlisting}[label=lst:histViewport,caption=Auswahl des Dashboards anhand der Bildschirmgrösse, language=JavaScript, style=htmlcssjs]
(function () {
	var viewportWidth = $(window).width();
	if (viewportWidth > 767) {
		$('#area_3').load('application/php/tableauDesktop.php');}
	else {
		$('#area_3').load('application/php/tableauMobile.php');}
}());
\end{lstlisting}
\vspace{3mm}

% Nachteil 1: Keine direkte Verbindung zur Datenbank
\noindent
Ein weiterer Nachteil der Public-Version von Tableau ist, dass sie keinen Live-Zugriff auf die Datenbank ermöglicht. Die Daten müssen somit manuell periodisch auf den Tableau Public Server kopiert werden, damit sie im Dashboard verfügbar sind.

% Wenn mit dem Mauszeiger über einen Messpunkt gefahren wird, werden zusätzliche Informationen angezeigt.
% Für die Auswertung haben wir uns auf die Lufttemperatur, Sonnenstrahlung, Windrichtung und -geschwindigkeit sowie den Regen und Sonnenscheindauer konzentriert.


\subsubsection{Historische Daten der alten Wetterstation}
Um einen Überblick über die vorhandenen Messdaten der alten Wetterstation zu erhalten, wurden die Daten grafisch aufbereitet wie in Abbildung\,\ref{img:histAlt} dargestellt. Darin lassen sich diverse Messlücken erkennen wie zum Beispiel das komplette Jahr 2010 und 2012. Weiter sind zum Teil unplausible Grössenordnungen vorhanden wie zum Beispiel bei der Windgeschwindigkeit, wo die Werte im Bereich von 0 bis 1000 liegen und nicht klar ist, in welcher Einheit diese Daten vorliegen. Die Angaben der Wellenhöhe sind in einem sehr kleinen Bereich, sodass davon auszugehen ist, dass es sich um das Signalrauschen handelt und nicht um die effektiven Wellenhöhen. Alles in allem sind in diesen Daten nicht viele nützliche Informationen vorhanden. Es wurde deshalb beschlossen, auf der Webseite nur die historischen Daten der neuen Wetterstation d.h. ab 2015 anzuzeigen. Alleine die Pegeldaten werden verwendet und die historische Pegeldatenbank integriert, wie in Abschnitt \ref{subsec:pegelhistory} beschrieben.

\begin{figure}[h!]
	\centering
  \fbox{\includegraphics[width=\textwidth-2\fboxsep-2\fboxrule]{img/histAlt}}
	\caption{Übersicht über die gespeicherten Daten der alten Wetterstation}
	\label{img:histAlt}
\end{figure}



\subsubsection{Historische Pegeldaten}
\label{subsec:pegelhistory}
Auf der Webseite wird der Pegelverlauf der letzten Jahre grafisch dargestellt. Als Grundlage diente unter anderem ein Excel-File mit den Pegelständen von 1953 bis 2005. Von der alten Wetterstation liegen die Pegeldaten von 2005 bis 2009 vor und ab 2018 kommen die Messwerte der neuen Wetterstation dazu. Die Darstellung ist in Abbildung\,\ref{img:histPegel} ersichtlich. Sämtliche Datenquellen sind in einer gemeinsamen Tabelle in der Datenbank gespeichert. Die grafische Darstellung erfolgt wiederum mit Tableau, was die interaktiven Anzeige mit Filterung der Daten ermöglicht. Das Dashboard ist unter dem Link \url{https://dev.wetter-arbon.ch/pegelstand} erreichbar.

\begin{figure}[h!]
	\centering
  \fbox{\includegraphics[width=\textwidth-2\fboxsep-2\fboxrule]{img/histPegel}}
	\caption{Darstellung der historischen Pegelmesswerte (gefiltert)}
	\label{img:histPegel}
\end{figure}



%% ############################################################################
%% Unterkapitel
%% ############################################################################
\subsection{Automatische Aktualisierung der Anzeigeelemente}
Die Wetterstation erzeugt jede Minute einen Datenbank-Eintrag mit den aktuellen Messwerten. Damit eine geöffnete Webseite immer auf dem aktuellen Stand ist, wird eine poll-Funktion verwendet, die selbständig alle 60\,Sekunden von der API die aktuellen Werte abfragt, wie in Listing\,\ref{lst:poll} verkürzt dargestellt. Sobald die JSON-Daten eingetroffen sind, wird die Funktion \emph{updateData} aufgerufen, die wiederum die Anzeigeelemente und Textfelder aktualisiert. Die gesamte Aktualisierung wird asynchron mittels AJAX durchgeführt. Die Seite wird dabei nicht neu geladen, nur die Anzeigewerte.

\begin{lstlisting}[label=lst:poll,caption=Automatische Aktualisierung der Werte, language=JavaScript, style=htmlcssjs]
(function poll() {
        $.get('https://api.wetter-arbon.ch/v1/')
          .done(function(response) {updateData(response);})
          .always(function() {setTimeout(poll, 60000); });
})();
function updateData(res){
        pressure_gauge.refresh(res.v1.data.pressure.value);
        $("#wassertemp1m") = res.v1.data.watertemperature1m.value;
        // usw.
}
\end{lstlisting}

\noindent
Damit die aktuellen Messwerte beim Aufbau der Seite zur Verfügung stehen, werden sie bereits beim Rendern der php-Seite von der API abgerufen und als JSON in der Variable \emph{initialValues} gespeichert (siehe Listing\,\ref{lst:initialValues}).

\begin{lstlisting}[label=lst:initialValues,caption=Übergabe der Initialisierungswerte durch php, language=JavaScript, style=htmlcssjs]
<script>
  var initialValues = JSON.parse('<?echo file_get_contents("https://api.wetter-arbon.ch/v1/");?>');
  document.getElementById("pegel").innerHTML = initialValues.v1.data.waterlevel.value;
  // usw.
</script>

\end{lstlisting}

%% ############################################################################
%% Unterkapitel
%% ############################################################################
\subsection{Barrierefreier Zugang}
Die Wetterstation und ihre Webseite ist eine Dienstleistung der Stadt Arbon. Sie gehört der Bevölkerung und soll deshalb möglichst für alle zugänglich sein. Sowohl die \emph{Web Content Accessibility Guidelines}~\cite{w3cwcag} des W3C-Konsortiums, als auch die deutsche \emph{Barrierefreie-Informationstechnik-Verordnung}~\cite{BITVde} bieten diverse Inputs, wie die Bedienbarkeit und somit Zugänglichkeit einer Webseite verbesserte werden kann. Von einer verbesserten Zugänglichkeit profitieren nicht nur Menschen mit Einschränkungen. Es geht darum, die Webseite so zu gestalten, dass sie möglichst für alle Benutzergruppen zugänglich ist. Eine von Microsoft beauftragte Studie~\cite{ForresterResearch2004E:Abilities} der \emph{Forrester Research Inc.} schätzt, dass über 60 Prozent aller Computernutzer von Barrierefreiheit profitieren können und gemäss Einschätzung des Verfassers des Buchs \emph{Interface Design}~\cite{ThesmannStephan2016ID:U} wird Barrierefreiheit bald Standard sein.

\noindent
Die aktuellen \emph{Web Content Accessibility Guidelines}~\cite{w3cwcag} fordern die Einhaltung von vier Designprinzipien, welche in gesamthaft zwölf Richtlinien genauer spezifiziert sind.

\begin{itemize*}
\item Designprinzip 1: wahrnehmbar
\item Designprinzip 2: bedienbar
\item Designprinzip 3: verständlich
\item Designprinzip 4: robust
\end{itemize*}

\noindent
Aus den Richtlinien wurden diejenigen Anforderungen ausgewählt, die auf die Webseite der Wetterstation anwendbar und die im Rahmen des CMS\footnote{CMS: Content Management System} umsetzbar sind. Alle relevanten Anforderungen und deren Umsetzung werden im Folgenden beschrieben.

\subsubsection{Wahrnehmbarkeit}
% Richtlinie 1.1
\href{https://www.w3.org/Translations/WCAG20-de/#text-equiv}{\textbf{Richtlinie 1.1}} fordert, dass für alle Nicht-Text-Inhalte eine Textalternative zur Verfügung steht. Auf der Webseite mit den aktuellen Messwerten gibt es drei Typen von Nicht-Text-Inhalten: Gauges, Icons und Graphen. Bei den Gauges und Graphen werden fertige Javascript-Bibliotheken verwendet. Diese stellen leider keine Möglichkeit zur Verfügung, einen Alternativtext hinzuzufügen. Für die Icons, welche als \emph{img} gekennzeichnet sind, wurde das \emph{alt}-Attribut ausgefüllt, wie in Listing \ref{lst:altImg} beispielhaft dargestellt.

\begin{lstlisting}[label=lst:altImg,caption=Alternativtext für Icons, language=HTML5, style=htmlcssjs]
<img src="img/wi-humidity.svg" alt="Icon eines Barometers">
\end{lstlisting}

\noindent
Wie die verwendeten javascript-Bibliotheken bietet auch Tableau, mit welchem die interaktive Darstellung der historischen Daten erfolgt, keine Möglichkeit, die Visualisierung als Text darzustellen. Das bedeutet, dass eine Reihe von Informationen auf dem gegenwärtigen technischen Stand noch nicht durch alternative Texte barrierefrei dargestellt werden können. \newline

% Richtlinie 1.3
\noindent
\href{https://www.w3.org/Translations/WCAG20-de/#text-equiv}{\textbf{Richtlinie 1.3}} fordert, dass die Struktur d.h. die Zusammenhänge und die Bedeutung einer Webseite unabhängig der Formatierung erhalten bleiben. Man spricht hier auch von \emph{semantischem} Web. Auf der Webseite wurde eine hierarchische Struktur implementiert, siehe Abbildung \ref{img:semWeb}. Die Hierarchiestufe wird einerseits erkennbar durch die Stufe der Überschriften (h1...h3) und andererseits durch die Bezeichnung der verschiedenen Inhaltsblöcke. Auf der Webseite sind zwei Hauptblöcke erkennbar, sogenannte \emph{Sektionen}. Eine Sektion beinhaltet die aktuellen Messwerte und die andere die Messdatenverläufe. Eine \href{https://www.w3.org/TR/2011/WD-html5-20110525/sections.html#the-section-element}{\emph{Sektion}} ist eine thematische Gruppierung von Elementen. Innerhalb der Sektion gibt es mehrere \href{https://www.w3.org/TR/2011/WD-html5-20110525/sections.html#the-article-element}{\emph{Artikel}}, welche einen unabhängigen Informationsblock darstellen.
\newpage

\begin{figure}[h!]
	\centering
  \fbox{\includegraphics[width=\textwidth-2\fboxsep-2\fboxrule]{img/semWeb}}
	%\includegraphics[width=1\linewidth]{img/sturm}
	\caption{Umsetzung der Anforderung bezüglich Semantischem Web}
	\label{img:semWeb}
\end{figure}

% Richtlinie 1.4
\noindent
\href{https://www.w3.org/Translations/WCAG20-de/#visual-audio-contrast}{\textbf{Richtlinie 1.4}} fordert, dass die visuelle Darstellung von Text ein Kontrastverhältnis von mindestens 4,5:1 hat. Zudem muss es möglich sein die Textgrösse um bis zu 200 Prozent zu erhöhen, ohne dass dabei Inhalt oder Funktionalität verloren geht. Um diese Anforderung zu überprüfen, wurde das Firefox-Plugin \textit{tota11y}\footnote{\url{http://khan.github.io/tota11y/}} verwendet. Das Plugin ermöglicht es, auf einfache Weise zu überprüfen, ob die Webseite die Barrierefrei-Bediungungen erfüllt. Die Kontrastverhältnisse liegen gemäss diesem Tool zwischen 5.75 und 12.63 und erfüllen somit die Anforderung. Da die gesamte Webseite responsive ist, stellt die Vergrösserung auf 200\% kein Problem dar.


\subsubsection{Bedienbarkeit}
\href{https://www.w3.org/Translations/WCAG20-de/#keyboard-operation}{\textbf{Richtlinie 2.1}} fordert, dass alle Funktionen über eine Tastaturschnittstelle bedienbar sind und dass die Tastaturfokusanzeige sichtbar ist.
Da es sich bei der Webseite der Wetterstation primär um eine Anzeige handelt, ist dieses Anforderung nur für das Eingabeformular des Benachrichtigungsservice (siehe Kapitel\,\ref{notifications}) relevant. Das Eingabeformular ist problemlos mittels Tastatur bedienbar.\newline

\noindent
\href{https://www.w3.org/Translations/WCAG20-de/#navigation-mechanisms}{\textbf{Richtlinie 2.4}} fordert, dass Webseiten einen zweckbeschreibenden Titel aufweisen und dass Zwischentitel den Inhalt und Zweck des entsprechenden Blocks beschreiben. Wie in Abbildung \ref{img:semWeb} dargestellt, besitzt die Webseite sowohl einen Seitentitel, als auch Zwischentitel.

\newpage
\subsubsection{Verständlichkeit}
\href{https://www.w3.org/Translations/WCAG20-de/#minimize-error}{\textbf{Richtlinie 3.3}} fordert, dass Eingabefehler automatisch erkannt und dem Benutzer angezeigt werden. Zudem sind Korrekturvorschläge anzuzeigen. Diese Anforderung betrifft insbesondere das Formular des Benachrichtigungsservices, welches in Kapitel\,\ref{notifications} beschrieben wird. Diese Anforderung wird erfüllt, wie in Abbildung \ref{img:notificationFE} links auf Seite \pageref{img:notificationFE} dargestellt.


\subsubsection{Robustheit}
\href{https://www.w3.org/Translations/WCAG20-de/#ensure-compat}{\textbf{Richtlinie 4.1}} fordert, dass Inhalte, die mit Hilfe von Markup-Sprachen implementiert wurden,  vollständige Start- und End-Tags haben. Es sind keine doppelten Attribute vorhanden und alle IDs sind eindeutig. Diese Anforderung wurde ebenfalls vollständig umgesetzt. Beispielhaft ist dies in Listing \ref{lst:gaugeHTML} auf Seite \pageref{lst:gaugeHTML} ersichtlich.

\section{Sensoren}


%% ###################################################################################################
%%   Unterkapitel                                                                                                                                                                              #
%% ###################################################################################################
\subsection{Pegelsensor}
\Diskussionspunkt{Pegelmessung, Pegelberechnung, Wellenhöhenberechnung}
\Diskussionspunkt{Gegenüberstellung Messprinzipien, Vor- und Nachteile}


%% ###################################################################################################
%%   Unterkapitel                                                                                                                                                                              #
%% ###################################################################################################
\subsection{Strahlungssensor}
\Diskussionspunkt{Ziel: Sonnenstunden messen? Wen interessieren die Sonnenstunden?}
\Diskussionspunkt{Wann reicht Globalstrahlung, wann wird Direktstrahlung benötigt?}
\Diskussionspunkt{Was kann auf PV-Berechnungstools eingegeben werden?}


Das Pyranometer basiert auf dem Messprinzip eines Thermoelements. Die eintreffende Strahlung trifft auf einen Absorber, welcher erwärmt wird. Die Wärme „fliesst“ dann über das Gehäuse an die Umgebung ab. Die Strahlungsleistung ist proportional zum Wärmestrom bzw. zur Temperaturdifferenz vom Absorber zum Gehäuse. Die Temperaturdifferenz wird mit Thermoelementen gemessen. Um die Signalspannung zu erhöhen werden mehrere Thermoelemente in Reihe geschalten, welches Thermosäule genannt wird. Durch das thermische Messprinzip ist ein Pyranometer träge. Die Response time liegt bei wenigen Sekunden. Das schwarz-poröse Absorbermaterial muss eine hohe Langzeitstabilität insbesondre gegenüber kurzwelliger Strahlung aufweisen. Das spektrale Verhalten wird durch den Glasdom definiert. Bei Glas liegt dieser im Bereich von 350 bis 2800 nm. Durch Erhöhung der Güte des Doms (Quarzglas) kann der spektrale Bereich erweitert werden von 300 bis 3600 nm. Für Pyranometer existiert ein Standard nach WMO: ISO 9060, mit folgenden Güteklassen:

\begin{itemize}  
\item Secondary Standard
\item first class
\item second class 
\end{itemize}

\Diskussionspunkt{ISO 9060:1990 Solar energy -- Specification and classification of instruments for measuring hemispherical solar and direct solar radiation}

Die Serie SR05 ist die preiswerteste Serie von Pyranometern, die die Anforderungen der zweiten Klasse nach ISO 9060 erfüllt. SR05 misst die von einer ebenen Fläche empfangene Sonnenstrahlung in W/m2 aus einem Blickwinkel von 180 Grad. Es ist ideal für allgemeine Sonnenstrahlungsmessungen in (agro-)meteorologischen Netzen und PV-Monitoring. Das Pyranometer ist einfach zu montieren und zu installieren, insbesondere mit dem Kugelausgleichsmechanismus des SR05. Zur einfachen Integration stehen verschiedene digitale und analoge Ausgänge zur Verfügung.


\subsubsection{Berechnung der Sonnenstunden}
Zur Messung der Sonnenstunden gibt es gemäss  \flqq Guide to Meteorological Instruments and Methods of Observation\frqq ~\cite{WMO2014Gtmi}  fünf Messprinzipien, wobei die pyranometrische Methode, die einfachste und kostengünstigste Methode darstellt. Nach WMO ist die Sonnenscheindauer während eines bestimmten Zeitraums definiert als die Summe der Zeit, für die die direkte Sonneneinstrahlung 120 W m -2 übersteigt. Die physikalische Größe der Sonnenscheindauer (SD) ist Zeit. Die verwendeten Einheiten sind Sekunden oder Stunden. Der Messzeitraum (Tag, Dekade, Monat, Jahr usw.) ist ein wichtiger Zusatz zur Einheit. Sonnenstunden sollten mit einer Auflösung von 0,1 h gemessen werden.

Der wichtigste Zusammenhang zwischen Sonnenscheindauer und Globalstrahlung G wird durch die sogenannte Ångström-Formel beschrieben:
\Diskussionspunkt{Ångström-Formel }

Pyranometrische Methode: Pyranometrische Messung der globalen (G) Sonneneinstrahlung zur Abschätzung der Sonnenscheindauer. Art des Instruments: Ein Pyranometer in Kombination mit einem elektronischen oder computergesteuerten Gerät, das in der Lage ist 1 Minute globale (G) Sonneneinstrahlung zu liefern.

\Diskussionspunkt{ANNEX 8.B. ALGORITHM TO ESTIMATE SUNSHINE DURATION FROM 1 MIN GLOBAL IRRADIANCE MEASUREMENTS}




%% ###################################################################################################
%%   Unterkapitel                                                                                                                                                                              #
%% ###################################################################################################
\subsection{Wassertemperatur-Sensoren}
\Diskussionspunkt{Berechnung, Verlauf der Wassertemperatur in Abhängigkeit der Tiefe}
\Diskussionspunkt{Welcher Sensor muss auf Grund des Pegels ausgewählt werden -> Skizze}
\Diskussionspunkt{Offset des defekten Sensors -> Grafik}




\section{Serverseitige Implementierung und Datenbank-Architektur}
Die serverseitige Implementierung bildet die Schnittstelle zwischen dem Client das heisst dem Internet Browser und den Messdaten. Diese müssen von den Sensoren in regelmässigen Abständen abgerufen und in der Datenbank gespeichert werden. Da die Datenbank das Herzstück der Wetterstation darstellt muss sie auch entsprechend geschützt werden vor Datenverlust und oder -manipulation. Die einzelnen Komponenten sowie der Aufbau der Datenbank werden im Folgenden erläutert.

%% ############################################################################
%% Unterkapitel
%% ############################################################################
\subsection{Datenerfassung}
Die einzelnen Daten werden unterschiedlich erfasst, wobei mit Ausnahme der Messwerten des Kombi-Wettertransmitters sämtliche Abfragen mittels Skripten durchgeführt werden. Ein Übersicht des Datenflusses ist in Abbildung\,\ref{img:datenerfassung} dargestellt. Im Folgenden werden die einzelnen Skripte erklärt.

\begin{figure}[htbp!]
  \fbox{\includegraphics[width=\textwidth-2\fboxsep-2\fboxrule]{img/datenerfassung}}
	\centering
	\caption{Schematische Übersicht der Datenerfassung}
	\label{img:datenerfassung}
\end{figure}


\subsubsection{Erfassung der Messdaten des Kombi-Wettertransmitters}
Die Daten des Kombi-Wettertransmitters werden weiterhin von WeatherDisplay über eine virtuelle serielle Schnittstelle abgerufen, aufbereitet und im Minutentakt in die Datenbank gespeichert. Kann ein Datenbankeintrag aus irgendwelchen Gründen nicht erstellt werden, wird dies, wie später im Kapitel\,\ref{kap:Funktionsüberwachung} beschrieben, gemeldet. Weiter erzeugt Weather Display mehrere Text Files, welche unter anderem für die Aussenanzeige (Laufschrift am Hafengebäude) benötigt werden. Die Aussenanzeige ist aber nicht Bestandteil dieser Arbeit.

\subsubsection{Einlesen von Pegel-, Strahlungs, und Wassertemperaturmesswerten}
Strahlungs- und Pegelsensor sind an einem Web-IO angeschlossen (siehe Datenblatt im Anhang\,\ref{Spec_webio}), welches die analogen Werte (in Milliampere) per Web-Schnittstelle zur Verfügung stellt. Die Abfrage erfolgt einmal pro Minute durch ein Python-Skript, wie in Listing \ref{lst:webIo} dargestellt. Die Temperatursensoren (PT100-Elemente) werden ähnlich erfasst mit dem Unterschied, dass nicht der Widerstandswert sondern direkt Grad Celsius geliefert wird. Als Schnittstelle wird an Stelle des Web-IO ein Web-Thermograph eingesetzt (siehe Datenblatt im Anhang\,\ref{Spec_webthermograph}).

\begin{lstlisting}[label=lst:webIo,caption=Python-Script zur Web-Abfrage des Pegel-Messwerts, language=python, style=py]
try:
    buffer = StringIO()
    c = pycurl.Curl()
    c.setopt(c.URL, 'http://webcam.wetter-arbon.ch:50506/single1')
    c.setopt(c.WRITEDATA, buffer)
    c.perform()
    c.close()
    requestMeasurement = buffer.getvalue()
\end{lstlisting}

\noindent
Der URL-Request wird mittels cURL ausgeführt. Die Abfrage ist für alle genannten Sensoren die gleiche, mit Ausnahme der Portnummer, welche sich je nach Sensor unterscheidet. Die Abfragen sind nach dem try... catch - Prinzip aufgebaut, sodass das Skript weiterläuft auch wenn ein Web-Interface nicht antwortet.



\subsubsection{Abgreifen der Sturmwarndaten} % screen scrapping
Die Sturmwarndaten werden periodisch mittels Python-Skript ausgelesen. Als Tool kommt die Python-Bibliothek \href{https://www.crummy.com/software/BeautifulSoup/}{\emph{BeautifulSoup}} zum Einsatz. Das Auslesen von Webseiten wird auch \emph{web scraping} genannt. Die Idee dahinter ist, die gewünschte Information auf Grund ihrer Element-Bezeichnung (z.B. \texttt{<td>}), ihres Attributs (z.B. \texttt{class="titelfett"}) oder Hierarchistufe (z.B. \texttt{tr:nth-of-type(4)}) eindeutig zu identifizieren. Um diese eindeutige Identifikation herauszufinden wurde das Google Chrome-Plugin \href{https://selectorgadget.com/}{\emph{SelectorGadget }} verwendet. Listing \ref{lst:kttgCrawler} zeigt die gesamte Abfrage. Darin ist auch das Problem dieser Methode gut erkennbar. Ändert die URL oder die Bezeichnung bzw. Struktur der Seite, liefert das Skript nicht mehr die gewünschten Daten. Da aber das Web-Scrapping die einzige kostenlose Möglichkeit darstellt um die Daten zu erhalten, muss dieses Risiko akzeptiert werden.


\begin{lstlisting}[label=lst:kttgCrawler,caption=Web-Scrapper für die Sturmwarndaten, language=python, style=py]
page = requests.get('http://www.kttg.ch/kapo/htm/stwarn.shtml')
soup = BeautifulSoup(page.text, 'html.parser')
# Einschaltzeit auslesen
soup.select('table tr:nth-of-type(4) td'):
# Status auslesen
soup.select('.titelfett strong'):
\end{lstlisting}


\subsubsection{Windvorhersagedaten}
Die Abfrage der Datenban stellt sich als recht kompliziert heraus. Sodass entschieden wurde eine VIEW einzusetzten. Diese wird dynamisch bei Aufrufen erzeugt und enthält die gewünschten Daten. Der SQL-Befehl für die Erzeugung der VIEW ist vereinfacht in Listing \ref{lst:viewForecast} aufgeführt. Primär gibt es zwei Bediungungen. Erstens sollen nur die Dreitageverhersagewerte verwendet werden und zweitens sollen die Vorhersagewerte verwendet werden, die am nächsten bei der aktuellen Zeit liegen. Da das Vorhersageintervall 3 Stunden beträgt ist die Verhersage maximal 1.5 Stunden verschoben.

%View erstellen
\begin{lstlisting}[label=lst:viewForecast,caption=Erzeugung der VIEW für den Forecast-Vergleich, language=SQL, style=htmlcssjs]
SELECT *
FROM 'tblcompwindfinder'
WHERE
(datetime + interval 3 day) <= now() #heute vor drei Tagen
AND
abs(timediff(datetime,now())) <= 13000) #innerhalb +/- 1.5h
ORDER BY datetime DESC
LIMIT 3
\end{lstlisting}


%% ############################################################################
%% Unterkapitel
%% ############################################################################
\subsection{Datenspeicherung}
Für die Datenspeicherung stellt Hostpoint, der Webhosting-Provider der Wetterstation, seinen Kunden eine MariaDB Version 10.1 mit dem Administrationstool phpMyAdmin zur Verfügung. MariaDB ist eine relationales Open-Source-Datenbankverwaltungssystem und basiert auf MySQL.

Die Datenbank wurde komplett neu aufgebaut. Im Folgenden werde die einzelnen Tabellen und deren Struktur erklärt.

%Für die Umstrukturierung wurde von der Methodik aus dem Artikel Grundlagen und Entwurf \cite{Datenbanken:GrundlagenUndEntwurf:VeikkoKrypczyk} gebrauch gemacht.
%Wie in Introduction to relational Databases \cite{IntroductionToRelationalDatabases:MariaDB} beschrieben wird,

\subsubsection{Aufbau und Inhalt der Datenbank}
Die Datenbank besteht aus mehreren Tabellen mit unterschiedlichem Inhalt. Die Auflistung sämtlicher Tabellen mit einer Beschreibung des Inhalts befindet sich in Tabelle\,\ref{table:dbtabellen}. Die obersten vier Tabellen werden zur Speicherung der Messwerte/Zusatzinformationen verwendet. Anschliessend kommen die drei Tabellen für die historischen Daten und zum Schluss zwei Tabelle, die für den Benachrichtungsservice und die Webcam benötigt werden.
%\Diskussionspunkt{Warum wurde keine zweite DB für Benachrichtigungsservice und Webcam erstellt? Zweite DB mit igwetter_Services?}

% Datenbank-Tabellen
\begin{table}[htbp!]
  \setlength\extrarowheight{3pt} % for a more "open" look
  \begin{tabularx}{\textwidth}{|>{\RaggedRight\hspace{0pt}}p{4.5cm}|X|}

  \hline
  %\textbf{Tabelle}
  & \bfseries Beschreibung \\

  \hline
  \textbf{tblwettertransmitter}
  & Messwerte des Wettertransmitters \\

  \hline
  \textbf{tblextsensors}
  & Messwerte der Sensoren (Bsp. Pegel) \\

  \hline
  \textbf{tblmisc}
  & Daten von Dritten (Bsp. Sturmwarnung) \\

  \hline
  \textbf{tblcompwindfinder}
  & Windvorhersagewerte von Windfinder \\

  \hline
  \hline
  \textbf{tbldatemaster}
  & Daten vom 1.1.2015 bis 31.12.2035 im Minutenabstand \\

  \hline
  \textbf{tblhistorical}
  & Messwerte zusammengefasst auf 1 Eintrag/h \\

  \hline
  \textbf{tblwaterlevelhistorical}
  & Pegelwerte von 1953 bis heute 1 Eintrag/d\\

  \hline
  \hline
  \textbf{tblnotifications}
  & Abos des Benachrichtigungsservices \\

  \hline
  \hline
  \textbf{tblqueue}
  & Warteschlange der Webcam \\

  \hline
  \end{tabularx}
  \caption{Datenbanktabellen und deren Inhalt}
  \label{table:dbtabellen} % label muss NACH caption stehen!!!!
\end{table}

\noindent
Die Tabellen haben untereinander keine Verknüpfung (Relation), sondern sind alle eigenständig. Das einzige, was sie verbindet ist der Zeitstempel des Messzeitpunkts. Es wird deshalb auf die Erstellung eines ER-Modells, wie in Datenbanken Grundlagen und Design \cite{FrankGeisler2011mitpu} beschrieben, verzichtet. In Abbildung\,\ref{img:tabellenstruktur} ist beispielhaft die Struktur der Tabelle \emph{tblextsensors} aufgeführt. Als Primärschlüssel wird jeweils der Zeitstempel des Messzeitpunkts verwendet. Da der Primärschlüssel ein UNIQUE-Wert ist, wird verhindert, dass es zwei Einträge mit dem selben Zeitstempel gibt. Das komplette relationale Datenmodell mit allen Entitäten und deren Attributen befindet sich in Anhang\,\ref{anhang:relationalesDatenbankmodell}.

\begin{figure}[htbp!]
  \fbox{\includegraphics[width=\textwidth-2\fboxsep-2\fboxrule]{img/tabellenstruktur}}
	\centering
	\caption{Struktur der Tabelle \emph{tblextsensors}}
	\label{img:tabellenstruktur}
\end{figure}

%Die Normalisierung ist, wie in Datenbanken Grundlagen und Design \cite{FrankGeisler2011mitpu}, ein Prozess mit deren Hilfe die Datenbankstruktur optimiert wird und hilft dabei Datenredundanzen zu vermeiden. Da bei der Datenbank nur das Datum redundant ist, ist eine Normalisierung nicht notwendig.
%Das relationale Datenmodell unterscheidet sich in der Struktur nicht bedeutend vom ER-Modell. Der Unterschied  ist, dass die Primärschlüssel und die Datentypen festgelegt werden. Die sogenannten Schlüssel sind im relationalen Datenmodell auch ein wichtiges Merkmal. Bei zukünftigen Datenbankeinträgen sind entscheidend die Schlüssel, so kann verhindert werden, dass für einen gewissen Zeitpunkt nochmals Datensätze geschrieben werden.
%Die drei konzipierten Tabellen konnten wie gewünscht umgesetzt werden. Der Code für die Umsetzung der Tabellen, kann aus dem Anhang \ref{anhang:Datenbankcode} entnommen werden.\\


\subsubsection{Aggregation der historischen Daten}
Bei der Wetterstation fallen pro Minute rund 40 Datenpunkte an, die gespeichert werden. Pro Jahr sind dies über 21 Millionen Datenpunkte. Damit die Anzeige der historischen Messwerte nicht so viele Daten laden muss, werden die Messdaten periodisch zusammengefasst.
\newline

% Tabelle mit Minutenwerten
% Tabelle mit Tageswerten
% Wie funktioniert das Zusammenfassen der Daten -> SQL-Befehle

% \paragraph{Konzept}
\noindent
Die minütlich gespeicherten Messdaten werden einmal pro Stunde zusammengefasst und in die Tabelle mit den historischen Werten \emph{tblhistorical} geschrieben. Für die Aggregation wird die Median-Funktion verwendet um den Einfluss von Messfehlern zu reduzieren. Einmal pro Tag werden zudam die Pegeldaten des gesamten Tags gemittelt und in die Tabelle \emph{tblwaterlevelhistorical} geschrieben, wie in Abbildung\,\ref{img:historical} dargestellt.

\begin{figure}[htbp!]
  \fbox{\includegraphics[width=\textwidth-2\fboxsep-2\fboxrule]{img/historical}}
	\centering
	\caption{Konezpt der Datenaggregation}
	\label{img:historical}
\end{figure}


% \paragraph{LEFT JOIN}
Das Script, welches die minütlichen Daten zu Stundendaten aggregiert greift nicht direkt auf die Messwerttabellen zu, sondern auf sogenannte VIEWs. Eine VIEW ist eine virtuelle Tabelle. Sie enthält keine Daten, sondern verweist auf die zugrundeliegenden Basistabellen. VIEWs ermöglichen es komplexere Abfragen zu vereinfachen und mehrere Tabellen über JOIN-Verknüpfungen zu einer einzigen Tabelle zusammenzufassen.

\begin{figure}[htbp!]
  \fbox{\includegraphics[width=\textwidth-2\fboxsep-2\fboxrule]{img/leftjoin}}
	\centering
	\caption{Konezpt der Datenaggregation}
	\label{img:leftjoin}
\end{figure}


Die beiden VIEWs, die für die Aggregation benötigt werden beinhalten jeweils genau 60 Einträge. Einen für jede Minute der vergangenen Stunde. Damit immer für jede Minute ein Eintrag vorhanden ist, auch wenn zum Beispiel der Sensor aus irgendeinem Grund keine Messwert geliefert hat, wurde ein LEFT JOIN mit der Datemaster-Tabelle erstellt. Diese Tabelle enthält sämtliche Datum/Zeit Einträge von 2015 bis 2035 im Minutenabstand.



%Bevor sich Gedanken um die Datensicherheit gemacht werden, sollten die Bedingungen an den Speicherplatz klar sein. Während der Laufzeit werden grosse Mengen an Daten in die Datenbank geschrieben. Vor der Neukonzipierung werden täglich 1440 Datensätze gespeichert. Das bedeutet jede Minute einen Datensatz. Ein Datensatz beinhaltet 65 Einträge, die gesamte relevante Datenbank igwetter wettertest benötigt (Stand 2018-04-24) 323.2  Mb. Die Tabelle wx data, beinhaltet die Minutenwerte des Wettertransmitters, benötigt davon (Stand 2018-04-24) 311.9 Mb, daraus erfolgt das ein Datensatz ca 0.025 Mb benötigt. Für den Speicherplatz, welcher 50 Gb bietet, stellt dies kein Problem dar. Hochrechnet reicht der Platz für die kommenden 45 Jahren.


% Die Herausforderung bei der Umsetzung war es, dass die historische Tabelle, welche aus den beiden Tabellen tblwettertransmitter und tblextsensors bestehen, mittels einer Query zusammenzubringen.

% Das Problem war die Zeit. Die Daten des Wettertransmitters werden Konfigurationsbedingt, nicht auf die Minute genau, sondern 31 Sekunden später geschrieben. Bei einem LEFT JOIN, welches über die Zeit geht, werden auch die Sekunden angeschaut. Nach mehreren gescheiterten Versuchen, ist es anschliessend gelungen, eine passende Query wie in \ref{lst:LeftJoinQuery} zu entwickeln.

% Zudem wird eine historische Tabelle erstellt, diese soll, die zu stündlich aufbereiteten Daten aus den beiden Tabellen enthalten.
% Um die historische Tabelle ohne Ausfälle zu füllen wird eine Tabelle entstehen, welche alle Zeitstempel von 2015 bis 2030 beinhaltet. Welche Datenpunkte übernommen werden, kann aus dem Anhang \ref{anhang:Datenbankschema} entnommen werden.\\

% Beim Modell der historischen Daten sieht das ganze anders aus (siehe Abb. \ref{img:ER_Modell historische Daten}). Hier beinhaltet jeder Zeitstempel, den Median, sowie die Extremwerte der Daten vom Wettertransmitter und die der externen Sensoren.

% Zusätzlich sollen die Daten, welche weiterhin im Minutentakt geschrieben werden, so aufbereitet werden, dass Benutzer auf der neu erstellen historischen Webpage die Wetterdaten der vergangenen Jahre einsehen können.


% \begin{lstlisting}[label=lst:LeftJoinQuery,caption=Json Struktur, language=JavaScript, style=htmlcssjs, mathescape]
% SELECT * FROM `DateMaster`
% LEFT JOIN `tblwettertransmitter`
% ON MINUTE(dt) = MINUTE(datetime)
% WHERE ((`DateMaster`.`dt` > (now() - interval 2 hour))
% AND((`tblwettertransmitter`.`datetime` > (now() - interval 2 hour))
% AND (hour((now() - interval 1 hour)) = hour(`tblwettertransmitter`.`datetime`)))
% AND (hour((now() - interval 1 hour)) = hour(`DateMaster`.`dt`))
% AND (`DateMaster`.`dt` < now()))
% order by `DateMaster`.`dt` desc
% \end{lstlisting} */


\subsubsection{Zeitsynchronisation}
% welche Daten erhalten von wo einen Zeitstempel?
% woher erhalten diese Geräte die Zeit? Zeitserver?

\Diskussionspunkt{Grafik einfügen}


\subsubsection{Datumsformatierung}
Die Norm DIN ISO 8601\footnote{DIN ISO 8601: Informationsaustausch - Darstellung von Datum und Uhrzeit} standardisiert die Darstellung von Datum und Zeit. Das internationale Datumsformat muss entweder als \texttt{2018-07-29T15:34:30} oder aber, wie empfohlen, mit der Differenz zur Koordinierten Weltzeit (UTC) \texttt{2018-07-29T15:34:30+02:00} angegeben werden. In der Datenbank der Wetterstation wird jedoch das nichtstandartisierte Format \texttt{2018-07-29 15:34:30} mit Leerzeichen zwischen Datum und Zeit verwendet.


\begin{figure}[htbp!]
  \fbox{\includegraphics[width=\textwidth-2\fboxsep-2\fboxrule]{img/datetime}}
	\centering
	\caption{Formatierung des Datums in der Datenbank}
	\label{img:datetime}
\end{figure}

\Diskussionspunkt{Warum kann normiertes Format nicht angewendet werden?}

\subsubsection{Umgang mit der Zeitumstellung}
Da in der Schweiz Sommer- und Winterzeit herrscht, besteht auch die Problematik der Zeitumstellung. Für die Datenbank besteht das Problem, dass die Tabellen nur einen Datumzeit enthalten dürfen. So entsteht die Problematik im Winter, da wenn die zurück gestellt wird doppelte Einträge entstehen. Wird in die historische Daten geschaut, fällt auf das diese Zeitumstellung nie berücksichtigt wurde. Zudem ist die Zeitumstellung in der Nacht, also für viele Benutzer eher uninteressant. Somit spielt es keine Rolle, dass die Daten bei der Zeitumstellung verloren gehen.

\begin{figure}[htbp!]
  \fbox{\includegraphics[width=\textwidth-2\fboxsep-2\fboxrule]{img/sommerzeit}}
	\centering
	\caption{Formatierung des Datums in der Datenbank}
	\label{img:sommerzeit}
\end{figure}


%% ############################################################################
%% Unterkapitel
%% ############################################################################
\subsection{Automatisierung der repetitiven Aufgaben (Cronjobs)}

%wiederkehrende Aufgaben – sogenannte Cronjobs – zu automatisieren.
% Der Cron-Daemon dient der zeitbasierten Ausführung von Prozessen in Unix und unixartigen Betriebssystemen wie Linux

Die Wetterstation basiert auf vielen repetitiven Aufgaben wie zum Beispiel das minütliche Einlesen der Messdaten. Linux, welches auf dem Hostpoint-Server verwendet wird, bietet mit dem Cron-Daemon ein Werkzeug um zeitbasiert Befehle beziehungsweise Skripte (Cronjobs) auszuführen. Eine Liste sämtlicher Cronjobs, die die Wetterstation verwendet ist in Tabelle \ref{table:cronjobs} dargestellt. Um einen Cronjob zu definieren muss einerseits das auszuführende Skript angegeben werden und der Zeitpunkt, zu dem es ausgeführt werden soll. Es kann zwischen Minute, Stunde und Tag gewählt werden. Ein Stern (*) bedeutet zu jeder Minute/Stunde/Tag. Die historischen Daten (\emph{historical.py}) werden beispielsweise zu jeder Stunde jeweils um 5 Minuten nach zusammengefasst. Die Windvorhersage von Windfinder (\emph{forecastWindfinder.py}) werden alle drei Stunden abgefragt (02:37, 05:37 usw.).


%Viele Funktionen werden mit Cronjobs ausgeführt. Wie von Hostpoint \cite{Hostpoint:CronjobsEinrichten} dargestellt, werden Cronjob für wiederkehrende Abläufe verwendet. Anders formuliert, kann ein Script zu einem bestimmten Zeitpunkt automatisiert ausgeführt werden. Es wird angegeben zu welchem Zeitpunkt oder in welchem Intervall das Programm ausgeführt werden soll. Der Rest übernimmt dann anschliessend der Cronjob. In diesem Fall, das Auslesen der externen Sensordaten, das erstellen der historischen Daten und das auslesen der Sturmwarnung. Auf Hostpoint kann mittels Knopfdruck ein Cronjob erstellt werden. Dabei kann auch gleich konfiguriert werden, zu welcher Zeit ein Cronjob ausgeführt werden soll (Tabelle \ref{table:cronjobs}).


% Tabelle Cronjobs
\begin{table}[htbp!]
  \setlength\extrarowheight{3pt} % for a more "open" look
  \begin{tabularx}{\textwidth}{|X|>{\RaggedRight\hspace{0pt}}p{3.5cm}|X|>{\RaggedRight\hspace{0pt}}p{5.5cm}|}

  \hline
  %\textbf{Tabelle}
  \bfseries Minute
  & \bfseries Stunde
  & \bfseries Tag
  & \bfseries Befehl/Skript \\

  %\hline
  \hline
  37
  & 2,5,8,11,14,17,20,23
  & *
  &  forecastWindfinder.py \\


  \hline
  *
  & *
  & *
  & sturmwarnung.py \\

  \hline
  *
  & *
  & *
  & externSensors.py \\

  \hline
  *
  & *
  & *
  & notifications.py \\

  \hline
  5
  & *
  & *
  & historical.py \\

  \hline
  7
  & 0
  & *
  & historicalWaterlevel.py \\


  \hline
  \end{tabularx}
  \caption{Konfiguration der Cronjobs}
  \label{table:cronjobs} % label muss NACH caption stehen!!!!
\end{table}


Die erwähnten Cronjob sind die wichtigsten, werden diese nicht durchgeführt, werden auch keinen Daten ausgelesen bzw. erstellt. Neben diesen drei Cronjobs bestehen noch zwei weitere. Diese lesen die Wettervorhersage für den Vergleich aus und schreiben die Daten in die Datenbank.

%Rechnern welche nicht durch einen Dauerbetrieb gekennzeichnet sind, kommen meist andere Varianten wie anacron zum Einsatz.





%% ############################################################################
%% Unterkapitel
%% ############################################################################
\subsection{Datenbanksicherheit (Datenmanipulation und -verlust)}


\subsubsection{Angriffssicherheit}
Laut den OWASP\footnote{OWASP: Open Web Application Security Project} Top 10, einer Liste\footnote{ \url{https://www.owasp.org/index.php/Category:OWASP_Top_Ten_Project}}, welche die wichtigsten Sicherheitsrisiken von Web-Applikationen aufzeigt, ist SQL-injection seit Jahren auf Platz 1. SQL-Injection ist eine Methode eine Datenbankabfrage so zu manipulieren, dass der Angreifer Daten ausspähen oder verändern kann.

Bei der Wetterstation Arbon sind keine persönlichen oder sicherheitsrelevanten Daten in der Datenbank gespeichert. Somit ist diese aus Sicht eines potentiellen Angriffes eher uninteressant. Dennoch sollten die Daten hinreichend geschützt sein, um einen Datenverlust beziehungsweise eine Datenmanipulation zu verhindern.

Die Zugriffe auf die Datenbank sind so gestaltet, dass nur serverseitig darauf zugegriffen wird. Die Darstellung der Anzeigen auf der Webseiten werden über die API erstellt. Die API wiederum ist so aufgebaut, dass ein PHP-Skript auf dem Server die Daten aus der Datenbank abgreift und sie richtig formatiert. Somit kann sichergegangen werden, dass keine SQL-Injection möglich ist. Wie im Kapitel Daten der neuen Wetterstation mit Tableau \ref{kap:Tableau} erwähnt, müssen die Daten für das Tableau manuell übertragen werden. Dies bedeutet, dass für die historische Seite kein Datenbankzugriff notwendig ist. Lediglich bei der Warteschlang der Webcam und beim Benachrichtigungsservice werden Daten direkt in die Datenbank geschrieben beziehungsweise daraus gelesen. In diesem Fall werden die Eingaben maskiert wie in Listing\,\ref{lst:maskierung} dargestellt.

\begin{lstlisting}[label=lst:maskierung,caption=Maskierung von Datenbank-Eingaben, language=PHP, style=PHP]
$threshold = mysql_escape_string($_POST['threshold']);
\end{lstlisting}


\subsubsection{DB-Nutzer Konzept}
\Diskussionspunkt{welche Nutzer gibt es und welche Rechte haben sie -> theoretisch erklären}

\subsubsection{Backup}
Um die Datenbank gegen einen allfälligen Datenverlust zu sichern, ist ein Backup sinnvoll. Da alle Daten in der DB gleich relevant sind, sollten auch alle im Backup vorhanden sein. Dabei muss aber entschieden werden, ob ein tägliches Backup Sinn machen würde. Da die Wetterstation von einem Verein betrieben wird, ist es wichtig den Aufwand mit dem Ertrag zu vergleichen. Zusätzlich sollte das Backup nicht auf dem Server des Providers gespeichert werden, sollte etwas mit dem Server nicht in Ordnung sein, sind diese Daten auch weg. Deswegen ist es wichtig ein 'externes' Backup zu erstellen. Hostpoint bietet für das Backup verschiedene zwei Varianten: Backup mittels Cronjob und Backup auf Knopfdruck. Um den Aufwand klein zu halten, wird empfohlen ein monatliches Backup mit der Backup Funktion auf Knopfdruck zu erstellen und dieses auf einer Festplatte zu speichern (Abbildung \ref{img:backup}). Sollte trotzdem mal das Backup nicht funktionierten oder vergessen gegangen sein, kann auf den Hostpoint-Service zurück gegriffen werden. Hostpoint erstellt selber ein tägliches Backup, welches für 100 Franken wieder eingespielt werden kann. Somit können die Kosten für beispielsweise einigen Cloudaccount gespart werden.

\begin{figure}[htbp!]
  \fbox{\includegraphics[width=\textwidth-2\fboxsep-2\fboxrule]{img/backup}}
	\centering
	\caption{Backup auf Knopfdruck von Hostpoint}
	\label{img:backup}
\end{figure}


%% ############################################################################
%% Unterkapitel
%% ############################################################################
\subsection{Datenarchivierung und Speicherbedarf}
In der Datenarchivierung werden die historischen Daten aufbereitet und komprimiert. Dafür ist wird von der historischen Tabelle gebrauch gemacht. Mittels einem Cronjob werden die Minuten Daten komprimiert. Anschliessend bestehen zwei Varianten:

\begin{itemize*}
\item \textbf{Variante 1}\\
Die Daten aus den beiden Minutentabellen werden gelöscht, stehen somit nicht zu weiteren Zwecken zur Verfügung.
\item \textbf{Variante 2}\\
Die Daten aus den beiden Minutentabellen werden nicht gelöscht und können für spätere Anwendungen verwendet werden.
\end{itemize*}

Da beim Server der Speicherplatz kein ausschlaggebender Punkt ist, wurde für die zweite Variante entschieden.



%% ############################################################################
%% Unterkapitel
%% ############################################################################
\subsection{Funktionsüberwachung mit Mail-Service}\label{kap:Funktionsüberwachung}
Um diese Funktionen zu erstellen wird Gebrauch vom try, except Verfahren in Python gemacht. Zu Beginn wird der Code im try ausgeführt, tritt keine exception auf, wird das except übersprungen und der anschliessende Code ausgeführt. Tritt aber während dem try eine exception auf, wird der Code unterbrochen und der im except weitergeführt. Anschliessend wird der Code nach dem Exception Handling ausgeführt.\cite{ThePythonTutorial8.ErrorsAndExceptions:Python}

Um die Funktion der Programme/Skripte zu überwachen, erzeugen folgende Funktionen bei einem Fehler (exeption) eine Meldung:

\begin{itemize*}
\item Einlesen der Sensordaten (Wassertemperatur, Pegel, Strahlung)
\item Einlesen Wettertransmitter-Daten
\item Erstellung der stündlichen und täglichen historischen Daten
\item Einlesen der Sturmwarn- und Windprognosedaten
\item Benachrichtiungsservice
\end{itemize*}

Die aufgezählten Funktionen werden alle bis auf das Einlesen der Wettertransmitter Daten über einen Cronjob ausgeführt. Hostpoint bietet die Möglichkeit, dass sämtliche Textausgaben (print-Funktion) eines Cronjobs an eine bestimmte Mailadresse gesendet werden (Listing \ref{lst:printfunction}). Der Service wurde so konfiguriert, dass die Fehlermeldungen an die Mailadresse \emph{cronjob@wetter-arbon.ch}. Der Posteingang kann dann an die entsprechenden Personen weitergeleitet werden.

\begin{lstlisting}[label=lst:printfunction,caption=Beispiel für print Funktion, language=Python, style=py]
except Exception as e:
    print "Es ist ein Problem mit der Sonnenstrahlungs-Abfrage aufgetaucht: "
    print e
\end{lstlisting}

Für das Einlesen der externen Sensordaten sieht der Mailservice folgendermassen aus. Kann einer der Webservices vom Web-IO nicht erreicht werden, wird die folgende Mail generiert:

\begin{quote}
Es ist ein Problem mit (der Temperatur, dem Pegel, dem Strahlungssensor) aufgetreten. Exception: ...\\
\end{quote}

Je nach Sensor wird dieser genannt und das Problem welches aufgetreten ist. Beim Auslesen des Wettertransmitters ist diese Möglichkeit nicht direkt anwendbar. Hier wird beim erstellen der historischen Daten kontrolliert ob alle 60 Einträge der letzen Stunde vorhanden sind. Ist dies nicht der Fall, würde es bedeuten, dass das WeatherDisplay abgestürzt ist und neu gestartet werden muss. Die Meldung sieht folgendermassen aus:\\

\begin{quote}
Bitte starte das WeatherDisplay neu, es wurden nur (Anzahl Datensätze) Daten geschrieben.
\end{quote}

Auch beim schreiben in die historische Datenbank können Ausnahmen auftauchen. Ist dies der Fall wird folgende Meldung generiert:\\

\begin{quote}
Die historischen Daten können nicht geschrieben werden, es besteht folgendes Problem (Exception).
\end{quote}

\input{api}
\section{Webcam}

%% ###################################################################################################
%%   Unterkapitel                                                                                                                                                                              #
%% ###################################################################################################

\subsection{Warteschlange}


%% ###################################################################################################
%%   Unterkapitel                                                                                                                                                                              #
%% ###################################################################################################

\subsection{Selektive Zoombeschränkung}


%% ###################################################################################################
%%   Unterkapitel                                                                                                                                                                              #
%% ###################################################################################################

\subsection{Foto-Funktion}

\section{Benachrichtigungsservice}
% die Idee dahinter
Die Idee hinter dem Benachrichtigungsservice ist, dass die Benutzer per Mail informiert werden sobald der von ihnen definierte Grenzwert über- bzw. unterschritten wird. Dies kann zum Beispiel ein Segler sein, der wissen möchte wann genügend Wind vorhanden ist, oder ein Bootsbesitzer, der wissen will wann der Pegel unter einen bestimmten Wert fällt.

Der Benachrichtungsservice ist so aufgebaut, dass auf der Webseite ein Formular mit den gewünschten Daten ausgefüllt werden kann. Das Ganze funktoniert ohne Benutzeraccount. In Abbildung \ref{img/notificationKonzept} ist der Ablauf grafisch dargestellt: Der Benutzer füllt ein Formular aus, dessen Daten per POST an den Server gesendet werden. Der Server speicher die Daten zusammen mit einem Schlüssel (Hash) in der Datenbank. Damit sichergestellt ist, dass derjenige dessen E-Mail-Adresse angegeben wurde auch wirklich der Besteller ist, muss der Benachrichtiungsservice zuerst aktiviert werden. Der Server sendet dazu ein Mail mit dem Aktivierungslink an die genannte E-Mail-Adresse. Erst wenn dieser Link aufgerufen wurde, ist der entsprechende Service aktiv. Da es keinen Benutzeraccount gibt, wird in jedem Mail, das versendet wird ein Unsubscribe-Link mitgeschickt. So kann der Benutzer den Service selbständig wieder löschen.

\begin{figure}[h!]
  \fbox{\includegraphics[width=\textwidth-2\fboxsep-2\fboxrule]{img/notificationKonzept}}
	\centering
	\caption{Funktionsprinzip des Benachrichtungsservices}
	\label{img:notificationKonzept}
\end{figure}

Der Benachrichtungsservice selbst basiert auf einem Cronjob, der einmal pro Stunde läuft. Das Prinzip ist in Abbildung \ref{img/notificationKonzept} in violett dargestellt. Der Cronjob ruft sämtliche Einträge in der Notification-Tabell ab und vergleicht sie mit den Messwerten. Sobald eine Bedingung erfüllt ist, wir ein Notification-Mail an die hinterlegete E-Mail-Adresse gesendet. Neben der eigentlichen Meldung ist ein Unsubscribelink vorhanden, damit der Benutzer den Benachritigungsservice abbestellen kann.

%% ###################################################################################################
%%   Unterkapitel                                                                                                                                                                              #
%% ###################################################################################################
\subsection{Formular mit integrierter Überprüfung der Eingabe}
Das Formular des Benachrichtungsservices besteht aus vier Eingabe- bzw. Auswahlfeldern, wie in Abbildung \ref{img/notificationFE}, links, dargestellt. Zuoberst kann die gewünschte Messgrösse ausgewählt werden. Das Dropdown-Menü stellt sicher, dass nur Werte ausgewählt werden können, die von der Wetterstation zur Verfügung gestellt werden. Mit den Radio-Buttons kann ausgewält werden ob der Messwert grösser oder kleiner als der Grenzwert sein soll. Der Grenzwert wird als Zahl eingeben auf eine Stelle nach dem Komma genau. Zuletzt muss die E-Mail-Adresse, an welche die Benachrichtigung gesendet werden soll eingetragen werden. Durch Anklicken des Abonnieren-Buttons wird das Formular abgesendet.

\begin{figure}[h!]
  \fbox{\includegraphics[width=\textwidth-2\fboxsep-2\fboxrule]{img/notificationFE}}
	\centering
	\caption{Eingabeformular mit Fehlerüberprüfung (rechts)}
	\label{img:notificationFE}
\end{figure}

Bisher musst die clientseitige Formularüberprüfung manuell programmierte werden bzw. eine JavaScript-Bibliothek verwendet werden. Unter HTML5 ist die Formularüberprüfung direkt integriert. Über die Defintion des \textit{type} und gegebenenfalls einiger Zusatzangaben, wird die  Überprüfung automtisch beim Absenden des Formulars durchgeführt und bei Fehlern direkt eine Meldung angezeigt. Der Wertebereich auf unsere Webseite ist zum Beispiel eingeschränkt auf Zahlen von 1 bis 50, wie in Listing \ref{lst:HTML5form} aufgezeigt. Wir nun ein Wert ausserhalb dieses Bereichs eingegeben, so erhält der Benutzer eine Fehlermeldung wie in Abbildung \ref{img/notificationFE}, rechts dargestellt und das Formular wird nicht abgesendet.

% Überprüfung der Eingaben
\begin{lstlisting}[label=lst:HTML5form,caption=Integrierte Formularüberprüfung mit HTML5, language=HTML5, style=htmlcssjs]
<!-- Auswahlmenue -->
<select name="measurand" required>
  <option value="" disabled selected>Messgrösse auswählen</option>
  <option value="windspeed">Windgeschwindigkeit</option>
</select>

<!-- Radio Button -->
<input type="radio" value="greater" checked>

<!-- Zahlenbereich -->
<input type="number" min="1" step="0.1" max="50" value="6">

<!-- E-Mail Adresse -->
<input type="email" placeholder="max.mustermann@web.ch" required>
\end{lstlisting}



%% ###################################################################################################
%%   Unterkapitel                                                                                                                                                                              #
%% ###################################################################################################
\subsection{Identificaiton des Abos via URL}
Die Tabelle für die Speicherung der Abos in der Datenbank enthält neben den vom Benutzer eingegbenen Daten noch weitere Einträge. Dies ist einerseits der Status des Abos d.h. aktiv oder inaktiv, den Zeitstempel des letzen Mails, sodass maximal einmal pro Tag ein Mail versendet wird, sowie den eindeutigen, nicht erratbaren Schlüssel (Hash), welcher für sämtliche Datenbank-Operationen benötigt wird.

% Bilder der DB-Tabellenstruktur?

Neben der \textit{subscribe.php}-Seite gibt es noch eine \textit{verify.php}-Seite und eine \textit{unsubscribe.php}-Seite. Diese führen jeweils die gewünschten Änderungen in der Datenbank aus. Zur Identifizierung des Abos enthält der Link sowohl die E-Mail-Adresse, als auch den Schlüssel (Hash) des entsprechenden Abos. Der Link für die Aktivierung des Abos lautet beispielhaft wie in Listing \ref{lst:verify} aufgeführt.

\begin{lstlisting}[label=lst:verify,caption=Beispiellink für die Aktivierung eines Abos, language=Python, style=py]
https://dev.wetter-arbon.ch/application/php/verify.php?
email=ladina.bilgery@ntb.ch&
hash=81f27f8a7d8766c72c0307a31327c1fad9007c6c3d33724ad2a5c0a8fe0df33d
\end{lstlisting}

Auf diese Weise kann auf Grund der aufgerufen Seite (subscribe, verify, unsubscribe) und dem Paar E-Mail und Schlüssel die Datenbankoperation genau definiert werden.

% braucht es das?
\FloatBarrier

\newpage
\input{schluss}
\newpage
\bibliography{literatur}{}
\bibliographystyle{plain}
\newpage
\section{Anhang}
        
\begin{itemize}
	\item Schema aller Geräte
	\item  Datenbankschema komplett
	\item Schema Aufbau Back end
	\item  Schema Aufbau Webseite (screenbox)
	\item Links (URL, Git, GitPages, usw.)
	\item Analyse Google Analytics
	\item Anforderungen ink. Angabe ob erfüllt oder nicht -> Testresultate
	\item  Datenblätter
\end{itemize}


\end{document}
