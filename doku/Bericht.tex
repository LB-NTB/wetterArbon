\documentclass[a4paper,ngerman, 11pt, pagesize]{report}

%% Päambel
\usepackage[T1]{fontenc}
\usepackage[utf8]{inputenc}
\usepackage{babel}
\usepackage{cite}
\usepackage{xcolor}
\newcommand\Diskussionspunkt[1]{\textcolor{red}{#1}}

\usepackage{url}
\usepackage{hyperref}

% Grafikpaket laden
\usepackage{graphicx}

% Tabellen
\usepackage{booktabs}
\usepackage{longtable}

% pdf einbinden (A3)
\usepackage{nextpage}
\usepackage{afterpage}
\usepackage{pdfpages}
\usepackage{typearea}
\usepackage{pdfpages}

\usepackage{lscape}


% Quelltext
\usepackage{listings}
 \usepackage{color}
 
 \definecolor{middlegray}{rgb}{0.5,0.5,0.5}
 \definecolor{lightgray}{rgb}{0.8,0.8,0.8}
 \definecolor{orange}{rgb}{0.8,0.3,0.3}
 \definecolor{yac}{rgb}{0.6,0.6,0.1}
 
  \lstset{
   basicstyle=\scriptsize\ttfamily,
   keywordstyle=\bfseries\ttfamily\color{orange},
   stringstyle=\color{green}\ttfamily,
   commentstyle=\color{middlegray}\ttfamily,
   emph={square}, 
   emphstyle=\color{blue}\texttt,
   emph={[2]root,base},
   emphstyle={[2]\color{yac}\texttt},
   showstringspaces=false,
   flexiblecolumns=false,
   tabsize=2,
   numbers=left,
   numberstyle=\tiny,
   numberblanklines=false,
   stepnumber=1,
   numbersep=10pt,
   xleftmargin=15pt
 }


%%  Variablen
\newcommand{\authorName}{Ladina Bilgery \and Thomas Wieling}
\newcommand{\auftraggeber}{Interessengemeinschaft Wetterstation Arbon}
\newcommand{\auftragnehmer}{Interstaatliche Hochschule für Technik NTB}
\newcommand{\projektName}{User Interface und Datenmanagement für die Wetterstation Arbon}
\title{\projektName~(Fachmodul)}
\author{\authorName}
\date{\today}

%%  Create a shorter version for tables. DO NOT CHANGE 
\newcommand\addrow[2]{#1 &#2\\ }
\newcommand\addheading[2]{#1 &#2\\ \hline}
\newcommand\tabularhead{\begin{tabular}{lp{13cm}}
\hline
}
\newcommand\addmulrow[2]{ \begin{minipage}[t][][t]{2.5cm}#1\end{minipage}
   &\begin{minipage}[t][][t]{8cm}
    \begin{enumerate} #2   \end{enumerate}
    \end{minipage}\\ }
\newenvironment{usecase}{\tabularhead}
{\hline\end{tabular}}



%%  Beginn Dokument
\begin{document}
\pagenumbering{roman}
\input{Deckblatt}
\setcounter{page}{2}
\tableofcontents          
\clearpage
\pagenumbering{arabic}
%%%%%%%%%%%%%%%%%%%%%%%%%%%%%%%%%%%
%%  Einführung
%%%%%%%%%%%%%%%%%%%%%%%%%%%%%%%%%%%
\chapter{Einführung}

Im Fachmodul geht es darum, die heutige Situation, das Problem der Situation und die Lösung zu erarbeiten. In der heutigen Situation wird beschrieben, welche Komponenten das Projekt hat, in diesem Fall die der Wetterstation in Arbon. Bei der Problemanalyse wird aufgezeigt mit welchen Problemen die Wetterstation zu kämpfen hat. Im letzen Schritt, werden dann die Möglichkeiten zur Problemlösung aufgezeigt. Dies heisst aber nicht, dass die Probleme schon gelöst werden und in der Bachelorarbeit "nur" noch Programmiert bzw. eingebaut werden muss. Das Fachmodul wird folgende Komponenten enthalten:
 
\begin{itemize}  
\item Spezifikation
\item Risikoanalyse (Zeit, Umsetzung,..)
\item Entwicklungsprozess (z.B. V-Modell -> skizzieren)
\item Konzeptbeschreibung für die zu entwickelnden Artefakte
\item Konzept der Dokumenten- und Versionsverwaltung
\item Ein Dokument, welches die rechtlichen Ansprüche regelt
\item Ein Projektplan (Meilensteine)
\end{itemize}
\Diskussionspunkt{Ziel des Fachmoduls, Aufträge, Vorgehensweise, Vorstellung Wetterstation Arbon}

Die Wetterstation Arbon ist online erreichbar. Auf der Webseite der Station sind Informationen zum Wetter, Warnungen und die Geschichte der Station publiziert. Im Fachmodul, sowie später in der Bachelorarbeit wird nur ein Teil der Webseite und deren Komponenten gesprochen. Diese sind im Bild \ref{img:cms_struktur} auf Seite \pageref{img:cms_struktur} gelb umrandet. 
\begin{figure}[htbp]
	\centering
	\includegraphics[width=0.9\linewidth]{img/cms_struktur}
	\caption{Struktur der Webseite sichtbar im CMS}
	\label{img:cms_struktur}
\end{figure}


\Diskussionspunkt{Foto Wetterstation}


\Diskussionspunkt{Beispiel für eine Bildintegration inkl. Referenz darauf:}
\begin{figure}[htbp]
	\centering
	\includegraphics[width=0.9\linewidth]{img/grafik}
	\caption{eine Grafik ohne Sinn und Verstand}
	\label{img:grafik-dummy}
\end{figure}

Weiterhin wollen wir an dieser Stelle Bezug auf die Grafik
\ref{img:grafik-dummy} auf Seite \pageref{img:grafik-dummy} nehmen, was uns
hiermit gelungen sein dürfte. Latex passt die Seitenzahl aber auch die Nummer
der Grafik automatisch an, wir müssen uns um nichts kümmern.
\chapter{Hauptteil}
\section{IST-Zustand}
\Diskussionspunkt{Hardware und Software, Skizze, Übersicht, Verbindungen/Verknüpfungen untereinander}
\Diskussionspunkt{Bild}

\subsection{Installierte Komponenten}

Die Wetterstation Arbon besteht aus folgenden Sensoren bzw. Sensor-Einheiten:
\begin{itemize}  
\item Webcam
\item Kombi-Wetter-Transmitter
\item Wassertemperatur-Sensor
\item Pegel-Sensor (defekt)
\end{itemize}


Die Webcam ist 360 Grad drehbar, schwenkbar, verfügt über eine Zoomfunktion und kann ferngesteuert werden.  Der Kombi-Wetter-Transmitter vereint mehrere Sensoren in einem Gehäuse. Dies sind Windgeschwindigkeit und -richtung, Lufttemperatur, relative und absolute Luftfeuchtigkeit, Regenmenge und Luftdruck. Der Wassertemperatur-Sensor besteht aus mehreren PT100-Widerständen, die in einem Kunststoffrohr im Abstand von 20cm montiert sind. Bei den Temperaturwiderständen ist einer defekt. Der Wert dieses Sensors wird mit Hilfe der beiden Nachbar-Wiederständen interpoliert. Den defekten Temperaturwiderstand zu ersetzen ist zu aufwändig. Der Pegelsensor ist im gleichen Kunststoffrohr verbaut wie die Temperatur-Widerstände und misst den hydrostatischen Druck. Das Kunststoffrohr ist gegen den Seegrund hin offen und nach oben verschlossen.

Die Wetterstation ist auf einem Pfahl ausserhalb des Hafens Arbon montiert. Auf dem Pfahl befindet sich ein kleiner Schaltschrank, jedoch keine Auswertelogik. Sämtliche Daten werden in IP-Pakete verpackt und über eine Glasfaser-Leitung an den Server gesendet.

Die verbauten Komponenten sind in der Grafik \ref{img:HW-Aufbau} schematisch dargestellt.

\begin{figure}[htbp]
	\centering
	\includegraphics[width=1\linewidth]{img/HW-Aufbau}
	\caption{Hardware-Aufbau der Wetterstation Arbon}
	\label{img:HW-Aufbau}
\end{figure}



\section{Problemanalyse}
\Diskussionspunkt{Beschreibung, 
Begründung, 
Erkenntnisse aus IST-Analyse}

\subsection{Hardware}
Sowohl die Webcam, als auch der Kombi-Wetter-Transmitter funktionieren einwandfrei. Der defekte Temperaturwiderstand wird akzeptiert, da dessen Wert interpoliert werden kann. Der Pegel-Sensor hingegen ist defekt und muss ersetzt werden. Der bisherige Sensor nutzte das Prinzip der hydrostatischen Druckmessung. Für die Pegelmessung konnten wir drei verschiedene Messprinzipien eruiert, die eingesetzt werden können:

\begin{itemize}  
\item Hydrostatische Druckmessung
\item Ultraschall-Distanzmessung
\item Radar-Distanzmessung
\item Time-of-light-Distanzmessung
\end{itemize}




\section{SOLL-Zustand}
\Diskussionspunkt{Lösungsansätze = zu entwickelnde Artefakte, 
Resultat aus Literaturrecherche, 
Konzepte}





 
\section{Darstellungsprobleme auf der Webseite}
Die Webseite der Wetterstation Arbon besteht neben der Homepage aus zwölf Unterseiten. Für uns wichtig sind all jene, die mit den Sensordaten, der Webcam, oder der Datebank in Verbindung stehen. (hervorgehoben in Abb.\ref{img:sitemap}). Im folgenden werden diese Seiten und deren Probleme genauer erläutert.

\begin{figure}[h!]
	\centering
	\includegraphics[width=0.9\linewidth]{img/sitemap}
	\caption{Sitemap der Webseite}
	\label{img:sitemap}
\end{figure}


\subsection{Adobe Flash: Workaround mit Nachteilen}

% ################################
% Problem Adobe Flash
% ################################
Um die Daten des Wettertransmitters auszulesen, wird Weather Display \footnote{ \url{http://www.weather-display.com/index.php }} verwendet. Dieses Programm liest die Daten des Wettertransmitters und stellt sie anderen Programmen zum Beispiel in Form eines Text-Files zur Verfügung. Weahter Display Live (WDL) \footnote{ \url{http://www.weather-display.de }} liest nun eben dieses Text-File und erstellt damit eine Adobe Flash Animations wie in Abb.\ref{img:responsive} dargestellt (gelb markiert).

Adobe Flash war eine einfache Möglichkeit animierte Grafiken auf Webseiten darzustellen und wurde praktisch von allen Browsern, nach Installation des Plug-ins, unterstützt. Diverse Sicherheitslücken und der Umstand, dass es sich um eine proprietäre d.h. closed-source Software handelte, führten dazu, dass Apple entschied Adobe Flash auf ihrem Mobile-Betiebsystem iOS nicht mehr zu unterstützen. \cite{Apple:ThoughtsOnFlash} 

Sämtliche Adobe Flash Animationen können somit nicht auf iPhone und iPad angezeigt werden. Da ein Grossteil der schweizer Bevölkerung jedoch genau diese Mobilgeräte verwendet, wurde für die Wetterstation folgender Workaround geschaffen: Der Browser prüft zuerst, ob das Gerät Adobe Flash unterstützt. Wenn ja wird die normale Applikation geladen, wenn nicht wird ein Printscreen der Applikaiton geladen (Code: \ref{lst:flash})

\begin{lstlisting}[caption={Adobe Flash workaround für iOS},label={lst:flash},language=html]
if (swfobject.hasFlashPlayerVersion("1")) {document.write('
<iframe src="https://www.wetter-arbon.ch/WDL/index.html"></iframe>');} 
else {document.write('
<img class="pageImage" src="https://www.wetter-arbon.ch/WDL/WDL.png" />');}
\end{lstlisting}

% Problem Workaround
Der Nachteil dieses Workarounds ist jedoch, dass die Anzeige weder dynamisch noch interaktiv ist. Um die aktuellen Werte zu erhalten muss die Seite jeweils neu geladen werden. Die interaktiven Elemente sind unbrauchbar d.h. die Änderung von Einheiten, Anzeige von Rekordwerten und weiteren Graphen ist nicht möglich.

\begin{figure}[h!]
	\centering
	\includegraphics[width=1\linewidth]{img/responsive}
	\caption{Responsive Design; Problem Flash-Applikation}
	\label{img:responsive}
\end{figure}


% Lösung Flash
Mit dem aufkommen der Smartphones wurde Flash immer weiter verdrängt durch HTML5 und Javascript. Dadurch haben auch andere Hersteller von mobilen Geräten nachgezogen und auf Flash möglichst verzichtet. 

Der wichtigste Grund ist das Adobe ab 2020 Flash nicht mehr weiterentwickelt und keine Updates mehr herausgeben wird. Dies aufgrund der oben genannten Tatsachen und das zur heutigen Zeit vielmals Flash umgangen wird und andere Lösungen gebraucht werden. Aufgrund von dies wird ab 2020 auch kein Browser mehr das Plugin zulassen und Seiten welche Flashinhalte benutzen sind nicht mehr Vollständig verfügbar. Somit ist auch die Wetter-Arbon Seite von dieser Tatsache betroffen. \cite{Adobe:FlashTheFutureofInteractiveContent}\\


% ################################
% Problem Responsive Design
% ################################
\subsection{Wetterdaten ohne Responsive Design}
Die Webseite der Wetterstation ist mit dem Content-Management-System (CMS) \textit{Openfile64Light} der Firma Screenbox erstellt. Dieses unterstütz grundsätzlich responsive Design. Eigene Änderungen oder dynamische Inhalte, werden in openfile64Light als sogenannte Applikationen behandelt und in die Seite eingebettet. Unterstützt die eingebettete Applikation kein responsive Design, so wird dieser Teil einfach linear skaliert. Dies führt dazu, dass die Anzeige der Wetterstationsdaten auf einem Mobilgerät kaum mehr lesbar sind. (Abbildung \ref{img:responsive})





% ################################
% Problem Graphen
% ################################
\subsection{Verwirrende Graphen}
Die Graphen auf den beiden Wetterseiten verwirren teilweise mehr, als das sie einem klar eine die vergangenen Werte zeigen. Das Ziel in der BA ist, dass die Aussagen der Graphen auf dem ersten Blick klar sind.

% Windgeschwindigkeit
\subsubsection*{Automatische y-Skalierung von Graphen}
Die Graphen, bspw. der Windanzeige mit ihrer Richtung sowie die Stärke, sind schwer lesbar, da die Skalierung des Graphen automatisch die, je nach Windstärke, wechselt. Das Bild \ref{img:wind-geschw} zeigt auf wie die Skalierung der Y-Achse bei unterschiedlichen Achsen wechselt. Beim linken Graph geht die Skalierung von 0 bis 30 Knoten, beim rechten von 0 bis 14 Knoten. Das Problem ist, dass ein schnelles ablesen der Anzeige nicht möglich ist.

\begin{figure}[h!]
	\centering
	\includegraphics[width=1\linewidth]{img/wind-geschw}
	\caption{Anzeige der Windgeschwindigkeit mit variabler y-Skalierung}
	\label{img:wind-geschw}
\end{figure}



% Windrichtung
\subsubsection*{Windrichtung}

\Diskussionspunkt{Bild der Windrichtungsproblematik}

\Diskussionspunkt{Erklärung der Problematik}


% ################################
% Problem Sturmwarnung
% ################################
\subsection{Unsichere Sturmwarnung mit Öffnungszeiten}
Die Seite mir der Sturmwarnung, siehe \ref{img:sturm}, wurde anfangs November umgestellt aus Sicherheitsgründen, welche in der Problemanalyse erläutert werden. Zum jetzigen Zeitpunkt wird nur noch ein Link zur Verfügung gestellt um auf die kantonale Sturmwarnseite zu kommen. Die Daten dieser Seite werden mit dem deutschen Wetterdienst in Stuttgart sowie Meteo Schweiz erstellt und dienen auf der Webseite nur als Information. Zu beachten ist hierbei, dass die Sturmwarnungen Bürozeiten haben. D.h. konkret vom 1. April bis 31 Oktober zwischen 6 und 22 Uhr und vom 1. November bis 31. März zwischen 7 und 20 Uhr. Der deutsche Wetterdienst und Meteo Schweiz unterscheiden zwei verschiedene Kategorien. Zum einen starke Windböen zwischen 25 und 33 Knoten, dies wird 40 Blitze pro Minute an den Leuchten signalisiert. Zum anderen Sturmböen von 34 und mehr Knoten, welche mit 90 Blitze pro Minute signalisiert werden. Zusätzlich zu den beiden Kategorien wird der Bodensee in 3 verschiedene Zonen unterteilt, West, Mitte und Ost, wobei Arbon in zur Zone Ost gehört. Wie schon erklärt ist das Problem hierbei der HTTP Standard, viele der Webbrowser stellen die Unterstützung dieses Standards langsam aber sicher ein und werden dann nur noch HTTPS unterstützen\cite{Mozilla:DeprecatingNon-SecureHTTP}. Deswegen ist die Sturmwarnung nur über einen Link aufrufbar. 


\begin{figure}[h!]
	\centering
	\includegraphics[width=1\linewidth]{img/sturm}
	\caption{Sturmwarnung vom Kanton Thurgau}
	\label{img:sturm}
\end{figure}

Das Problem der Webseite und vorallem der verschiedenen Applikationen ist, dass viele Geräte Flash nicht mehr oder in naher Zukunft nicht mehr unterstützen. Zusätzlich ist die Lösung mit dem Screenshot der aktuellen Verhältnisse auch keine optimale Lösung. Auch sind die Schreibfehler, welche entdeckt wurden bei näherer Betrachtung auch nicht Vorteilhaft. Weiter ist die Wetterapplikation nicht nach dem Prinzip responsive Design aufgebaut, welches in der heutigen Zeit ein wichtiger Bestandteil einer Webseite ist. Zusätzlich zum Flash, von der Gebrauch gemacht wird sind die Anzeigen auf der Touristik bzw. Wassersport Seite unschön. 

Um die Webseite auf den neusten Stand der Technik zu bringen sollten folgende Änderungen durchgeführt werden.Die Flash-Software wird ausgemustert und die Applikation wird auf HTML5 und Javascript umgestellt. Die Webseite soll zudem im responsive Design entwickelt werden, damit auch auf mobilen Geräten die aktuelle Wetterlage sichtbar ist. Die dynamischen, sowie auch die teilweise statischen Anzeigen, werden wo möglich mithilfe der Javascript Bibliothek D3.js oder Google Charts erstellt, hiermit lassen sich ansehnliche und moderne Grafiken erstellen. Die Grafiken, sollten so gestaltet sein das auch Sehbehinderte Personen erkennen wie das Wetter momentan ist. Das heisst beispielsweise, dass die Farben auch für Farbenblinde unterscheidbar sein sollten oder blinde Personen anhand eines Vorleseprogramms erkennen wie das Wetter ist. Ein weiterer Punkt ist die Auswahl der Einheiten, diese sollen nach dem ersten Besuch gespeichert bleiben beim Client mithilfe von Webstorage.

\section{Die Sturmwarnung}\Diskussionspunkt{zu webseite hinzu oder einzelnes Kapitel zusammen mit notification?}

Die Seite mir der Sturmwarnung, siehe \ref{img:Sturmwarnung} auf Seite \pageref{img:Sturmwarnung} gibt es im eigentlichen nicht mehr, dies wurde beginn November umgestellt aus Sicherheitsgründen, welche in der Problemanalyse erläutert werden. Zum jetzigen Zeitpunkt wird nur noch ein Link zur Verfügung gestellt um auf die kantonale Sturmwarnseite zu kommen. Die Daten dieser Seite werden mit dem deutschen Wetterdienst in Stuttgart sowie Meteo Schweiz erstellt und dienen auf der Webseite nur als Information. Es sollte vor Ort beachtet werden ob die Warnlampen eingeschalten sind. Zu beachten ist hierbei, dass die Sturmwarnungen Bürozeiten haben. D.h. konkret vom 1. April bis 31 Oktober zwischen 6 und 22 Uhr und vom 1. November bis 31. März zwischen 7 und 20 Uhr. Der deutsche Wetterdienst und Meteo Schweiz unterscheiden zwei verschiedene Kategorien. Zum einen starke Windböen zwischen 25 und 33 Knoten, dies wird 40 Blitze pro Minute an den Leuchten signalisiert. Zum anderen Sturmböen von 34 und mehr Knoten, welche mit 90 Blitze pro Minute signalisiert werden. Zusätzlich zu den beiden Kategorien wird der Bodensee in 3 verschiedene Zonen unterteilt, West, Mitte und Ost, wobei Arbon in zur Zone Ost gehört. 
\begin{figure}[htbp]
	\centering
	\includegraphics[width=0.9\linewidth]{img/sturmwarnung}
	\caption{Sturmwarnung vom Kanton Thurgau}
	\label{img:Sturmwarnung}
\end{figure}

Wie schon erklärt ist das Problem hierbei der HTTP Standard, viele der Webbrowser stellen die Unterstützung dieses Standards langsam aber sicher ein und werden dann nur noch HTTPS unterstützen\cite{Mozilla:DeprecatingNon-SecureHTTP}. Deswegen ist die Sturmwarnung nur über einen Link aufrufbar. Des Weiteren wurde öfters der Wunsch nach einer SMS-Benachrichtigung geäussert, dies ist momentan nicht realisiert und soll auch teil unserer BA sein. 

Die Sturmwarnung soll wieder auf der Webseite sichtbar sein und der Besucher der Webseite soll die Möglichkeit haben sich für eine Notification zu registrieren. Dafür wurden 3 verschiedene Möglichkeiten ausgewählt und mit der Nutzwertanalyse ausgewertet. Ziel bei allen Möglichkeiten ist es, dass der Benutzer die Möglichkeit hat sich zu registrieren und Alarmkriterien zu bestimmen. Werden die gewählten Alarmkriterien erreicht bzw. wird eine Sturmwarnung herausgegeben, wird der Benutzer benachrichtigt. Hiermit soll die Möglichkeit gegeben werden, dass der Benutzer in Echtzeit informiert wird und somit keine Warnung oder sein "perfektes" Segelwetter verpasst. Für die evaluierung der Notifications wurde eine Nutzwertanalyse erstellt. Dies ist eine gute Möglichkeit, um verschiedene Lösungsansätze zu bewerten. Der Nachteil hierbei ist jedoch, dass die Bewertung sehr subjektiv ist. 

\begin{center}
\begin{tabular}{ |p{3.5cm}||p{1cm}|p{2cm}|p{3.5cm}|p{2.5cm}|p{1.5cm}|}
 \hline
 \multicolumn{6}{|c|}{Nutzwertanalyse} \\
 \hline
	Möglichkeiten & Kosten & Einfachheit & Programmieraufwand & Anpassbarkeit & Support\\
 \hline
	SMS & 1 & 4 & 3 & 3 & 5\\
	E-Mail & 5 & 4 & 5 & 5 & 1\\
	FacebookMessenger & 5 & 4 & 3 & 4 & 1\\
 
\hline
\end{tabular}
\end{center}

Aus der Nutzwertanalyse ist Sichtbar, dass die Benachrichtigung per E-Mail und Facebook Messenger die Lösungsansätze mit der höchsten Punktzahl sind . Der grösste Vorteil der beiden möglichen Lösungen sind, dass sie Kostenlos sind. Der Nachteil an Facebook Messenger ist, das nicht davon ausgegangen werden kann, dass jeder Benutzer ein Facebookprofil hat. 

\section{Datenbank}

\Diskussionspunkt{Beginn Einlesen in DB}
Verschiedene Arten von Datenbanken:
\begin{itemize}
\item relationale DB
\item hierarchisches Datenmodell
\item Netzwerkdatenmodell
\item Objekt relationale Datenbank
\end{itemize}

Das relationale Datenmodell ist das weit verbreiteste Modell, dass hierarchische wegen der beschränkten Anwendbarkeit kaum noch vorhanden.

Vorgehen Datenbankentwicklung:

\begin{itemize}
\item Externe Phase (Ermittlung der Informationsstruktur)
\item Konzeptionelle Phase (ER-Modell)
\item Logische Phase (relationales Datenmodell)
\item Physische Phase (Erstellung des Datenmodell)
\end{itemize}

Punkte zur Überlegung neuer Datenbankstruktur:
\begin{itemize}
\item Welche Daten?
\item In welchem Intervall?
\item Welche Tabellen? (Welche Daten zusammen?, eine grosse Tabelle?)
\item Tabellennamen?
\item 
\end{itemize}

Datentypen angeschaut auf w3schools:
\begin{itemize}
\item Welcher Datumstyp?

\end{itemize}

https://entwickler.de/online/datenbanken/datenbanken-grundlagen-und-entwurf-115676.html
\Diskussionspunkt{Ende Einlesen in DB}

\section{Die Webcam}
Zur Wetterstation Arbon gehört auch eine Webcam der Marke Axis. Diese ist auch wieder über ein Applikationsplugin in die Webseite integriert. Auf dieser können per Shortlinks sechs verschiedene Positionen angefahren werden:
\begin{itemize}  
\item Home
\item See Nord
\item Schloss
\item Hafeneinfahrt
\item Horn
\item Seerettung
\end{itemize}

Diese Positionen sind in der Betriebseigenen Software konfiguriert. Neben den voreingestellten Positionen, kann die Kamera auch frei Positioniert werden.\Diskussionspunkt{Bild Webcam} Diese freie Positionierung erfolgt über Pfeile, sowie die Plus und Minus am Bildrand, jenachdem welcher Button geklickt wird, sendet die Webseite das Kommando mittels HTTP an die Webcam. Die zusammenstellung der URL geschieht über ein Javascript, wie man dem folgenden Code entnehmen kann.
\begin{lstlisting}
 <div class="container webcam" id="webcam_585">
	<div class="up"></div>
	<div class="left"></div>
	<div class="right"></div>
	<img class="pageImage" src="https://webcam.wetter-arbon.ch/mjpg/video.mjpg" alt="" />
	<div class="down"></div>
	<div class="zoomOut"></div>
	<div class="zoomIn"></div>
	<span class="home">Home</span>
	<span class="SeeNord">See Nord</span>
	<span class="Schloss">Schloss</span>
	<span class="Hafeneinfahrt">Hafeneinfahrt</span>
	<span class="Horn">Horn</span>
	<span class="Seerettung">Seerettung</span>
</div>
<script type="text/javascript">
	function changeWebCam(command) {
		var urlAddition;
		
		switch (command) {
			case 'up':
			case 'down':
			case 'left':
			case 'right':
			case 'home':
				urlAddition = 'move=' + command;
				break;
				
			case 'zoomIn':
				urlAddition = 'rzoom=2500';
				break;
				
			case 'zoomOut':
				urlAddition = 'rzoom=-2500';
				break;
				
			case 'SeeNord':
			case 'Schloss':
			case 'Hafeneinfahrt':
			case 'Horn':
			case 'Seerettung':
				urlAddition = 'gotoserverpresetname=' + command;
				break;
				
		}
		
		console.log('changeWebCam');
		$.get('https://webcam.wetter-arbon.ch/axis-cgi/com/ptz.cgi?camera=1&' + urlAddition);
	}
\end{lstlisting}

In der Betriebseigenen Software lassen sich viele Parameter Konfigurieren, wie auch der Zoomfaktor. Diese ist jedoch aus Datenschutzgründen auf die 4-fache Vergrösserung limitiert, möglich wäre aber eine 216-fache Vergrösserung. Daraus wird schon deutlich das die Webcam mehr Potential hat und die Limitierung des Zooms in der BA so Konfiguriert werden, dass diese möglichst dynamisch sein. D.h., dass jenach Position der Zoomfaktor auch mit ändert. Der Zoom soll, vor allem auf den See hinaus in vollem Umfang benutzt werden können. Die Position der Kamera kann mit einer einfachen HTTP GET anfrage aufgerufen werden. Daraus wird ersichtlich in welche Richtung die Kamera zeigt und kann so, mittels einem einfachen Javascript den Zoom beschränken oder erweitern.
\section{Nutzeranalyse}
Welches sind die Nutzer und was sind deren Bedürfnisse

\section{Funktionale Anforderungen}
\begin{usecase}
  \addheading{Nummer}{Beschreibung} 
  \addrow{/FA10/}{Temperaturanzeige in Grad und Fahrenheit}
  \addrow{/FA20/}{Windgeschwindigkeitsanzeige in Knoten, Km/h, m/s, mph, Bft  }
  \addrow{/FA30/}{Luftdruckanzeige in hPa, mmHg, kPa, inHg, mb, }
  \addrow{/FA40/}{Windrichtung }
  \addrow{/FA50/}{Niederschlagsmenge in mm } 
\end{usecase}


\section{Nicht-Funktionale Anforderungen}
\begin{usecase}
  \addheading{Nummer}{Beschreibung} 
  \addrow{/FA10/}{Webseite soll im responsive Design erstellt sein}
  \addrow{/FA20/}{Webseite soll auch für Menschen mit beeinträchtigungen zur Verfügung stehen}
  \addrow{/FA30/}{Webseite soll mit HTML5 erstellt sein}
  \addrow{/FA30/}{Webseite soll mit JavaScript erstellt sein}
\end{usecase}


   
\chapter{Projektmanagement}
Wir wollen das Projektmanagement schlank halten um möglichst viel Zeit in die Entwicklung der Artefakte stecken zu können.
Dieser Grundgedanke hat uns bei der im Folgenden beschrieben Auswahl der Modelle und Prozesse geleitet.

\section{Vorgehensmodell}

Die Anforderungen an das Vorgehensmodell haben wir folgendermassen definiert:

\begin{itemize}  
\item wenig administrativer Aufwand, schlank
\item passend zur Projektgrösse
\item kompatibel mit den NTB-Vorgaben (Aufteilung Fachmodul, Bachelor-Arbeit)
\end{itemize}

Schnell merkten wir, dass die heutzutage beliebten agilen Vorgehensmodelle wie XP oder Scrum für uns ein Overkill darstellen und aus mehrerer Hinsicht nicht geeignet sind. Bei der Bachelor-Arbeit sind die Anforderungen im Fachmodul-Bericht definiert und ändern sich während der Bachelor-Arbeit nicht mehr. Die zu bearbeitenden Themen-Blöcke weisen untereinander nur sehr wenige Schnittstellen auf und können dadurch als eigenständige Teilprojekte das Modell durchlaufen. Unser Team besteht zudem nur aus zwei Personen, was den Koordinationsaufwand auf ein minimum reduziert.

\begin{figure}[htbp]
	\centering
	\includegraphics[width=0.9\linewidth]{img/royce-largePrograms}
	\caption{Vorgehensmodell nach Royce}
	\label{img:royce-largePrograms}
\end{figure}


Unsere Bedürfnisse deckt das Vorgehensmodell von Royce ~\cite{Royce1970}, welches in Abbildung  \ref{img:royce-largePrograms} dargestellt ist, am besten ab. Es besteht grundsätzlich aus einem sequentiellen Ablauf der Entwicklungsphasen, berücksichtigt dabei aber auch die Notwendigkeit von Rücksprüngen zur vorherigen Phase.
Die ersten Phasen von der Definition der "System Requirement" bis zu den ersten Gedanken zum Thema "Program Design" behandeln wir im Fachmodul. Der zweite Teil mit der genauen Definition des Programm Designs bis zum Betrieb der Software findet anschliessend während der Bachelor-Arbeitszeit statt.

\section{Entwicklungsprozess}

Den Entwicklungsprozess führen wir mit Kanban. Kanban basiert auf dem Pull-Prinzip d.h. jeder, der im Projekt arbeitet, holt sich selbst einen neuen Arbeitsauftrag, sobald er mit einem fertig ist. Die führt dazu, dass die Arbeiten speditiver abgewickelt werden und spart zudem die Stelle des Projektmanagers, der die Aufgaben verteilt.

\Diskussionspunkt{Bild Kanban-Board}

David Anderson \cite{AndersonDavidJ2011K:eC} hat das System Kanban, welches ursprünglich aus der Industrie kommt, auf die IT angepasst und dadurch das "Virtuelle Kanban System" entwickelt. Die grundlegenden Regeln daraus lauten:

\begin{itemize}  
\item Jede Karte ist eine Aufgabe
\item Die Aufgabe soll maximal 8-16h benötigen
\item Pro Spalte sind die Anzahl Karten limitiert
\item Eine neue Karte darf erst gezogen werden, wenn die vorherige fertig ist (Multitasking-Vermeidung)
\end{itemize}


\section{Projektplan für die Bachelor-Arbeit}
Der Zeitplan für die Bachelor-Arbeit ist in Abbildung \ref{img:terminplan} auf Seite \pageref{img:terminplan} dargestellt.
Im oberen Teil sind die allgemeinen Termine und Abwesenheiten aufgeführt. Der mittlere Teil zeigt die Arbeitsverteilung über das Semester und am Schluss kommen die Zeitaufwände für Doku und Meetings. Die Dokumentation wollen wir kontinuierlich erstellen, sodass wöchentlich ein entsprechender Block vorgesehen ist.
% Abbildung (A3)
\newpage
\clearpage
\pagebreak
\afterpage{ % Insert after the current page
\clearpage
\KOMAoptions{paper=a3, paper=landscape} 
\recalctypearea

\begin{figure}[htbp]
	\centering
	\includegraphics[width=1.1\linewidth]{img/terminplan} % ab heigth = 0.6 auf eigener Seite!
	\caption{Terminplan}
	\label{img:terminplan}
\end{figure}
\clearpage
\KOMAoptions{paper=A4,pagesize}
\recalctypearea
}

\section{Risikoanalyse}
\subsection{Risikoliste}
Ausgearbeitet mit dem Risikolexikon aus dem Buch xxxx, Risikoschablone
siehe ~\cite{AhrendtsFabian2008Il:w}.


\subsection{Risikoanalyse und Risikomatrix}





   
\section{Dokumentation}
\Diskussionspunkt{Versionsverwaltung, Dokumentationskonzept, Tools?, Printscreens}

%Allgemein
Für die Bachelor-Arbeit verwenden wir unterschiedliche Dokumentationswerkzeuge. Bei der Auswahl haben wir darauf geachtet, das die Tools kostenlos nutzbar und für sämtliche Plattformen verfügbar sind (Windows, Mac, iPad, usw.) Weiter war uns wichtig, dass die Tools untereinander kommunizieren können. 

% github / Trello / Toggl
Sämtliche Artefakte speichern wir auf \textit{github}. Wir haben somit eine automatische Versionierung der Dokumente und können unabhängig voneinander an den Dokumenten arbeiten. Die Planung bzw. Darstellung des Entwicklungsprozesses erledigen wir mir \textit{Trello}. Es ist ein intuitives Tool, welche diverse Integrationsmöglichkeiten mit den anderen Tools bietet. Für die Zeiterfassung verwenden wir \textit{Toggl}, welches mittels Plugin direkt in Trello integriert werden kann.

% Kommunikation nach aussen
Damit wir keine Besprechungsprotokolle verschicken müssen und dass alle Informationen für alle immer zugänglich sind, haben wir entschieden das Reporting in Form einer öffentlichen Webseite zu erstellen. github bietet mit \textit{GitPages} einen Hosting-Service an, der genau dies ermöglicht. Der Vorteil von GitPages ist, dass wir sämtliche Daten in einem einzigen Ort bzw. Repository vereint haben. Damit wir uns nicht mit Formatierung u.a. herumschlagen müssen und uns auf den Inhalt konzentrieren können, verwenden wir \textit{mkdocs} als Template Engine. Die Webseiten-Einträge können wir dadurch auf simple Art in Form von Markdown-Files erstellen.

% Kommunikation nach innen
Innerhalb des Teams nutzen wir das Kommunikationstool \textit{Slack}. Dieses ermöglich uns Konversationen als Chat aufzuzeichnen und nach Themen zu gruppieren. Weiter lassen sich Dokumente austauschen. Sämtliche git-Posts werden von Slack automatisch geloggt und können, falls gewünscht, als push-Notification angezeigt werden.
Das wöchentliche Team-Meeting findet über \textit{Skype} statt, da wir den regelmässigen mündlichen Austausch aus zentralen Punkt erachten.

% Bericht = LaTeX
Den Bericht werden wir in LaTeX verfassen. Wir haben uns für LaTeX entschieden, da wir uns auf den Inhalt konzentrieren können und das Layout automatisiert ist. Weiter ist LaTeX in der Wissenschaft weit verbreitet. Die Bachelor-Arbeit ist deshalb eine gute Gelegenheit uns in dieses Thema einzuarbeiten.


\section{Rechtliche Ansprüche}
siehe separates Dokument


%%%%%%%%%%%%%%%%%%%%%%%%%%%%%%%%%%%
%%  Schluss
%%%%%%%%%%%%%%%%%%%%%%%%%%%%%%%%%%%
\section{Schluss}
\Diskussionspunkt{Erkenntnisse, Einschätzungen}\\
\Diskussionspunkt{Vergleich FM-Planung zu FM-IST}




\bibliography{literatur}{}	
\bibliographystyle{plain}

\end{document}
