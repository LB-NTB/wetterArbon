\section{front end}

%% ###################################################################################################
%%   Unterkapitel                                                                                                                                      
%% ###################################################################################################

\subsection{Grafische Darstellung der aktuellen Wetterdaten}

\subsubsection*{HTML5}



%% ###################################################################################################
%%   Unterkapitel                                                                                                                                      
%% ###################################################################################################


\subsection{Grafische Darstellung der Wetterdatenverläufe}

\Diskussionspunkt{Samplerate, fixe Y-Achse, Barrierefreiheit}


%% ###################################################################################################
%%   Unterkapitel                                                                                                                                      
%% ###################################################################################################

\subsection{Responsive Design / Mobile First}

Die Idee des Responsive Web Design wurde 2010 von Ethan Marcotte in einem Artikel \footnote{ \url{http://alistapart.com/article/responsive-web-design}} des Magazins A List Apart veröffentlicht. Hintergrund war, dass man nicht für jedes neue Gerät eine eigene Webseite erstellen musste. Marcotte schreibt von drei Faktoren, die ein responsive Design benötigt: Fluid grids, flexible images und media queries.

Fluid grids: Die Anordnung der Elemente passt sich dynamisch der Bildschirmgrösse an.
flexible images: Bilder haben keine feste Grösse sondern nutzen den ihnen zu Verfügung stehenden Platz optimal aus.
Media queries: sind die technische Basis für die beiden oberen Anforderungen. 

Die "Mobile Web Best Practices" des W3C \footnote{ \url{https://www.w3.org/TR/mobile-bp}} empfiehlt, dass sämtliche Informationen, die in der Desktop-Version zur Verfügung stehen, auch von der mobilen Seite aufgerufen werden können (One Web Design).

Mobile First ist ein Designkonzept, bei dem die optimale Darstellung einer Website auf mobilen Endgeräten oberste Priorität hat. Bei Mobile First beginnt der Designer mit dem Mobile-Design und arbeitet sich dann schrittweise zur grösseren Desktop-Version vor. Vorteil: Da bei Mobile First die Grafiken und interaktiven Elemente schon zu Beginn für die Mobilversion optimiert werden, wird automatisch sichergestellt, dass diese auf jeder anderen Screengrösse auch umsetzbar sind.

Mobile First wurde von Luke Wroblewski 2009 das erste Mal vorgeschlagen. Es geht darum sich beim Designen einer Webseite zu erst auf die Kernfunktion zu konzentireren und diese zielgerichtet umzusetzen. Die benötigt eine vertiefte Analyse der Zielgruppe.

Vorgehen:

\begin{itemize}  
\item Ziel der Webseite definieren
\item Zielgruppe definieren
\item Ziel/Erwartungen der Zielgruppe erfassen
\item Wo wird die Infrmation gelesen
\item Was sind die Bedürfnisse der Benutzer
\item Scribbles erstellen
\item Wireframes erstellen
\end{itemize}

\subsubsection*{Ziel der Wetterstation-Webseite}
Interessierten, die wichtigsten Wetterparameter in leicht verständlicher Art zu präsentieren
\subsubsection*{Zielgruppe}
Touristen, Bevölkerung, Sportler, Interessierte
\subsubsection*{Erwartungen der Zielgruppe}
einfach und schnell verständliche Information
\subsubsection*{Scribbles}
Papierzeichnungen als erste Entwürfe
\subsubsection*{Wireframes}
Layout, gezeichnet mit balsamiq, ohne genaue Designangaben.




%% ###################################################################################################
%%   Unterkapitel                                                                                                                                      
%% ###################################################################################################

\subsection{Barrierefreiheit}

Von einer verbesserten Zugänglichkeit profitieren nicht nur Menschen mit Einschränkungen. Es geht darum, die Webseite so zu gestalten, dass sie möglichst für alle Benutzergruppen zugänglich ist. Dazu zählen nicht nur Menschen mit Einschränkungen. Auch veraltete Technik oder aber der allerneueste Stand der Technik können zu Schwierigkeiten führen. Ein hoher Lärmpegel (z.B. in einer Fabrikhalle) oder der Zwang zur Stille (z.B. in einer Bibliothek), der keine akustische Ausgabe gestatten oder Lichtverhältnisse, die einen besonders hohen Kontrast erfordern beeinflussen die Bedienbarkeit einer Webseite. Eine von Microsoft beauftragte Studie~\cite{ForresterResearch2004E:Abilities} der \flqq Forrester Research Inc.\frqq schätzt, dass über 60 Prozent aller Computernutzer von Barrierefreiheit profitieren können und gemäss \flqq Interface Design\frqq ~\cite{ThesmannStephan2016ID:U} wird Barrierefreiheit bald Standard sein.

Die aktuellen Web Content Accessibility Guidelines\footnote{ \url{https://www.w3.org/TR/WCAG20/}} fordern die Einhaltung von vier Designprinzipien:

\begin{itemize}  
\item Prinzip 1: Wahrnehmbarkeit 
\item Prinzip 2: Bedienbarkeit
\item Prinzip 3: Verständlichkeit
\item Prinzip 4: Robustheit
\end{itemize}

Die Ziele dieser vier Prinzipien sind durch zwölf Richtlinien genauer spezifiziert. 

% Zu jeder Richtlinie geben die WCAG2 testbare Erfolgskriterien (Success Criteria) vor.

% Priorität 1 („Muss-Kriterien“): Webauftritte müssen alle A-Anforderungen erfüllen, weil es sonst für eine oder mehrere Benutzergruppen unmöglich wäre, auf die Information im Dokument zuzugreifen.

% Priorität 2 („Soll-Kriterien“): Die Erfüllung der AA-Anforderungen beseitigt signifikante Hindernisse und erleichtert einer oder mehreren Benutzergruppen den Zugriff auf Web-Dokumente.

% Priorität 3 („Kann-Kriterien“): Diese AAA-Anforderungen können erfüllt werden, um den Zugriff auf Web-Dokumente für eine oder mehrere Benutzergruppen zu erleichtern. Sind die Prioritäten 1 bis 3 erfüllt, erhält das Informationsangebot die Konformitätsstufe AAA.


Daraus ergeben sich folgende relevante Anforderungen für die Webseite der Wetterstation:


\subsubsection*{Wahrnehmbarkeit}
% Anforderung 1.1: Text-Alternativen 
% Anforderung 1.2: Zeitbasierte Medien -> nicht relevant
% Anforderung 1.3: Anpassbarkeit
% Anforderung 1.4: Unterscheidbarkeit
Alle Nicht-Text-Inhalte, die dem Benutzer präsentiert werden, haben eine Textalternative, die dem entsprechenden Zweck dient.
Für die Darstellung der Seite muss CSS verwendet werden. Die einzelenen Blöcke müssen sematisch korrekt bezeichnet werden (RL 1.3)
Alle Nicht-Text-Inhalte, die dem Benutzer präsentiert werden, haben eine Textalternative, die dem entsprechenden Zweck dient.
Farbe ist nicht das einzige visuelle Mittel, um Informationen zu vermitteln, eine Handlung anzuzeigen, eine Reaktion auszulösen oder ein visuelles Element zu unterscheiden. ext kann ohne Hilfsmittel bis zu 200 Prozent vergrössert werden, ohne dass der Inhalt oder die Funktionalität verloren geht. Die visuelle Darstellung von Text hat ein Kontrastverhältnis von mindestens 7:1.
Für die visuelle Darstellung von Textblöcken steht ein Mechanismus zur Verfügung, um folgendes zu erreichen: (Level AAA)
Vorder- und Hintergrundfarben können vom Anwender frei gewählt werden.
Die Breite beträgt nicht mehr als 80 Zeichen oder Glyphen (40 bei CJK).
Der Text ist links ausgerichtet
Der Zeilenabstand (Vorspann) ist innerhalb von Absätzen mindestens anderthalb mal so groß wie der Zeilenabstand, und der Absatzabstand ist mindestens 1,5 mal so groß wie der Zeilenabstand.
Die Größe des Textes kann ohne Hilfsmittel bis zu 200 Prozent verändert werden, ohne dass der Benutzer horizontal scrollen muss, um eine Textzeile in einem Vollbildfenster zu lesen.

Interessante Farb-Auswahlwerkzeuge:
Colorchecker\footnote{ \url{http://accessible-colors.com}} 
Colorsave\footnote{ \url{http://colorsafe.co}} 
Colorbrewer\footnote{ \url{http://colorbrewer2.org}} 


Problematisch ist die Kombination von Komplementärfarben, weil sie zu Flimmern führen können.~\cite{HellbuschJanEric2011Bvuu}

\subsubsection*{Bedienbarkeit}
% Anforderung 2.1: Zugänglichkeit per Tastatur
% Anforderung 2.2: Bereitstellung ausreichender Zeit
% Anforderung 2.3: Vermeidung von Anfällen
% Anforderung 2.4: Navigierbarkeit
Alle Funktionen des Inhalts sind über eine Tastaturschnittstelle bedienbar.
Webseiten enthalten nichts, was in einer Sekunde mehr als dreimal blinkt.
Der Zweck eines jeden Links kann aus dem Linktext allein oder aus dem Linktext zusammen mit seinem programmatisch festgelegten Linkkontext bestimmt werden. Es gibt mehr als eine Möglichkeit, eine Webseite innerhalb einer Gruppe von Webseiten zu lokalisieren. Überschriften und Labels beschreiben das Thema oder den Zweck. Jede Tastatur bedienbare Benutzeroberfläche hat eine Betriebsart, bei der die Tastaturfokusanzeige sichtbar ist. Informationen über den Standort des Benutzers innerhalb einer Reihe von Webseiten sind verfügbar. Abschnittsüberschriften dienen der inhaltlichen Gliederung. 



\subsubsection*{Verständlichkeit}
% Anforderung 3.1: Lesbarkeit
% Anforderung 3.2: Vorhersehbarkeit
% Anforderung 3.3: Hilfestellung bei der Eingabe

Die voreingestellte menschliche Sprache jeder Webseite kann programmgesteuert bestimmt werden. Die menschliche Sprache jeder Passage oder Phrase im Inhalt kann programmatisch bestimmt werden, mit Ausnahme von Eigennamen, Fachbegriffen, Wörtern unbestimmter Sprache und Wörtern oder Phrasen, die Teil der Umgangssprache des unmittelbar umgebenden Textes geworden sind. Ein Mechanismus zur Identifizierung der erweiterten Form oder Bedeutung von Abkürzungen steht zur Verfügung.  Wenn eine Komponente den Fokus erhält, löst sie keinen Kontextwechsel aus. Wenn ein Eingabefehler automatisch erkannt wird, wird das fehlerhafte Element identifiziert und der Fehler dem Benutzer in Textform beschrieben. Labels oder Anweisungen werden bereitgestellt, wenn der Inhalt eine Benutzereingabe erfordert.  Wird ein Eingabefehler automatisch erkannt und sind Korrekturvorschläge bekannt, so werden diese dem Anwender zur Verfügung gestellt.  Es steht eine kontextsensitive Hilfe zur Verfügung.



\subsubsection*{Robustheit}
% Anforderung 4.1: Kompatibilität
 In Inhalten, die mit Hilfe von Markup-Sprachen implementiert wurden, haben Elemente vollständige Start- und End-Tags, Elemente werden entsprechend ihrer Spezifikationen verschachtelt, Elemente enthalten keine doppelten Attribute und alle IDs sind eindeutig. 
 
 
 



%% ###################################################################################################
%%   Unterkapitel                                                                                                                                      
%% ###################################################################################################

\subsection{Vergleich Prognose/Ist Windgeschwindigkeit}

%% ###################################################################################################
%%   Unterkapitel                                                                                                                                      
%% ###################################################################################################

\subsection{Historische Daten}
\Diskussionspunkt{Fehleingaben verhindern, Sicherheit gegen Angriffe?}

%% ###################################################################################################
%%   Unterkapitel                                                                                                                                      
%% ###################################################################################################

\subsection{Anzeige Sturmwarndienst}
