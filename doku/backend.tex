\section{Datenbank / back end / cronjobs}

%% ###################################################################################################
%%   Unterkapitel                                                                                                                                                                              #
%% ###################################################################################################

\subsection{Übersicht}

In diesem Kapitel wird aufgezeigt welche Änderungen an der Datenbank vorgenommen werden und wo diese überall zum Einsatz kommen. Zudem wird aufgezeigt, wie diese zusätzlichen Anwendungen entwickelt wurden. Die Datenbank glich vor der Bearbeitung einem Datenfriedhof. Im jetzigen Zustand werden die Daten die schon geschrieben waren, sowie die neu geschrieben werden aktiv für folgende Aufgaben benutzt:\\
\begin{itemize}
\item API
\item Dreitage Rückschau der verschiedenen Sensoren
\item Historische Daten
\end{itemize}

Die Aufgabe der Datenbank ist es den Umgang der Daten einfacher zu gestalten, sei es in Bezug auf die Aufbereitung der historischen Daten oder aber auch die API, welche einfach erweiterbar sein soll.



%% ###################################################################################################
%%   Unterkapitel                                                                                                                                                                              #
%% ###################################################################################################
\subsection{Datenerfassung}
\subsubsection{Erfassung der Daten des Wetter-Transmitters}
\Diskussionspunkt{- Konfiguration von WeatherDisplay}\newline
\Diskussionspunkt{- Was passiert wenn DB-Eintrag nicht erstellt werden kann}\newline

\subsubsection{Einlesen von Pegel- und Strahlungsmesswerten (Strom to Ethernet)}
\Diskussionspunkt{- wie werden Daten abgerufen?}\newline
\Diskussionspunkt{- was passiert wenn keine Antwort kommt}\newline



%% ###################################################################################################
%%   Unterkapitel                                                                                                                                                                              #
%% ###################################################################################################

\subsection{Datenspeicherung}
\subsubsection{Warum eine Umstrukturierung der Datenbank?}
Die Datenbank ist zur Zeit vor der Umstellung chaotisch und die Namensgebung ist nicht sehr aufschlussreich. Aus diesem Grund wird die Datenbank umstrukturiert. Hierbei soll vor allem die Übersichtlichkeit und Datenverteilung im Vordergrund stehen. Die Daten werden zudem nicht verwendet. Dies ist sehr schade, da es auch für vergangene Daten Anwendungen gibt, bspw. könnten Benutzer zurück schauen um zu sehen wie das Wetter vor einem Jahr war. Hierfür wird eine Seite mit einer solchen Anwendung erstellt.

\subsubsection{Vorgaben}
Bei der Entwicklung einer Datenbank müssen verschiedene Punkte vorher klar sein. Zum einen welche Datenbank steht zur Verfügung und welches Datenmodell bildet diese. Der wichtigere Punkt der beiden Fragen ist welches Datenmodell wird benutzt denn es bestehen folgende Arten von Datenmodellen:
\begin{itemize}
\item relationale DB
\item hierarchisches Datenmodell
\item Netzwerkdatenmodell
\item Objekt relationale Datenbank
\end{itemize}

Im Falle der zu bearbeitenden Datenbank ist steht eine MariaDB mit einer relationalen Struktur zur Verfügung. Die relationale Datenstruktur heisst so, weil es in diesem Fall der Modellbildung sich überlappende Spalten geben kann.\cite{IntroductionToRelationalDatabases:MariaDB} Hiermit werden die verschiedenen Tabellen welche entwickelt werden auf beliebige Weise durch dieses Feld verknüpft. Im Falle der neuen Datenbank wird es die Spalte datetime sein, dazu aber später mehr.\\

\subsubsection{Methodik}
Nachdem die Vorgaben klar sind kann die Datenbank entwickelt werden. Hierbei kann es verschiedene Methoden geben. Im Falle dieser Datenbank wurde entschieden nach einer Methode mit vier verschiedenen Phasen zu nehmen. \cite{Datenbanken:GrundlagenUndEntwurf:VeikkoKrypczyk} Dieses Vorgehen sieht folgendermassen aus:
\begin{itemize}
\item Externe Phase (Ermittlung der Informationsstruktur)
\item Konzeptionelle Phase (ER-Modell)
\item Logische Phase (relationales Datenmodell)
\item Physische Phase (Erstellung des Datenmodell)
\end{itemize}

Diese vier Phasen führen vom Beginn der Enwicklung bis zum Endprodukt und werden im nächsten Kapitel erklärt sowie umgesetzt.
\subsubsection{Neugestaltung der Datenbank}
\textbf{Externe Phase}\\
In der externen Phase gilt es herauszufinden, welche Daten aus der realen Welt dargestellt werden sollten, sowie das Problem darzustellen. Das Problem bei dieser Arbeit sind die Menge an Daten, vor der Bearbeitung der Datenbankt fällten Stündlich 60 Einträge mit 65 Datenpunkte an. Dies ergab über die 3 Jahre an der Die Datenbank mehr oder weniger minütlich gefüllt wurde 1143847 Einträge. Dieser ganze Satz an Daten gilt es zu optimieren und die zukünftigen Einträge auf die wichtigsten zu minimieren. Dies ging auch aus der Diskussion mit dem Spezialisten hervor, welcher einige Ansätze, für eine neu Organisation aufgezeigt hat hervor. Die Umstrukturierung der Datenbank sieht folgende Massnahmen vor:
\begin{itemize}
\item Die Tabelle wx data welche die Daten des Wettertransmitters beinhaltet wird in ihrem jetzigen Zustand belassen, allerdings sollten die nicht benutzten Spalten entfernt werden, wile es viele Spalten welche entweder immer eine 0, 100 oder ein NULL beinhalten.
\item Tabelle tblwellen wird umbenannt in tblextsensors und umstrukturiert
\item Tabelle tblhistorical wird neu aufgebaut, sodass die Daten aus den beiden oben genannten Tabellen in Stundenwerte komprimiert werden.
\end{itemize}

Die neue Tabellenstruktur der wx data würde folgende Spalten beinhalten:
 \begin{itemize}
\item average windspeed
\item max average windspeed day
\item gust windspeed
\item max gust current day
\item wind direction
\item temperature
\item max daily temperature
\item min daily temperature
\item dew point temperature
\item current windchill
\item max windchill
\item min windchill
\item current humidex
\item min humidex
\item max humidex
\item heat index
\item max heat index
\item min heat index
\item outdoor humidity
\item barometer
\item barometer trend last hour
\item daily rainfall
\item rain rate
\item cloud height
\item icon type
\item forecast icon
\end{itemize}

Obwohl die in diesem Stadium die externe Sensoren noch nicht alle funktionieren oder genau definiert sind welche umgesetzt werden sollten die  Spalten für die Tabelle tblextsensors schon definiert sein:
\begin{itemize}
\item waterlevel
\item waveheight
\item watertemperature50cm
\item watertemperature100cm
\item radiation
\end{itemize}

Bei der historischen Tabelle tblhistorical, werden die Spalten aus den beiden obengenannten Tabellen zusammengesetzt.
\textbf{Konzeptionelle Phase}\\
Nachdem die externe Phase abgeschlossen ist und entschieden ist, welche Datenpunkte relevant für die API sowie die neue Webseite sind kann mit der konzeptionellen Phase begonnen werden. In dieser geht es darum die in der vorhergehenden Phase in ein Konzept zu überführen. Bei der Erstellung einer Datenbank besteht die Möglichkeit ein ER-Modell zu entwickeln. Ein ER-Modell zeigt die verschiedenen Beziehungen unter den Tabellen auf, welche Entitäten eine Datenbank hat und welche Attribute eine Datenbank beinhaltet. Um diese Punkte zu definieren muss klar sein welche Attribute eine Tabelle hat. Wird eine Tabelle mit allen Attribute erstellt oder mehrere logisch aufgeteilte Tabellen? Im Falle der Neukonzipierung der Wetterstation Datenbank sieht das ER-Modell wie in Bild() aus.\Diskussionspunkt{ER-Modell Bild}

\textbf{Logische Phase (relationales Datenmodell)}\\
Die Tabelle tblextsensors  enthält sie als Fremdschlüssel einen Datums und Zeitstempel. Die beiden Tabellen wx data und tblExtSensors werden über eine Tabelle mit verschiedenen Formaten von Datums bzw Zeitstempel als Primärschlüssel miteinander Kombiniert. Somit kann sichergestellt werden, dass die Datenbank in Zukunft weiter wachsen kann und die Tabelle wx data in ihrem jetzigen Zustand so belassen werden kann. Die Zusammenführung kann in einer abfrage mit dem Befehl LEFT OUTER JOIN umgesetzt werden. So kann kann eine Tabelle mit den aktuellen Werten nach Wunsch erstellt werden.
\Diskussionspunkt{relationales Datenmodell Bild}
\Diskussionspunkt{Fertigstellen}\\

Punkte zur Überlegung neuer Datenbankstruktur:
\begin{itemize}
\item Welche Daten?
\item In welchem Intervall?
\item Welche Tabellen? (Welche Daten zusammen?, eine grosse Tabelle?)
\item Tabellennamen?
\item
\end{itemize}

Datentypen angeschaut auf w3schools:
\begin{itemize}
\item Welcher Datumstyp?

\end{itemize}
\textbf{Physische Phase (Erstellung des Datenmodell)}

\Diskussionspunkt{Fertigstellen}\\

\subsection{Datenbanksicherheit}
\subsubsection{Speicherplatz}
Vor dass sich Gedanken um die Datensicherheit gemacht werden, sollten die die Bedingungen an Speicherplatz klar sein. Während der Laufzeit werden grosse Mengen an Daten in die Datenbank geschrieben, vor der Neukonzipierung werden täglich 1440 Datensätze gespeichert. Dies bedeutet jede Minute einen Datensatz. Ein Datensatz beinhaltet 65 Einträge, die gesamte für die Arbeit relevante Datenbank igwetter wettertest benötigt 323.17  Mb. Nur schon die Tabelle wx data, diese beinhaltet die Minutenwerte des Wettertransmitters, benötigt stand 1.3.18 311.94 Mb, daraus erfolgt das ein Datensatz ca 0.025 Mb benötigt. Für den Speicherplatz, welcher 50 Gb bietet, stellt dies kein Problem dar, da wenn man die Zahlen hochrechnet genügend Platz für die kommenden 45 Jahren.\\


\subsubsection{Angriffssicherheit}

Bei der Recherche nach Datenbanksicherheit taucht immer wieder das Wort Injection auf. Laut den OWASP top 10, eine Liste welche die wichtigsten Schwachstellen aufzeigt, ist die SQL-injection in 2017 auf dem Platz 1. Was ist den eigentlich SQL Injection? SQL injection ist eine Methode eine Datenbankabfrage so zu manipulieren, dass der Angreifer im schlimmsten Fall auf die gespeicherten Daten des Administators kommt. Ein anderes Beispiel wäre, dass der Angreifer an die Daten der Benutzer eines Online-Shops mit Kreditkartendaten oder ähnlichen sensitiven Daten kommt.
Weitere Fragen die auftauchen bei der Suche nach Datenbanksicherheit sind:
\begin{itemize}
\item Was für Arten von Daten beherbergt die Datenbank?
\item Hat es sensitive Daten?
\item Ist die Datenbank überhaupt ein potentielles Angriffsziel?
\item Wer sind die Benutzer der Datenbank?
\end{itemize}

Bei der Wetterstation Arbon, bestehen keine persönliche oder Sicherheitsrelevante Daten in der Datenbank. Somit ist diese aus sicht eine potentiellen Angriffes eher uninteressant. Dennoch sollten die Daten hinreichend geschützt sein, da es vorallem schade wäre wenn die Daten verloren gingen. Andererseits aber auch das kein Missbrauch vom Server gemacht werden kann.

Die Zugriffe auf die Datenbank sind so gestaltet, dass darauf nur Server seitig zugegriffen wird. Die Darstellung der Anzeigen auf der Webseiten werden über die API erstellt. Die API wiederum ist so aufgebaut, dass ein PHP-Skript auf dem Server die Daten aus der Datenbank abgreift und sie richtig formatiert. Somit kann sichergegangen werden, dass keine SQL-Infection möglich ist. Die einzige Seite welche direkt auf die Datenbank zugreift ist die historische Datenbank. Hier wird jedoch das Risiko so unterbunden, dass mit der Openversion von Tableau gearbeitet wird, hierdurch kann sichergegangen werden das Tableau den direkten Zugriff durch die Benutzer verweigert und die Daten so sicher sind. Somit sind keine weiteren Sicherheitsmassnahmen notwendig.

\subsubsection{Backup}

Um die Datenbank auch gegen einen allfälligen Datenverlust zu sichern ist ein Backup von wichtiger Bedeutung. Deswegen ist das Backup ein weiterer Aspekt gegen den Verlust von Daten. Hierbei sind folgende Fragen relevant:
\begin{itemize}
\item Welche Daten sollen gesichert werden
\item Wie oft sollten die Daten gesichert werden?
\item Wie bzw. wo sollte das Backup gelagert werden?
\end{itemize}

Da alle Daten in der DB gleich relevant sind sollten auch alle so behandelt werden und im Backup vorhanden sein. Dabei muss aber entschieden werden, ob ein tägliches Backup Sinn machen würde. Bezüglich des Speicherplatzes und des Aufwands, da die Wetterstation von einem Verein betrieben wird, ist es wichtig den Aufwand mit dem Ertrag zu vergleichen.Zusätzlich sollte das Backup nicht auf dem Server des Providers gespeichert werden, da wenn etwas schief geht mit dem Server die Daten auch weg sind. Deswegen ist es wichtig auch ein "externes" Backup zu erstellen. Hostpoint bietet für das Backup verschiedene Varianten:
\begin{itemize}
\item Cronjob
\item Backup auf Knopfdruck
\item Kostenpflichtige Wiederherstellung aller Daten
\end{itemize}


Um den Aufwand klein zu halten wird empfohlen ein monatliches Backup mit der Backup Funktion auf Knopfdruck zu erstellen und dieses auf einer Festplatte zu speichern. Sollte trotzdem mal das Backup nicht funktionierten oder vergessen gegangen sein, kann auf den Hostpoint-Service züruck gegriffen werden. Diese erstellen selber ein tägliches Backup, welches für 50 Franken wieder eingespielt werden kann. Somit können sich, davon ausgehend, dass Hostpoint einen guten Job macht Kosten für bspw. ein Cloudaccount gespart werden.

%% ###################################################################################################
%%   Unterkapitel                                                                                                                                                                              #
%% ###################################################################################################

\subsection{Datenarchivierung}

Bei der Datenarchivierung geht es vorallem darum, wie mit den vergangenen Daten umgegangen wird. Da nach der Umgestaltung eine neue Tabelle tblhistorical entstanden ist, bestehen zwei möglichkeiten um mit den Daten umzugehen. Allerdings ist eine Anforderung an die tblhistorical, dass diese nur Stundenwerte beinhaltet. Konkret bedeutet dies, dass aus den Werten einer Stunde der Median und die Extremwerte extrahiert werden. Auf die zwei Möglichkeiten bezogen bedeutet dies folgendes: \\
Die erste Variante ist die historische Tabelle stündlich zu füllen mit einem Datensatz und bei den beiden Tabellen tblwettertransmitter, sowie tblextsensors die Werte wieder zu löschen. Dies würde bedeuten, dass schlussendlich nur die historische Tabelle vorhanden ist um in die Vergangenheit zu schauen.\\
Die zweite Variante ist, die historische Tabelle stündlich zu füllen, jedoch die Tabelle des Wettertransmitters und der externen Sensoren gefüllt zu lassen mit den historischen Minuten Werte.\\
Da beim Server der Speicherplatz kein ausschlaggebender Punkt ist, wurde für die zweite Variante entschieden. Diese hat zudem noch den Vorteil, dass die Minuten Werte der Vergangenheit noch vorhanden sind, sollte eine zusätzliche Anwendung hierfür entstehen.


%% ###################################################################################################
%%   Unterkapitel                                                                                                                                                                              #
%% ###################################################################################################

\subsection{Funktionsüberwachung mit Mail-Service}

Um die Funktion der Software zu gewährleisten, sollte bei einem Absturz der Verantwortliche für die IG bei der IG benachrichtigt werden. Folgende Funktionen müssen bei einem Absturz eine Meldung geben.
\begin{itemize}
\item Einlesen Sensordaten extern
\item Einlesen Wettertransmitter Daten
\item Erstellung der stündlichen historischen Daten
\end{itemize}
Die Aufgezählten Funktionen werden alle bis auf das Einlesen der Wettertransmitter Daten über einen Cronjob ausgeführt. Für die Cronjobs bietet Hostpoint den Service, dass bei einem print() die Ausgaben per Mail an eine bestimmte Mailadresse gesendet werden. Für das Einlesen der externen Sensordaten sieht der Mailservice folgendermassen aus. Können einer der Webservices vom \Diskussionspunkt{A/E Wandler} nicht erreicht werden, wird die folgende Mail generiert:\\
Es ist ein Problem mit (der Temperatur, dem Pegel, dem Strahlungssensor) aufgetreten. Exception: ...\\
Je nach Sensor wird dieser genannt und das Problem welches aufgetreten ist. Für die Erstellung der historischen Daten und dem einlesen der Wettertransmitter Daten sieht die Lösung ähnlich aus. Jedoch werden diese beiden im gleichen Cronjob, dem erstellen der historischen Daten kontrolliert. Zuerst wird kontrolliert ob alle 60 Einträge der letzen Stunde vorhanden sind. Ist dies nicht der Fall, würde es bedeuten, dass das WeatherDisplay, welches die Daten des Transmitters aufbereitet, abgestürzt ist und neu gestartet werden muss. Das zweite was in diesem Cronjob kontrolliert wird, ist dass Kontrolliert werden muss ob die Daten in die historische Tabelle geschrieben sind. Die beiden Meldungen sehen dann so aus:\\
Bitte starte das WeatherDisplay neu, es wurden nur (Anzahl Datensätze) Daten geschrieben.\\
Die historischen Daten können nicht geschrieben werden, es besteht folgendes Problem (Exception).\\

Um diese Funktionen zu erstellen wird Gebrauch gemacht vom try, except Verfahren in Python. Zu Beginn wird der Code im try ausgeführt, tretet keine exception auf wird das except übersprungen und der anschliessende Code ausgeführt. Tritt aber während dem try eine exception auf, wird der Code unterbrochen und der im except weitergeführt. Anschliessend wird der Code nach dem Exception Handling ausgeführt.\cite{ThePythonTutorial8.ErrorsAndExceptions:Python}

%% ###################################################################################################
%%   Unterkapitel                                                                                                                                                                              #
%% ###################################################################################################
\subsection{Cron-Jobs}
\Diskussionspunkt{- Printscreen cronjobs-Konfig auf Hostpoint}\newline


%% ###################################################################################################
%%   Unterkapitel                                                                                                                                                                              #
%% ###################################################################################################
\subsection{Problematik Zeitumstellung}



