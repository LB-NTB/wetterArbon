\chapter{Spezifikation / Pflichtenheft}

Anforderungen nach dem SMART-Prinzip formulieren:

\begin{itemize}  
\item S: Spezifisch 
\item M: Messbar
\item A: Akzeptiert
\item R: Realistisch
\item T: Terminierbar
\end{itemize}


\section{User Interface}
\Diskussionspunkt{Responsive Design, ...}

\section{Datenbank}
\Diskussionspunkt{alle Daten in Datenbank erfassen (Wassertemperatur, Pegel), Webseite um Abfragen zu tätigen, Daten nach einem Jahr verringern}

\section{Sensoren}
\Diskussionspunkt{Randbedingungen, Kosten, Genauigkeit, Pegelsensor}


\Diskussionspunkt{Test-Tabelle:}
\begin{table}[]
%\centering
\caption{My caption}
\label{my-label}
\begin{tabular}{|l|l|l|l|l|}
\hline
ID      & \multicolumn{3}{l|}{Titel der Anforderung}            &   Typ\\ \hline
1        & \multicolumn{3}{l|}{Responsive Design}            &  FA \\ \hline
\multicolumn{5}{|l|}{blabla Beschreibung}                         \\ \hline
\multicolumn{5}{|l|}{blabal Test}                         \\ \hline
\multicolumn{2}{|l|}{MUSS} & \multicolumn{3}{l|}{Reserve} \\ \hline
\end{tabular}
\end{table}



\Diskussionspunkt{https://www.tablesgenerator.com}


\section{Vorhersage}
\Diskussionspunkt{...}

\section{Webcam}
\Diskussionspunkt{Warteschlange, evtl. sektorweise Zoombeschränung}


\section{Nutzeranalyse}

Welches sind die Nutzer und was sind deren Bedürfnisse


\section{Funktionale Anforderungen}
\Diskussionspunkt{Daten, Funktionen, Verhalten, Fehlerreaktionen}

\begin{usecase}

  \addheading{Nummer}{Beschreibung} 
  \addrow{/FA10/}{Temperaturanzeige in Grad und Fahrenheit}
  \addrow{/FA20/}{Windgeschwindigkeitsanzeige in Knoten, Km/h, m/s, mph, Bft  }
  \addrow{/FA30/}{Luftdruckanzeige in hPa, mmHg, kPa, inHg, mb, }
  \addrow{/FA40/}{Windrichtung }
  \addrow{/FA50/}{Niederschlagsmenge in mm } 
\end{usecase}


\section{Nicht-Funktionale Anforderungen}
\Diskussionspunkt{Leistungsanforderungen, Qualität, Randbedingungen}

\begin{usecase}
  \addheading{Nummer}{Beschreibung} 
  \addrow{/FA10/}{Webseite soll im responsive Design erstellt sein}
  \addrow{/FA20/}{Webseite soll auch für Menschen mit beeinträchtigungen zur Verfügung stehen}
  \addrow{/FA30/}{Webseite soll mit HTML5 erstellt sein}
  \addrow{/FA30/}{Webseite soll mit JavaScript erstellt sein}
\end{usecase}


