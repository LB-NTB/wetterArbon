\section{Barrierefreiheit}

Denn von einer verbesserten Zugänglichkeit profitieren nicht nur Menschen mit Einschränkungen, sondern auch die Informationsanbieter durch die Ansprache von Millionen zusätzlicher potenziellen Benutzer. Als Bonus kommt eine verbesserte Indizierung durch Suchmaschinen und infolgedessen eine bessere Position im Suchmaschinen-Ranking hinzu.

Im Kern geht es also darum, die Webapplikation so zu gestalten, dass sie möglichst für alle Benutzergruppen zugänglich ist. Dazu zählen nicht nur Menschen mit schweren und permanenten Einschränkungen, Kranke und Verletzte, funktionale Analphabeten und Legastheniker sowie die steigende Zahl an Seniorenauch veraltete Technik oder aber der allerneueste Stand der Technik (mobile Endgeräte wie etwa Tablet PC oder Smartphone) kann zu Schwierigkeiten führen.

hohen Lärmpegel (z. B. in einer Fabrikhalle) oder Zwang zur Stille (z. B. in einer Bibliothek) keine akustische Ausgabe gestatten, die Lichtverhältnisse einen besonders hohen Kontrast erfordern

Eine von Microsoft beauftragte Studie ~\cite{ForresterResearch2004E:Abilities} der \flqq Forrester Research Inc.\frqq schätzt, dass über 60 Prozent aller Computernutzer von Barrierefreiheit profitieren können. 


Gemäss \flqq Interface Design\frqq ~\cite{ThesmannStephan2016ID:U} wird Barrierefreiheit bald Standard sein.



\subsection{Web Content Accessibility Guidelines (WCAG)}

Die aktuellen Web Content Accessibility Guidelines 2.0 (WCAG2) fordern die Einhaltung von vier technikunabhängigen Designprinzipien:

\begin{itemize}  
\item Prinzip 1: Wahrnehmbarkeit 
\item Prinzip 2: Bedienbarkeit
\item Prinzip 3: Verständlichkeit
\item Prinzip 4: Robustheit
\end{itemize}

Die Ziele dieser vier Prinzipien sind durch zwölf Richtlinien (Guidelines) genauer spezifiziert. Zu jeder Richtlinie geben die WCAG2 testbare Erfolgskriterien (Success Criteria) vor.

Priorität 1 („Muss-Kriterien“): Webauftritte müssen alle A-Anforderungen erfüllen, weil es sonst für eine oder mehrere Benutzergruppen unmöglich wäre, auf die Information im Dokument zuzugreifen.

Priorität 2 („Soll-Kriterien“): Die Erfüllung der AA-Anforderungen beseitigt signifikante Hindernisse und erleichtert einer oder mehreren Benutzergruppen den Zugriff auf Web-Dokumente.

Priorität 3 („Kann-Kriterien“): Diese AAA-Anforderungen können erfüllt werden, um den Zugriff auf Web-Dokumente für eine oder mehrere Benutzergruppen zu erleichtern. Sind die Prioritäten 1 bis 3 erfüllt, erhält das Informationsangebot die Konformitätsstufe AAA.

\subsection{Relevante Anforderungen für die Webseite der Wetterstation}

\subsubsection*{Prinzip 1: Wahrnehmbarkeit}
Anforderung 1.1: Text-Alternativen
Anforderung 1.2: Zeitbasierte Medien
Anforderung 1.3: Anpassbarkeit
Anforderung 1.4: Unterscheidbarkeit


\subsubsection*{Prinzip 2: Bedienbarkeit}
Anforderung 2.1: Zugänglichkeit per Tastatur
Anforderung 2.2: Bereitstellung ausreichender Zeit
Anforderung 2.3: Vermeidung von Anfällen
Anforderung 2.4: Navigierbarkeit


\subsubsection*{Prinzip 3: Verständlichkeit}
Anforderung 3.1: Lesbarkeit
Anforderung 3.2: Vorhersehbarkeit
Anforderung 3.3: Hilfestellung bei der Eingabe

\subsubsection*{Prinzip 4: Robustheit}
Anforderung 4.1: Kompatibilität

