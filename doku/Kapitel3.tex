\section{Barrierefreiheit}

Denn von einer verbesserten Zugänglichkeit profitieren nicht nur Menschen mit Einschränkungen, sondern auch die Informationsanbieter durch die Ansprache von Millionen zusätzlicher potenziellen Benutzer. Als Bonus kommt eine verbesserte Indizierung durch Suchmaschinen und infolgedessen eine bessere Position im Suchmaschinen-Ranking hinzu.

Im Kern geht es also darum, die Webapplikation so zu gestalten, dass sie möglichst für alle Benutzergruppen zugänglich ist. Dazu zählen nicht nur Menschen mit schweren und permanenten Einschränkungen, Kranke und Verletzte, funktionale Analphabeten und Legastheniker sowie die steigende Zahl an Seniorenauch veraltete Technik oder aber der allerneueste Stand der Technik (mobile Endgeräte wie etwa Tablet PC oder Smartphone) kann zu Schwierigkeiten führen.

hohen Lärmpegel (z. B. in einer Fabrikhalle) oder Zwang zur Stille (z. B. in einer Bibliothek) keine akustische Ausgabe gestatten, die Lichtverhältnisse einen besonders hohen Kontrast erfordern

Eine von Microsoft beauftragte Studie ~\cite{ForresterResearch2004E:Abilities} der \flqq Forrester Research Inc.\frqq schätzt, dass über 60 Prozent aller Computernutzer von Barrierefreiheit profitieren können. 


Gemäss \flqq Interface Design\frqq ~\cite{ThesmannStephan2016ID:U} wird Barrierefreiheit bald Standard sein.







\subsection{Web Content Accessibility Guidelines (WCAG)}

Die aktuellen Web Content Accessibility Guidelines 2.0 (WCAG2) fordern die Einhaltung von vier Designprinzipien:

\begin{itemize}  
\item Prinzip 1: Wahrnehmbarkeit 
\item Prinzip 2: Bedienbarkeit
\item Prinzip 3: Verständlichkeit
\item Prinzip 4: Robustheit
\end{itemize}

Die Ziele dieser vier Prinzipien sind durch zwölf Richtlinien (Guidelines) genauer spezifiziert. Zu jeder Richtlinie geben die WCAG2 testbare Erfolgskriterien (Success Criteria) vor.

Priorität 1 („Muss-Kriterien“): Webauftritte müssen alle A-Anforderungen erfüllen, weil es sonst für eine oder mehrere Benutzergruppen unmöglich wäre, auf die Information im Dokument zuzugreifen.

Priorität 2 („Soll-Kriterien“): Die Erfüllung der AA-Anforderungen beseitigt signifikante Hindernisse und erleichtert einer oder mehreren Benutzergruppen den Zugriff auf Web-Dokumente.

Priorität 3 („Kann-Kriterien“): Diese AAA-Anforderungen können erfüllt werden, um den Zugriff auf Web-Dokumente für eine oder mehrere Benutzergruppen zu erleichtern. Sind die Prioritäten 1 bis 3 erfüllt, erhält das Informationsangebot die Konformitätsstufe AAA.

\subsection{Relevante Anforderungen für die Webseite der Wetterstation}

\subsubsection*{Prinzip 1: Wahrnehmbarkeit}
Anforderung 1.1: Text-Alternativen 
-> für alle Anzeigegrafiken und Bilder
Anforderung 1.2: Zeitbasierte Medien 
-> Film-Aufnahme
Anforderung 1.3: Anpassbarkeit
Anforderung 1.4: Unterscheidbarkeit


\subsubsection*{Prinzip 2: Bedienbarkeit}
Anforderung 2.1: Zugänglichkeit per Tastatur
Anforderung 2.2: Bereitstellung ausreichender Zeit
Anforderung 2.3: Vermeidung von Anfällen
Anforderung 2.4: Navigierbarkeit


\subsubsection*{Prinzip 3: Verständlichkeit}
Anforderung 3.1: Lesbarkeit
Anforderung 3.2: Vorhersehbarkeit
Anforderung 3.3: Hilfestellung bei der Eingabe

\subsubsection*{Prinzip 4: Robustheit}
Anforderung 4.1: Kompatibilität


\subsection{Accessible Rich Internet Applications (WAI-ARIA) 1.1 -> Semantik}

WAI-ARIA ist eine technische Spezifikation, die ein Framework für die Verbesserung der Zugänglichkeit und Interoperabilität von Web-Inhalten und -Anwendungen bietet.

Menschen mit bestimmten Arten von Behinderungen nutzen assistive Technologien (AT), um mit Inhalten zu interagieren. Assistive Technologien können die Präsentation von Inhalten in ein für den Benutzer besser geeignetes Format umwandeln und es dem Benutzer ermöglichen, auf unterschiedliche Weise zu interagieren. Um dies effektiv zu bewerkstelligen, muss die Software die Semantik der Inhalte verstehen. Semantik ist die Wissenschaft der Bedeutung; in diesem Fall wird sie verwendet, um Rollen, Zustände und Eigenschaften zuzuweisen, die für Benutzeroberflächen und Inhaltselemente gelten, wie sie ein Mensch verstehen würde. Rollen sind eine gemeinsame Eigenschaft von accessibility APIs für die Zugänglichkeit, die von assistiven Technologien verwendet werden, um dem Benutzer eine effektive Präsentation und Interaktion zu ermöglichen.

Rollen sind Elementtypen und ändern sich nicht mit der Zeit oder Benutzeraktionen. Rolleninformationen werden von assistiven Technologien durch Interaktion mit dem User-Agent verwendet, um eine normale Verarbeitung des angegebenen Elementtyps zu ermöglichen. Zustände und Eigenschaften werden verwendet, um wichtige Attribute eines Elements zu deklarieren, die die Interaktion beeinflussen und beschreiben. 

Schlagwörter:
accessibility APIs
WAI-ARIA Roles
WAI-ARIA States and Properties

Roles:
Hauptindikator des Typs. Diese semantische Assoziation ermöglicht es den Werkzeugen, die Interaktion mit dem Objekt in einer Weise darzustellen und zu unterstützen, die mit den Erwartungen der Benutzer an andere Objekte dieses Typs übereinstimmt.

Properties:
Attribute, die für die Beschaffenheit eines bestimmten Objekts wesentlich sind oder einen mit dem Objekt verknüpften Datenwert repräsentieren. Eine Änderung einer Eigenschaft kann die Bedeutung oder Präsentation eines Objekts erheblich beeinflussen.


States:
Ein Zustand (state) ist eine dynamische Eigenschaft, die Eigenschaften eines Objekts ausdrückt, die sich als Reaktion auf Benutzeraktionen oder automatisierte Prozesse ändern können. Zustände haben keinen Einfluss auf die essentielle Natur des Objekts, sondern stellen Daten dar, die mit dem Objekt verbunden sind, oder Möglichkeiten der Benutzerinteraktion.

https://www.w3.org/TR/wai-aria/#ua_noninterference
