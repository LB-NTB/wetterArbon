\section{API}
\Diskussionspunkt{Einlesen API}
Beim Einlesen in das Thema API ist es wichtig sich vorher Fragen zu stellen, damit nicht darauflos rechechiert wird. Folgende Fragen sollten zu am Ende des Einlesens vor der Konzeptionierung beantwortet sein:
\begin{itemize}
\item Was ist eine API?
\item Wie wird eine API entwickelt?
\item Was ist wichtig zu beachten bei der Entwicklung?
\item Was ist state of the Art?
\item 
\end{itemize}

\subsection{Was ist eine API?}
Eine API überbrückt die Schnittstellen zwischen verschiedenen Software teilen und strukturiert die dabei anfallende Datenübergabe dazwischen. Die API ermöglicht es Software zu modularisieren und die Kommunikation zu vereinfachen. Konkret heisst das, die einzelnen Programmteile werden voneinander abgekapselt und kommunizieren nur über die Festgelegte API. Der Vorteil hierbei ist, das die API veröffentlicht werden kann und somit auch anderen die Möglichkeit gegeben werden kann die angebotenen Dienste mittels der API zu erhalten. Nicht zuletz ist auch das Testen ein wichtiger Punkt, da alle Programmteile nur über die API miteinander kommunizieren sollten muss die Software auch nur in der Zusammenarbeit mit der API getestet werden. 

http://www.omkt.de/api/

\subsection{Was ist state of the Art?}
Heutzutage wird darauf geschaut, dass die API nach der RESTful standard entwickelt wird. Da taucht die Frage auf was ist RESTful? Ein kleiner Exkurs. 