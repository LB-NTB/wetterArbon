\section{Erweiterungen}
In diesem Kapitel geht es nicht um die Verbesserung von bestehenden Problemen, sondern um die Erweiterung des Funktionsumfangs der Wetterstation Arbon. Es ist eine Auflistung möglicher Erweiterungen.

% Benachrichtigungen
\subsection{Individueller Benachrichtigungs-Service}
Mit einem Benachrichtigungs-Service soll dem Benutzer die Möglichkeit gegeben werden, zeitnah über Wetteränderungen informiert zu werden und somit keine Warnung oder sein perfektes Segelwetter zu verpassen. Dafür wurden drei verschiedene Möglichkeiten ausgewählt und mit der Nutzwertanalyse ausgewertet. Ziel bei allen Möglichkeiten ist es, dass der Benutzer die Möglichkeit hat Alarmkriterien selbst zu bestimmen. Werden die gewählten Alarmkriterien erreicht bzw. wird eine Sturmwarnung herausgegeben, wird der Benutzer benachrichtigt. Für die Evaluierung der Notifications wurde eine Nutzwertanalyse (Tabelle \ref{table:nutzwertanalyse}) erstellt. Dies ist eine gute Möglichkeit, um verschiedene Lösungsansätze zu bewerten. Der Nachteil hierbei ist jedoch, dass die Bewertung sehr subjektiv ist. Aus der Nutzwertanalyse geht hervor, dass die Benachrichtigung per E-Mail und Facebook Messenger die Lösungsansätze mit der höchsten Punktzahl sind. Der grösste Vorteil der beiden möglichen Lösungen ist, dass sie kostenlos sind. Der Nachteil an Facebook Messenger ist, dass nicht davon ausgegangen werden kann, dass jeder Benutzer ein Facebookprofil hat. 

\begin{table}
\begin{center}
\begin{tabular}{ |p{3.5cm}||p{1.1cm}|p{2cm}|p{1.7cm}|p{2.3cm}|p{1.4cm}|}
 \hline
 \multicolumn{6}{|c|}{Nutzwertanalyse} \\
 \hline
	Möglichkeiten & Kosten & Einfachheit & Aufwand & Anpassbarkeit & Support\\
 \hline
	SMS & 1 & 4 & 3 & 3 & 5\\
	E-Mail & 5 & 4 & 5 & 5 & 1\\
	FacebookMessenger & 5 & 4 & 3 & 4 & 1\\
 
\hline
\end{tabular}
\end{center}
\caption{Nutzwertanalyse verschiedener Notifikations-Möglichkeiten}
\label{table:nutzwertanalyse}
\end{table}


% Windprognose-Genauigkeit
\subsection{Überprüfung der Windprognose-Genauigkeit }
Es gibt diverse Anbieter von Windprognosen für den Bodensee wie zum Beispiel Windfinder\footnote{ \url{https://www.windfinder.com/forecast/arbon}} und SRF Meteo\footnote{ \url{https://www.srf.ch/meteo/surf-und-segelwetter/detail/06621}}. Vorhersagen sind Extrapolationen von Wettermodell-Berechnungen und mit gewissen Unsicherheiten behaftet. Interessant ist nun zu wissen, wie gut die Windvorhersagen mit den Wind-Messwerten der Wetterstation Arbon übereinstimmen. Während der Bachelor-Arbeit soll eine Vergleichsgrafik erstellt werden, welche die Vorhersage den Messwerten gegenüber stellt.


% Wellenhöhe
\subsection{Berechnung und Darstellung der Wellenhöhe}
Sobald ein funktionstüchtiger Pegelsensor installiert ist, können die Pegeldaten auch für andere Zwecke verwendet werden, zum Beispiel zur Berechnung der Wellenhöhe. Dies ist insbesondere für Ruderer und Kanu-Fahrer interessant.


% Schichtung Wassertemperatur
\subsection{Verlauf der Wassertemperatur in Abhängigkeit der Tiefe}
Die Wetterstation Arbon verfügt über mehrere Temperatursensoren, die im Abstand von 50 Zentimeter die Wassertemperatur messen. Die Idee ist, die Temperaturschichtung des Wassers bestimmen zu können.


% API
\subsection{Schnittstelle zu den aktuellen Messwerten (API)}
Die Wetterstation Arbon misst die Lufttemperatur und die Wassertemperatur des Bodensees. Die Badi Arbon, welche ca. einen Kilometer von der Wetterstation entfernt ist, zeigt auf ihrer Infotafel und auf ihrer Webseite\footnote{ \url{https://www.schwimmbad-arbon.ch}} ebenfalls die Luft- und Wassertemperatur an, bezieht diese aber von \textit{openWeatherMap}, welche die Temperaturen von Stationen aus Zürich und Friedrichshafen interpoliert. 
\newline

\noindent
%\subsection*{Problem: Unterschiedliche Werte für Luft- und Wassertemperatur}
Dass die gemessenen Werte der Wetterstation nicht mit den interpolierten Werten von Zürich und Friedrichshafen übereinstimmt, ist nicht verwunderlich. Für die Bevölkerung ist die Differenz jedoch unverständlich. Die Stadt Arbon möchte deshalb, dass die Badi die Messwerte der Wetterstation nutzen kann.
\newline

\noindent
%\subsection*{Lösungsansatz}
Die aktuellen Werte der Wetterstation sollen über ein REST Web-API von Dritten, wie zum Beispiel der Badi Arbon, abgerufen werden können. Das heute am meisten verwendete Format dafür ist JSON.

